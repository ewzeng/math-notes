\section{Wigner's Semi-Circular Law}

Define a Wigner matrix ensemble to be a random matrix ensemble (a.k.a. distribution) of Hermitian matrices such that:
\begin{itemize}
    \item The non-diagonal entries are iid random variables with mean zero and unit variance.
    \item The diagonal entries are iid random variables with bounded mean and variance.
\end{itemize}

\subsection{Normalization}
The Bai-Yin theorem states that as a random $n \times n$ Hermitian matrix $H$ increases in size, then almost surely one has
\[
    \limsup_{n \to \i} \frac{\|H\|}{\sqrt{n}} \leq 2.
\]
As the spectral radius of $H$ is $\|H\|$, this implies that the spectrum of $H$ will ``increase'' at the rate of $O(\sqrt{n})$. Thus, it is natural to work with the normalized matrix $H/\sqrt{n}$.

\subsection{Semi-circular Law}
Wigner's semi-circular law concerns the asymptotic distribution of eigenvalues
\[
    \l_i\left( \frac{H}{\sqrt{n}}\right)
\]
of a random Wigner matrix $H$ as $n \to \i$. It states that the density of eigenvalues converges to the semi-circular shape described by:
\[
    \mu'_{sc}(x) = \frac{1}{2\pi}(4 - |x|^2)^{1/2}_+dx
\]
where $(x)_+ = \max(x, 0)$.

The proof of the semi-circular law uses the moment method.
