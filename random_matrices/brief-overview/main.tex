
\documentclass[12pt, letterpaper]{article}
\usepackage[utf8]{inputenc}
\usepackage{amsmath,amssymb}
\usepackage{xcolor} %For colored text
\usepackage{parskip}

%Title and margin formatting
\usepackage[left=2cm,top=3cm,bottom=3cm,right=2cm]{geometry} %Customize margins.
\usepackage{titling} %Customize the position of the title.
\setlength{\droptitle}{-50pt} %Raise the position of the title.

%Macros for Greek Letters
\renewcommand{\a}{\alpha}
\renewcommand{\b}{\beta}
\renewcommand{\d}{\delta}
\newcommand{\D}{\Delta}
\newcommand{\e}{\varepsilon}
\newcommand{\g}{\gamma}
\newcommand{\G}{\Gamma}
\renewcommand{\l}{\lambda}
\renewcommand{\L}{\Lambda}
\newcommand{\s}{\sigma}
\renewcommand{\th}{\theta}
\renewcommand{\o}{\omega}
\renewcommand{\O}{\Omega}
\renewcommand{\S}{\Sigma}
\renewcommand{\t}{\tau}
\newcommand{\var}{\varphi}
\newcommand{\z}{\zeta}

%Macros for math cal letters
\newcommand{\cA}{{\mathcal A}}
\newcommand{\cB}{{\mathcal B}}
\newcommand{\cC}{{\mathcal C}}
\newcommand{\cD}{{\mathcal D}}
\newcommand{\cE}{{\mathcal E}}
\newcommand{\cF}{{\mathcal F}}
\newcommand{\cH}{{\mathcal H}}
\newcommand{\cI}{{\mathcal I}}
\newcommand{\cK}{{\mathcal K}}
\newcommand{\cL}{{\mathcal L}}
\newcommand{\cM}{{\mathcal M}}
\newcommand{\cN}{{\mathcal N}}
\newcommand{\cO}{{\mathcal O}}
\newcommand{\cP}{{\mathcal P}}
\newcommand{\cS}{{\mathcal S}}
\newcommand{\cT}{{\mathcal T}}
\newcommand{\cU}{{\mathcal U}}
\newcommand{\cV}{{\mathcal V}}
\newcommand{\cW}{{\mathcal W}}
\newcommand{\cY}{{\mathcal Y}}

%Macros for blackboard bold letters
\newcommand{\bZ}{{\mathbb Z}}
\newcommand{\bR}{{\mathbb R}}
\newcommand{\bC}{{\mathbb C}}
\newcommand{\bT}{{\mathbb T}}
\newcommand{\bN}{{\mathbb N}}
\newcommand{\bQ}{{\mathbb Q}}
\newcommand{\bF}{{\mathbb F}}

%Other macros
\renewcommand{\i}{\infty}

\title{An Overview of Basic Random Matrix Theory}
\author{Edward Zeng}

\begin{document}

\maketitle
\begin{center}
    [Based on the lecture notes of Tiago Pereira]
\end{center}

% Start notes
\section{Motivation for Random Matrix Theory}
Computing the energy spectrum of a complicated atom (like uranium) requires solving the eigenvalue problem
\[
    H\psi = E\psi.
\]
There are two problems with this:
\begin{itemize}
    \item The quantum Hamiltonian $H$ is complicated to write down.
    \item With such complicated $H$, the equation is too difficult to solve.
\end{itemize}

The solution is to consider a random $N \times N$ Hermitian matrix instead of $H$, and study the eigenvalue distribution as $N \to \i$ (as $H$ is infinite-dimensional).
\begin{itemize}
    \item By random $N \times N$ Hermitian matrix, we mean a Hermitian matrix with entries independently chosen from nice probability distributions.
\end{itemize}
Surprisingly, this method gives the general structure of the spectrum of any nuceli too complicated to be understood in detail. The reason is because the spectrum of an arbitrary operator turns out to be well described by the expected value of the spectrum of a random operator.
\section{Wigner's Semi-Circular Law}

Define a Wigner matrix ensemble to be a random matrix ensemble (a.k.a. distribution) of Hermitian matrices such that:
\begin{itemize}
    \item The non-diagonal entries are iid random variables with mean zero and unit variance.
    \item The diagonal entries are iid random variables with bounded mean and variance.
\end{itemize}

\subsection{Normalization}
The Bai-Yin theorem states that as a random $n \times n$ Hermitian matrix $H$ increases in size, then almost surely one has
\[
    \limsup_{n \to \i} \frac{\|H\|}{\sqrt{n}} \leq 2.
\]
As the spectral radius of $H$ is $\|H\|$, this implies that the spectrum of $H$ will ``increase'' at the rate of $O(\sqrt{n})$. Thus, it is natural to work with the normalized matrix $H/\sqrt{n}$.

\subsection{Semi-circular Law}
Wigner's semi-circular law concerns the asymptotic distribution of eigenvalues
\[
    \l_i\left( \frac{H}{\sqrt{n}}\right)
\]
of a random Wigner matrix $H$ as $n \to \i$. It states that the density of eigenvalues converges to the semi-circular shape described by:
\[
    \mu'_{sc}(x) = \frac{1}{2\pi}(4 - |x|^2)^{1/2}_+dx
\]
where $(x)_+ = \max(x, 0)$.

The proof of the semi-circular law uses the moment method.

\section{Unitary Ensembles}
A Guassian Unitary Ensemble (GUE) is a (Hermitian) random matrix ensemble that has a probability density function of the form
\[
    P(H) = \frac{1}{Z}\exp\left\{ -\frac{1}{4}Tr H^2 \right\}
\]
where $Z$ is the normalization factor.

It happens that any (Hermitian) random matrix ensemble (invariant under unitary transformations) with statistically independent (upper triangular and diagonal) entries is a GUE. In particular, Wigner ensembles are GUEs.

If the entries of the matrix are not neccessarily independent, then we instead get a unitary ensemble, with has a probability density function of the form
\[
    P(H) = \frac{1}{Z}\exp\left\{ -\frac{1}{4}Tr V(H) \right\}.
\]
Here, $V$ is some polynomial, not necessarily $V(H) = H^2$.

From a unitary ensemble of $n \times n$ Hermitian matrices, we can compute the joint probability distribution of eigenvalues to get
\[
    P(\l_1, \dots \l_n) = \frac{1}{Z}e^{-n \sum_i V(\l_i)} \prod_{i < j} |\l_i-\l_j|^2
\]

\section{A Gas of Electrons}
Note we can rewrite the joint distribution of eigenvalues as
\[
    \begin{split}
        P(\l_1, \dots, \l_n) &= Z^{-1} \prod_{i < j} |\l_i-\l_j|^2e^{-n\sum V(\l_i)}\\
        &= Z^{-1} \exp\left\{- \left( \sum_{i \neq j} \log |\l_i-\l_j|^{-1} + n \sum V(\l_i) \right) \right\}\\
    \end{split}
\]
In statistical mechanics, this happens to be the equation representing the distribution of a gas of electrons in a plane where the electrons are confined to a 1-D wire!
\begin{itemize}
    \item $Z$ happens to be the partition function of this electron gas!
    \item The position of the electrons are the eigenvalues (the log difference in the equation represents the logarithmic Couloumb potential of a 2D gas). Thus we have a ``repulsion of eigenvalues.''
    \item $V$ represents an external confining potential that prevents the eigenvalues escaping to infinity.
\end{itemize}


% End notes

\end{document}
