\section{A Gas of Electrons}
Note we can rewrite the joint distribution of eigenvalues as
\[
    \begin{split}
        P(\l_1, \dots, \l_n) &= Z^{-1} \prod_{i < j} |\l_i-\l_j|^2e^{-n\sum V(\l_i)}\\
        &= Z^{-1} \exp\left\{- \left( \sum_{i < j} \log |\l_i-\l_j|^{-1} + n \sum V(\l_i) \right) \right\}\\
    \end{split}
\]
In statistical mechanics, this happens to be the equation representing the distribution of an electron gas on the real line!
\begin{itemize}
    \item $Z$ happens to be the partition function of this electron gas!
    \item The position of the electrons are the eigenvalues (the log difference in the equation represents the logarithic Couloumb potential). Thus we have a ``repulsion of eigenvalues.''
    \item $V$ represents an external confining potential that presents the eigenvalues escaping to infinity.
\end{itemize}

