\section*{Spectrum of Hermitian Operators}

\subsection*{Pertubations}
$A$, $B$ Hermitian. Often want to study spectrum of $A + B$ using the spectrum of $A$ and $B$.
\begin{itemize}
    \item To do so, need some inequalities/identities about the spectrum of Hermitian operators, like Courant-Fischer min-max, extremal partial trace, etc.
    \item These inequalities/identities can be intuitively explained with the spectral theorem!
\end{itemize}
Among the conclusions about the spectrum of $A + B$ (like Ky Fan, Weyl) is that spectrum of Hermitian operators is stable under perturbation (take $B$ to be small). In fact, the eigenvalue maps are Lipschitz.
\begin{itemize}
    \item When spectrum is simple (no repeated eigenvalues), the eigenvalue maps have even better regularity conditions.
\end{itemize}

Another type of spectrum pertubation is passing from $A$ to a principal minor. There are again inequalities for this.

\subsection*{Random}
The expression
\[
    v^*Av, \quad |v| = 1
\]
means ``how much of $v$ is left after applying $A$.''

\subsection*{Singular Values}
The singular values of $T$ are the eigenvalues of $|T| = \sqrt{T^*T}$.
\begin{itemize}
    \item For Hermitian $T$, just the abs values of the eigenvalues. Largest and smallest singular values correspond to largest and smallest ``stretch'' of $T$.
    \item Have SVD (i.e. polar decomp).
\end{itemize}