\section{Lecture 3-10-2020}
Here, we prove that the measurable sets of $\mu_\phi$ contains the Borel $\s$-algebra, and thus can be restricted to the Borel $\s$-algebra. This requires some of the following concepts:
\begin{dfn}
If $\mu$ is a measure or outer measure, then $\mu$ is:
\begin{itemize}
    \item inner regular if the measures of sets can be approximated from below by the measures of compact sets,
    \item outer regular if the measures of sets can be approximated from above by the measures of open sets.
\end{itemize}
\end{dfn}
The content $\mu_\phi$ is inner regular for open sets! However, we require a slightly modified definition of inner regularity as $\mu_\phi$ is defined only on the open sets. What we mean is
\[
    \mu_\phi(U) = \sup\{\mu_\phi(V): V \ \text{open}, \ \bar{V} \subset U\}, \quad U \ \text{open.}
\]

\begin{thm}
Open sets are measurable in $\mu_\phi$.
\end{thm}

\begin{details}{Proof gist}
For open set $U$, we want to show for any $A \subset X$, we have
\[
    \mu_\phi^*(A - U) + \mu_\phi^*(A \cap U) = \mu_\phi^*(A).
\]
Use inner regularity for open sets to prove for $A$ open, then extend using outer regularity.
\end{details}

Now we are ready for our main theorem.

\begin{thm}[Reisz-Markov]
If $\phi$ is a positive radon measure on $X$, then
\[
    \phi(f) = \int f d\mu_\phi,
\]
where $\mu_\phi$ is inner regular for open sets, outer regular, the $\s$-algebra of $\mu_\phi$ contains the Borel sets.
\end{thm}
\begin{details}{Proof details for integral representation}
    Show $\phi$ and $\int d\mu_\phi$ well approximate each other for simple scenarios, i.e.
\[
    \chi_A \leq f \leq \chi_B \implies \int \chi_A d\mu_\phi \leq \phi(f) \leq \int \chi_B d\mu.
\]
Generalize to arbitrary $f \in C_c(X)$ by breaking $f$ apart into many small functions, applying the above equation to each of the functions, and then putting everything back with the triangle inequality.
\end{details}
The proof of the properties of $\mu_\phi$ are easy.
