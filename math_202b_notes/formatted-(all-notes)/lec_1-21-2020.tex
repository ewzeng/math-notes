\section{Lecture 1-21-2020}
\begin{dfn}
    Let $V$ be a vector space. A gauge is a function $p: V \rightarrow \bR$ that is ``half-linear", i.e.
    \begin{itemize}
    \item
    For $r > 0$, we have $p(rv) = rp(v)$.
    \item
    $p(u + v) \leq p(u) + p(v)$.
    \end{itemize}
\end{dfn}

\begin{thm}[Main lemma for Hahn-Banach]
A linear functional defined on a subspace $W$ of $V$ subordinate to gauge $p$ can be extended to a subordinate linear functional defined on $W \oplus \text{span}(v_0)$.
\end{thm}

\begin{details}{Proof gist}
    Show the existence of $\alpha$ such that
    \[
    \tilde{\phi}(w + rv_0) = \phi(w) + r \alpha
    \]
    is subordinate to $p$. Key trick is a separation of variables.
\end{details}

\begin{thm}[Hahn-Banach]
    A linear functional defined on a subspace $W$ of $V$ subordinate to gauge $p$ can be extended to a subordinate linear functional defined on $V$.
\end{thm}

\begin{details}{Proof gist}
    The main lemma extends linear functionals one dimension at a time. Apply Zorn's lemma to it (by considering the family of pairs of vector subspaces and subordinate linear functionals defined on them).
\end{details}

\begin{remark}
    To show the existence of continous linear functionals on normed vector spaces, let the gauge $p$ be the norm. Then any linear functional subordinate to $p$ is continuous.
\end{remark}
