\section{Lecture 4-30-2020}
We start with a crucial observation.
\begin{lem}
    Suppose $T \in \cB(\cH)$ is self-adjoint and $p \in \cP_\bR$. Then $\|p\|_\i = \|p(T)\|$, where $p$ on the LHS is seen as an element of $C(\s(T))$.
\end{lem}
\begin{proof}
    Note $\|p\|_\i = \|\s(p(T))\|_\i$ by Spectral mapping theorem, and $\|\s(p(T))\|_\i = \|p(T)\|$ because $p(T)$ is self-adjoint + results from last time.
\end{proof}
From this lemma, we have established a partial mapping from $C(\s(T))$ into $\cB(\cH)$ that is so far an isometry. Because the spectrum of $T$ is real, applying Stone-Weierstrass, we can extend the mapping to an isometry from $C_\bR(\s(T))$ into $\cB(\cH)$. Considering real and imaginary parts separately, we can extend further to an isometry from $C(\s(T))$ into $\cB(\cH)$. Thus we have a rough proof of the following (check a few details to make rigorous):
\begin{thm}[The Continuous Functional Calculus]
    Suppose $T \in \cB(\cH)$ is self-adjoint. Then $p \rightarrow p(T)$ extends to an isometric homomorphism from $C(\s(T))$ into $\cB(\cH)$ whose range is the C* algebra generated by $T$ and $I$.
\end{thm}
\begin{remark}
    The isometry above is still denoted $p \rightarrow p(T)$.
\end{remark}

This theorem is incredibly powerful because we can construct linear operators with certain properties just by picking an element of $C(\s(T))$. For instance, to prove Theorem \ref{positive operator representation}, we simply choose $\sqrt{x} \in C(\s(T))$. This is very cool! (From another persective, we define $\sqrt{T}$ by approximating with polynomials of $T$).
