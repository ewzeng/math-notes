\section{Lecture 3-10-2020}
Here, we prove that the measurable sets of $\mu_\phi$ contains the Borel $\s$-algebra, and thus can be restricted to the Borel $\s$-algebra. This requires some of the following concepts:
\begin{dfn}
If $\mu$ is a measure or outer measure, then $\mu$ is:
\begin{itemize}
    \item inner regular if the measures of sets can be approximated from below by the measures of compact sets,
    \item outer regular if the measures of sets can be approximated from above by the measures of open sets.
\end{itemize}
\end{dfn}
The content $\mu_\phi$ is inner regular for open sets! However, we require a slightly modified definition of inner regularity as $\mu_\phi$ is defined only on the open sets. What we mean is
\[
    \mu_\phi(U) = \sup\{\mu_\phi(V): V \ \text{open}, \ \bar{V} \subset U\}, \quad U \ \text{open.}
\]

\begin{thm}
Open sets are measurable in $\mu_\phi$.
\end{thm}

\begin{details}{Proof gist}
For open set $U$, we want to show for any $A \subset X$, we have
\[
    \mu_\phi^*(A - U) + \mu_\phi^*(A \cap U) = \mu_\phi^*(A).
\]
Use inner regularity for open sets to prove for $A$ open, then extend using outer regularity.
\end{details}

Now we are ready for our main theorem.

\begin{thm}[Reisz-Markov]
If $\phi$ is a positive radon measure on $X$, then
\[
    \phi(f) = \int f d\mu_\phi,
\]
where $\mu_\phi$ is inner regular for open sets, outer regular, the $\s$-algebra of $\mu_\phi$ contains the Borel sets.
\end{thm}
The proof of the properties of $\mu_\phi$ are easy. The proof of the integral representation can be divided into two parts.

\subsection{Part I: $\phi$ and $\int d\mu_\phi$ play well with each other.} Let $f \in C_c(X)$. Then
\[
    \chi_A \leq f \leq \chi_B \implies \int \chi_A d\mu_\phi = \mu_\phi(A) \leq \phi(f) \leq \mu_\phi(B) = \int \chi_B d\mu.
\]

\begin{details}{Proof gist}
We prove each inequality separately. We always first prove for open sets, and then generalize.
\end{details}
\begin{details}{More detail}
We need to play around with the following tricks.
\begin{itemize}
    \item Vector lattice tricks (see lecture 3-3-2020).
    \item Positive radon measures are continuous for the inductive limit topology! So if $\{f_n\} \rightarrow f$ converges in the inductive limit topology, then $\phi(f_n) \rightarrow \phi(f)$.
\end{itemize}
\end{details}

\subsection{Part II.} 
For $f \in C_c(X)$ and $\e > 0$,
\[
    |\phi(f) - \int f d\mu_\phi| < \e.
\]
\begin{remark}
From an intuitive perspective, it is not hard to see that Part I is an important tool to craft out Part II. The harder part is the formalization and fleshing out the details.
\end{remark}
\begin{details}{Proof gist}
Break apart $f$ into the sum of many small functions, apply Part I to the small functions, and the use the triangle inequality to put everthing back.
\end{details}
\begin{details}{More detail}
The explicit construction is to break up the range of $f$. Define $f_n = f \wedge n\e$, let $g_n = f_{n+1} - f_n$, and observe
\[
    \e \{x: f(x) > (n + 1)\e\} \leq g_n \leq \e \{x: f(x) > n\e\}.
\]
\end{details}
