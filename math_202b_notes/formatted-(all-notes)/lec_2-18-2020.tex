\section{Lecture 2-18-2020}
\textbf{Lattices!} Here's a way to remember the two lattice operations: pretend that the shapes represent the set of \underline{bounds}.
\begin{itemize}
    \item
    The $\wedge$ shape looks like one point is greater than all the points, so this represents the greatest lower bound.

    \item
    Similarly, $\vee$ represents the least upper bound.

\end{itemize}
In order of increasing structure, we have: lattice ordered (abelian) group $ \rightarrow $ lattice ordered vector space $ \rightarrow $ lattice ordered normed vector space. A partial order is compatible with a normed vector space $V$ if
\begin{itemize}
    \item
    For $v,w \in V$, $v, w \geq 0 \implies v + w \geq 0$.
    
    \item
    For $v \in V$, $r \in \bR$, $v \geq 0, r \geq 0 \implies rv \geq 0$.

    \item
    For $v \in V$, $\|v\| = \| |v| \|$. (c.f. below for definition of $|\cdot|$).

    \item
    For $v,w \in V$, $0 \leq v \leq w \implies \|v\| \leq \|w\|$.
\end{itemize}
The definitions of all the structures mentioned can be interpolated from this.

Here are some key properties and definitions of lattices (with the appropriate structure):
\begin{itemize}
    \item
    We can impose an order structure in a group by defining a collection of elements to be positive. (This collection has to satisfy some additional properties, evidently not all collections work).

    \item
    We define $v^+ = v \wedge 0$, $v^- = (-v) \wedge 0$, $v^+ + v^- = |v|$.

    \item
    For function spaces, we usually use the order: $f \geq 0$ means $f(x) \geq 0$ for all $x$. For spaces of functionals, we usually use the order $f \geq 0$ means $f(x) \geq 0$ if $x \geq 0$ (for the function space order).
\end{itemize}
