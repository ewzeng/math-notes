\section{Lecture 2-27-2020}
\begin{thm}
The space of real measures $\cS$ forms a vector lattice, with the positive elements of the lattice being the finite positive measures.
\end{thm}

\begin{details}{Proof gist}
For $\mu \in \cS$, we show that the total variation measure is $|\mu|$ (here, $| \cdot |$ is interpreted in the lattice sense). As we have seen before, this implies that $\cS$ is a lattice.
\end{details}
\begin{details}{More detail}
Let $\|\mu\|$ denote total variation measure. We show $\|\mu\|$ is countably additive by the usual argument (prove the inequality both ways, consider disjoint unions, etc). Then we show $\|\mu\|$ is finite by the following technicalitiy:
\begin{itemize}
    \item Real (and Banach) measures need to be absolutely convergent by definition, i.e. for disjoint sets $E_n$,
        \[
            \sum \mu(E_n) = \sum \mu(E_j)
        \]
        where $E_j$ represents a different ordering.

    \item If $\|\mu\|(E) = \i$ (i.e. $E$ is unbounded) but $\mu(E) < \i$, we construct a sequence of disjoint sets whose ``partial sums" are not absolutely convergent.
\end{itemize}
Finally we show $\|\mu\| = |\mu|$ by using the fact
\[
    \|\mu(E)\| = \sup\{ \mu(E_1) - \mu(E_2): E_1 \sqcup E_2 = E\}.
\]
\end{details}

\begin{details}{A final detail}
The construction of the sequence of disjoint sets mentioned above is as follows:
\begin{itemize}
    \item split $E$ into two disjoing unbounded sets $E_1, F_1$ where $|\mu(F_1)| > 1$.
    \item repeat for $E = E_1$
\end{itemize}
Our result is $F_1, F_2, F_3 \dots$.
\end{details}

\begin{remark}
It is not difficult to see that total variation is also a norm, and this norm plays well with the lattice. In other words, $\cS$ is a normed vector lattice.
\end{remark}
