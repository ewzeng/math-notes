\documentclass[12pt, letterpaper]{article}
\usepackage[utf8]{inputenc}
\usepackage{amsmath,amssymb}
\usepackage{xcolor} %For colored text
\usepackage{parskip}

%Title and margin formatting
\usepackage[left=2cm,top=3cm,bottom=3cm,right=2cm]{geometry} %Customize margins.
\usepackage{fancyhdr}
\setlength{\headheight}{15pt} %To avoid compiler warnings (12pt too small)
\fancyhead[L]{Edward Zeng, SID: 3034036984}
\fancyhead[C]{Math 202B Notes}
\fancyhead[R]{\today}

%Inkscape
\usepackage{import}
\usepackage{pdfpages}
\usepackage{transparent}

\newcommand{\incfig}[2][1]{%
    \def\svgwidth{#1\columnwidth}
    \import{./figures/}{#2.pdf_tex}
}

%Macros for Greek Letters
\renewcommand{\a}{\alpha}
\renewcommand{\b}{\beta}
\renewcommand{\d}{\delta}
\newcommand{\D}{\Delta}
\newcommand{\e}{\varepsilon}
\newcommand{\g}{\gamma}
\newcommand{\G}{\Gamma}
\renewcommand{\l}{\lambda}
\renewcommand{\L}{\Lambda}
\newcommand{\s}{\sigma}
\renewcommand{\th}{\theta}
\renewcommand{\o}{\omega}
\renewcommand{\O}{\Omega}
\renewcommand{\S}{\Sigma}
\renewcommand{\t}{\tau}
\newcommand{\var}{\varphi}
\newcommand{\z}{\zeta}

%Macros for math cal letters
\newcommand{\cA}{{\mathcal A}}
\newcommand{\cB}{{\mathcal B}}
\newcommand{\cC}{{\mathcal C}}
\newcommand{\cD}{{\mathcal D}}
\newcommand{\cE}{{\mathcal E}}
\newcommand{\cF}{{\mathcal F}}
\newcommand{\cH}{{\mathcal H}}
\newcommand{\cI}{{\mathcal I}}
\newcommand{\cK}{{\mathcal K}}
\newcommand{\cL}{{\mathcal L}}
\newcommand{\cM}{{\mathcal M}}
\newcommand{\cN}{{\mathcal N}}
\newcommand{\cO}{{\mathcal O}}
\newcommand{\cP}{{\mathcal P}}
\newcommand{\cS}{{\mathcal S}}
\newcommand{\cT}{{\mathcal T}}
\newcommand{\cU}{{\mathcal U}}
\newcommand{\cV}{{\mathcal V}}
\newcommand{\cW}{{\mathcal W}}
\newcommand{\cY}{{\mathcal Y}}

%Macros for blackboard bold letters
\newcommand{\bZ}{{\mathbb Z}}
\newcommand{\bR}{{\mathbb R}}
\newcommand{\bC}{{\mathbb C}}
\newcommand{\bT}{{\mathbb T}}
\newcommand{\bN}{{\mathbb N}}
\newcommand{\bQ}{{\mathbb Q}}
\newcommand{\bF}{{\mathbb F}}

%Other macros
\renewcommand{\i}{\infty}

\begin{document}
\pagestyle{fancy}
\textcolor{red}{A possible point of confusion: $\mu_\phi$ denotes both the measure and the content. Use context.}

Here, we prove that the measurable sets of $\mu_\phi$ contains the Borel $\s$-algebra, and thus can be restricted to the Borel $\s$-algebra. This requires some of the following concepts:

If $\mu$ is a measure or outer measure, then $\mu$ is:
\begin{itemize}
    \item inner regular if the measures of sets can be approximated from below by the measures of compact sets,
    \item outer regular if the measures of sets can be approximated from above by the measures of open sets.
\end{itemize}

The content $\mu_\phi$ is inner regular for open sets! However, we require a slightly modified definition of inner regularity as $\mu_\phi$ is defined only on the open sets. What we mean is
\[
    \mu_\phi(U) = \sup\{\mu_\phi(V): V \ \text{open}, \ \bar{V} \subset U\}, \quad U \ \text{open.}
\]

\noindent\rule{\textwidth}{1pt}

\textbf{Theorem.} Open sets are measurable in $\mu_\phi$.

\textbf{Proof gist.} For open set $U$, we want to show for any $A \subset X$, we have
\[
    \mu_\phi^*(A - U) + \mu_\phi^*(A \cap U) = \mu_\phi^*(A).
\]
Use inner regularity for open sets to prove for $A$ open, then extend using outer regularity.

\textbf{More detail.} To prove for $A$ open, we need to use LCH Urysohn's friend and finite additivity. The key is that $A - U$ is not necessarily open, so we need to do a few tricks before applying finite additivity.

\textbf{Inner regularity for open sets.} Because $\mu_\phi$ is inner regular for open sets, this extends to $\mu_\phi^*$. It is also easy to see that $\mu_\phi^*$ is outer regular.

\end{document}
