\documentclass[12pt, letterpaper]{article}
\usepackage[utf8]{inputenc}
\usepackage{amsmath,amssymb}
\usepackage{xcolor} %For colored text
\usepackage{parskip}

%Title and margin formatting
\usepackage[left=2cm,top=3cm,bottom=3cm,right=2cm]{geometry} %Customize margins.
\usepackage{fancyhdr}
\setlength{\headheight}{15pt} %To avoid compiler warnings (12pt too small)
\fancyhead[L]{Edward Zeng, SID: 3034036984}
\fancyhead[C]{Math 202B Notes}
\fancyhead[R]{\today}

%Inkscape
\usepackage{import}
\usepackage{pdfpages}
\usepackage{transparent}

\newcommand{\incfig}[2][1]{%
    \def\svgwidth{#1\columnwidth}
    \import{./figures/}{#2.pdf_tex}
}

%Macros for Greek Letters
\renewcommand{\a}{\alpha}
\renewcommand{\b}{\beta}
\renewcommand{\d}{\delta}
\newcommand{\D}{\Delta}
\newcommand{\e}{\varepsilon}
\newcommand{\g}{\gamma}
\newcommand{\G}{\Gamma}
\renewcommand{\l}{\lambda}
\renewcommand{\L}{\Lambda}
\newcommand{\s}{\sigma}
\renewcommand{\th}{\theta}
\renewcommand{\o}{\omega}
\renewcommand{\O}{\Omega}
\renewcommand{\S}{\Sigma}
\renewcommand{\t}{\tau}
\newcommand{\var}{\varphi}
\newcommand{\z}{\zeta}

%Macros for math cal letters
\newcommand{\cA}{{\mathcal A}}
\newcommand{\cB}{{\mathcal B}}
\newcommand{\cC}{{\mathcal C}}
\newcommand{\cD}{{\mathcal D}}
\newcommand{\cE}{{\mathcal E}}
\newcommand{\cF}{{\mathcal F}}
\newcommand{\cH}{{\mathcal H}}
\newcommand{\cI}{{\mathcal I}}
\newcommand{\cK}{{\mathcal K}}
\newcommand{\cL}{{\mathcal L}}
\newcommand{\cM}{{\mathcal M}}
\newcommand{\cN}{{\mathcal N}}
\newcommand{\cO}{{\mathcal O}}
\newcommand{\cP}{{\mathcal P}}
\newcommand{\cS}{{\mathcal S}}
\newcommand{\cT}{{\mathcal T}}
\newcommand{\cU}{{\mathcal U}}
\newcommand{\cV}{{\mathcal V}}
\newcommand{\cW}{{\mathcal W}}
\newcommand{\cY}{{\mathcal Y}}

%Macros for blackboard bold letters
\newcommand{\bZ}{{\mathbb Z}}
\newcommand{\bR}{{\mathbb R}}
\newcommand{\bC}{{\mathbb C}}
\newcommand{\bT}{{\mathbb T}}
\newcommand{\bN}{{\mathbb N}}
\newcommand{\bQ}{{\mathbb Q}}
\newcommand{\bF}{{\mathbb F}}

%Other macros
\renewcommand{\i}{\infty}

\begin{document}
\pagestyle{fancy}

\textbf{Theorem.} The dual of a normed vector lattice $V$ is a normed vector lattice (with the order usually associated with functionals).

\textbf{Proof gist.} Given $\phi \in V'$, we independently construct $\phi^+ \in V'$ and show $\phi^+ = \phi \vee 0$. The translation properties of ordered vector spaces then shows $V'$ is lattice ordered. We then check the rest of the properties of a normed vector lattice (some inequalities and bashing required).

\textbf{Construction of $\phi^+$} We first define $\phi^+$ on $V^+$ by
\[
    \phi^+(v) = \sup\{\phi(x): 0 \leq x \leq v\}
\]
and prove it is linear. Then we linearly extend $\phi^+$ to $V$ and show it is continuous.

\textbf{Some techniques used.}
\begin{itemize}
    \item Prove something for positive $v$ and extend via $v = v^+ - v^-$.

    \item Prove some inequality for variables $x$ and $y$ satisfying some condition. The inequality holds if we take the supremum over $x$ and $y$ with this condition.

    \item Use the fact that $V$ is a normed vector lattice!
\end{itemize}

\noindent\rule{\textwidth}{1pt}
\textbf{Remark.} We already showed that if $p,q$ are Holder conjugates with $1 < p,q < \i$, then there is an isometric bijection between $(\cL^p)'$ and $\cL^q$. The theorem above extends this statement to all of $(\cL^p)'$ and $\cL^q$.

\end{document}
