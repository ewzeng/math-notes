\documentclass[12pt, letterpaper]{article}
\usepackage[utf8]{inputenc}
\usepackage{amsmath,amssymb}
\usepackage{xcolor} %For colored text
\usepackage{parskip}

%Title and margin formatting
\usepackage[left=2cm,top=3cm,bottom=3cm,right=2cm]{geometry} %Customize margins.
\usepackage{fancyhdr}
\setlength{\headheight}{15pt} %To avoid compiler warnings (12pt too small)
\fancyhead[L]{Edward Zeng, SID: 3034036984}
\fancyhead[C]{Math 202B Notes}
\fancyhead[R]{\today}

%Inkscape
\usepackage{import}
\usepackage{pdfpages}
\usepackage{transparent}

\newcommand{\incfig}[2][1]{%
    \def\svgwidth{#1\columnwidth}
    \import{./figures/}{#2.pdf_tex}
}

%Macros for Greek Letters
\renewcommand{\a}{\alpha}
\renewcommand{\b}{\beta}
\renewcommand{\d}{\delta}
\newcommand{\D}{\Delta}
\newcommand{\e}{\varepsilon}
\newcommand{\g}{\gamma}
\newcommand{\G}{\Gamma}
\renewcommand{\l}{\lambda}
\renewcommand{\L}{\Lambda}
\newcommand{\s}{\sigma}
\renewcommand{\th}{\theta}
\renewcommand{\o}{\omega}
\renewcommand{\O}{\Omega}
\renewcommand{\S}{\Sigma}
\renewcommand{\t}{\tau}
\newcommand{\var}{\varphi}
\newcommand{\z}{\zeta}

%Macros for math cal letters
\newcommand{\cA}{{\mathcal A}}
\newcommand{\cB}{{\mathcal B}}
\newcommand{\cC}{{\mathcal C}}
\newcommand{\cD}{{\mathcal D}}
\newcommand{\cE}{{\mathcal E}}
\newcommand{\cF}{{\mathcal F}}
\newcommand{\cH}{{\mathcal H}}
\newcommand{\cI}{{\mathcal I}}
\newcommand{\cK}{{\mathcal K}}
\newcommand{\cL}{{\mathcal L}}
\newcommand{\cM}{{\mathcal M}}
\newcommand{\cN}{{\mathcal N}}
\newcommand{\cO}{{\mathcal O}}
\newcommand{\cP}{{\mathcal P}}
\newcommand{\cS}{{\mathcal S}}
\newcommand{\cT}{{\mathcal T}}
\newcommand{\cU}{{\mathcal U}}
\newcommand{\cV}{{\mathcal V}}
\newcommand{\cW}{{\mathcal W}}
\newcommand{\cY}{{\mathcal Y}}

%Macros for blackboard bold letters
\newcommand{\bZ}{{\mathbb Z}}
\newcommand{\bR}{{\mathbb R}}
\newcommand{\bC}{{\mathbb C}}
\newcommand{\bT}{{\mathbb T}}
\newcommand{\bN}{{\mathbb N}}
\newcommand{\bQ}{{\mathbb Q}}
\newcommand{\bF}{{\mathbb F}}

%Other macros
\renewcommand{\i}{\infty}

\begin{document}
\pagestyle{fancy}
\textbf{Reisz-Markov.} If $\phi$ is a positive radon measure on $X$, then
\[
    \phi(f) = \int f d\mu_\phi,
\]
where $\mu_\phi$ is inner regular for open sets, outer regular, the $\s$-algebra of $\mu_\phi$ contains the Borel sets.

The proof of the properties of $\mu_\phi$ are easy. The proof of the integral representation can be divided into two parts.

\noindent\rule{\textwidth}{1pt}
\textbf{Part I: $\phi$ and $\int d\mu_\phi$ play well with each other.} Let $f \in C_c(X)$. Then
\[
    \chi_A \leq f \leq \chi_B \implies \int \chi_A d\mu_\phi = \mu_\phi(A) \leq \phi(f) \leq \mu_\phi(B) = \int \chi_B d\mu.
\]

\textbf{Proof gist.} We prove each inequality separately. We always first prove for open sets, and then generalize.

\textbf{More detail.} We need to play around with the following tricks.
\begin{itemize}
    \item Vector lattice tricks (see lecture 3-3-2020).
    \item Positive radon measures are continuous for the inductive limit topology! So if $\{f_n\} \rightarrow f$ converges in the inductive limit topology, then $\phi(f_n) \rightarrow f$.
\end{itemize}

\noindent\rule{\textwidth}{1pt}
\textbf{Part II.} For $f \in C_c(X)$ and $\e > 0$,
\[
    |\phi(f) - \int f d\mu_\phi| < \e.
\]

From an intuitive perspective, it is not hard to see that Part I is an important tool to craft out Part II. The harder part is the formalization and fleshing out the details.

\textbf{Proof gist.} Break apart $f$ into the sum of many small functions, apply Part I to the small functions, and the use the triangle inequality to put everthing back.

\textbf{More detail.} The explicit construction is to break up the range of $f$. Define $f_n = f \wedge n\e$, let $g_n = f_{n+1} - f_n$, and observe
\[
    \e \{x: f(x) > (n + 1)\e\} \leq g_n \leq \e \{x: f(x) > n\e\}.
\]


\end{document}
