\documentclass[12pt, letterpaper]{article}
\usepackage[utf8]{inputenc}
\usepackage{amsmath,amssymb}
\usepackage{xcolor} %For colored text
\usepackage{parskip}

%Title and margin formatting
\usepackage[left=2cm,top=3cm,bottom=3cm,right=2cm]{geometry} %Customize margins.
\usepackage{titling} %Customize the position of the title.
\setlength{\droptitle}{-50pt} %Raise the position of the title.

%Inkscape
\usepackage{import}
\usepackage{pdfpages}
\usepackage{transparent}

\newcommand{\incfig}[2][1]{%
    \def\svgwidth{#1\columnwidth}
    \import{./figures/}{#2.pdf_tex}
}

%Macros for Greek Letters
\renewcommand{\a}{\alpha}
\renewcommand{\b}{\beta}
\renewcommand{\d}{\delta}
\newcommand{\D}{\Delta}
\newcommand{\e}{\varepsilon}
\newcommand{\g}{\gamma}
\newcommand{\G}{\Gamma}
\renewcommand{\l}{\lambda}
\renewcommand{\L}{\Lambda}
\newcommand{\s}{\sigma}
\renewcommand{\th}{\theta}
\renewcommand{\o}{\omega}
\renewcommand{\O}{\Omega}
\renewcommand{\S}{\Sigma}
\renewcommand{\t}{\tau}
\newcommand{\var}{\varphi}
\newcommand{\z}{\zeta}

%Macros for math cal letters
\newcommand{\cA}{{\mathcal A}}
\newcommand{\cB}{{\mathcal B}}
\newcommand{\cC}{{\mathcal C}}
\newcommand{\cD}{{\mathcal D}}
\newcommand{\cE}{{\mathcal E}}
\newcommand{\cF}{{\mathcal F}}
\newcommand{\cH}{{\mathcal H}}
\newcommand{\cI}{{\mathcal I}}
\newcommand{\cK}{{\mathcal K}}
\newcommand{\cL}{{\mathcal L}}
\newcommand{\cM}{{\mathcal M}}
\newcommand{\cN}{{\mathcal N}}
\newcommand{\cO}{{\mathcal O}}
\newcommand{\cP}{{\mathcal P}}
\newcommand{\cS}{{\mathcal S}}
\newcommand{\cT}{{\mathcal T}}
\newcommand{\cU}{{\mathcal U}}
\newcommand{\cV}{{\mathcal V}}
\newcommand{\cW}{{\mathcal W}}
\newcommand{\cY}{{\mathcal Y}}

%Macros for blackboard bold letters
\newcommand{\bZ}{{\mathbb Z}}
\newcommand{\bR}{{\mathbb R}}
\newcommand{\bC}{{\mathbb C}}
\newcommand{\bT}{{\mathbb T}}
\newcommand{\bN}{{\mathbb N}}
\newcommand{\bQ}{{\mathbb Q}}
\newcommand{\bF}{{\mathbb F}}

%Other macros
\renewcommand{\i}{\infty}

\begin{document}

\textbf{Theorem.} For $\mu(x) < \i$, $\Big(\cL^1(X, \cS, \mu)\Big)'$ is isometric and isomorphic to $L^\i(X, \cS, \mu)$.

\textbf{Main idea.} Manuever $\cL^1$ into the $\cL^2$ setting and apply the Reisz-Representation theorem.

\textbf{More detail.} Observe for $\mu(X) < \i$, we have $\big(\cL^1(\mu)\big)' \subset \big(\cL^2(\mu)\big)'$. Thus by R-R, for every $\phi \in \big(\cL^1(\mu)\big)'$ there exists a $g$ such that
\[
    \phi(f) = \int fg.
\]
We then show $g \in \cL^\i(\mu)$. Conversely, we then show every $g \in \cL^\i(\mu)$ defines a continuous linear functional on $\cL^1(\mu)$ by the equation above.

\textbf{Key trick.} To $g$ is bounded a.e., we show for a closed set $C$, we have
\[
    \frac{1}{\mu(E)}\int_E g \in C, \text{  } \forall \text{ } \mu(E) > 0 \implies g(x) \in C \text{ a.e.}
\]
To do this, we show any open ball disjoint with $C$ must have preimage measure 0. Then as the range of $g$ is separable, we take a countable union of all such open balls.

\textbf{Motivation for first step.} For spaces of finite measure, $\cL^2 \subset \cL^1$. Thus it makes sense for linear functionals on $\cL^1$ to also be linear functionals on $\cL^1$.

\textbf{Extension remarks.} This theorem can be extended to $\s$-finite measures (as many statements about finite measures can).

\end{document}
