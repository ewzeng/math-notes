\documentclass[12pt, letterpaper]{article}
\usepackage[utf8]{inputenc}
\usepackage{amsmath,amssymb}
\usepackage{xcolor} %For colored text
\usepackage{parskip}

%Title and margin formatting
\usepackage[left=2cm,top=3cm,bottom=3cm,right=2cm]{geometry} %Customize margins.
\usepackage{titling} %Customize the position of the title.
\setlength{\droptitle}{-50pt} %Raise the position of the title.

%Inkscape
\usepackage{import}
\usepackage{pdfpages}
\usepackage{transparent}

\newcommand{\incfig}[2][1]{%
    \def\svgwidth{#1\columnwidth}
    \import{./figures/}{#2.pdf_tex}
}

%Macros for Greek Letters
\renewcommand{\a}{\alpha}
\renewcommand{\b}{\beta}
\renewcommand{\d}{\delta}
\newcommand{\D}{\Delta}
\newcommand{\e}{\varepsilon}
\newcommand{\g}{\gamma}
\newcommand{\G}{\Gamma}
\renewcommand{\l}{\lambda}
\renewcommand{\L}{\Lambda}
\newcommand{\s}{\sigma}
\renewcommand{\th}{\theta}
\renewcommand{\o}{\omega}
\renewcommand{\O}{\Omega}
\renewcommand{\S}{\Sigma}
\renewcommand{\t}{\tau}
\newcommand{\var}{\varphi}
\newcommand{\z}{\zeta}

%Macros for math cal letters
\newcommand{\cA}{{\mathcal A}}
\newcommand{\cB}{{\mathcal B}}
\newcommand{\cC}{{\mathcal C}}
\newcommand{\cD}{{\mathcal D}}
\newcommand{\cE}{{\mathcal E}}
\newcommand{\cF}{{\mathcal F}}
\newcommand{\cH}{{\mathcal H}}
\newcommand{\cI}{{\mathcal I}}
\newcommand{\cK}{{\mathcal K}}
\newcommand{\cL}{{\mathcal L}}
\newcommand{\cM}{{\mathcal M}}
\newcommand{\cN}{{\mathcal N}}
\newcommand{\cO}{{\mathcal O}}
\newcommand{\cP}{{\mathcal P}}
\newcommand{\cS}{{\mathcal S}}
\newcommand{\cT}{{\mathcal T}}
\newcommand{\cU}{{\mathcal U}}
\newcommand{\cV}{{\mathcal V}}
\newcommand{\cW}{{\mathcal W}}
\newcommand{\cY}{{\mathcal Y}}

%Macros for blackboard bold letters
\newcommand{\bZ}{{\mathbb Z}}
\newcommand{\bR}{{\mathbb R}}
\newcommand{\bC}{{\mathbb C}}
\newcommand{\bT}{{\mathbb T}}
\newcommand{\bN}{{\mathbb N}}
\newcommand{\bQ}{{\mathbb Q}}
\newcommand{\bF}{{\mathbb F}}

%Other macros
\renewcommand{\i}{\infty}

\begin{document}

\textbf{The weak topology is tight.} Suppose $V$ is a vector space, and $W$ is a collection of linear functionals on $V$. Then if $\phi$ is a continuous linear functional for the $W$-weak topology, $\phi \in W$.

\textbf{Key trick.} If $\phi_1, \dots, \phi_n$ are a collection of linear functionals, then
\[
    \bigcap \ker \phi_i \subset \ker \phi \implies \phi \in \text{span}\{\phi_i\}.
\]
To show the LHS, we compare the topology generated by $\phi$ with the subbase of the weak topology.

\noindent\rule{\textwidth}{1pt}

\textbf{Krein-Milman.} Let $C$ be a closed convex subset of a locally convex topological vector space $V$ (assumed Hausdorff). Then the convex hull of extreme points of $C$ is $C$.

\textbf{Main idea.} Consider the poset $P$ of all the faces of $C$. Then the minimal points of $P$ are the extreme points (need Zorn for existence). Then use this property to prove this theorem.

\textbf{Key technique.} If $D$ is compact convex set, we consider a continuous linear functional $\phi$ not constant on $D$ (use H-B or H-B separation). Then
\[
    \{v \in D: \phi(v) \text{ achieves its maximum.} \}
\]
is a proper (compact convex) face.

\end{document}
