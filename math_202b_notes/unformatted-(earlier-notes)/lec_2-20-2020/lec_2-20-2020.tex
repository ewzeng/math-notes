\documentclass[12pt, letterpaper]{article}
\usepackage[utf8]{inputenc}
\usepackage{amsmath,amssymb}
\usepackage{xcolor} %For colored text
\usepackage{parskip}

%Title and margin formatting
\usepackage[left=2cm,top=3cm,bottom=3cm,right=2cm]{geometry} %Customize margins.
\usepackage{fancyhdr}
\setlength{\headheight}{15pt} %To avoid compiler warnings (12pt too small)
\fancyhead[L]{Edward Zeng, SID: 3034036984}
\fancyhead[C]{Math 202B Notes}
\fancyhead[R]{\today}

%Inkscape
\usepackage{import}
\usepackage{pdfpages}
\usepackage{transparent}

\newcommand{\incfig}[2][1]{%
    \def\svgwidth{#1\columnwidth}
    \import{./figures/}{#2.pdf_tex}
}

%Macros for Greek Letters
\renewcommand{\a}{\alpha}
\renewcommand{\b}{\beta}
\renewcommand{\d}{\delta}
\newcommand{\D}{\Delta}
\newcommand{\e}{\varepsilon}
\newcommand{\g}{\gamma}
\newcommand{\G}{\Gamma}
\renewcommand{\l}{\lambda}
\renewcommand{\L}{\Lambda}
\newcommand{\s}{\sigma}
\renewcommand{\th}{\theta}
\renewcommand{\o}{\omega}
\renewcommand{\O}{\Omega}
\renewcommand{\S}{\Sigma}
\renewcommand{\t}{\tau}
\newcommand{\var}{\varphi}
\newcommand{\z}{\zeta}

%Macros for math cal letters
\newcommand{\cA}{{\mathcal A}}
\newcommand{\cB}{{\mathcal B}}
\newcommand{\cC}{{\mathcal C}}
\newcommand{\cD}{{\mathcal D}}
\newcommand{\cE}{{\mathcal E}}
\newcommand{\cF}{{\mathcal F}}
\newcommand{\cH}{{\mathcal H}}
\newcommand{\cI}{{\mathcal I}}
\newcommand{\cK}{{\mathcal K}}
\newcommand{\cL}{{\mathcal L}}
\newcommand{\cM}{{\mathcal M}}
\newcommand{\cN}{{\mathcal N}}
\newcommand{\cO}{{\mathcal O}}
\newcommand{\cP}{{\mathcal P}}
\newcommand{\cS}{{\mathcal S}}
\newcommand{\cT}{{\mathcal T}}
\newcommand{\cU}{{\mathcal U}}
\newcommand{\cV}{{\mathcal V}}
\newcommand{\cW}{{\mathcal W}}
\newcommand{\cY}{{\mathcal Y}}

%Macros for blackboard bold letters
\newcommand{\bZ}{{\mathbb Z}}
\newcommand{\bR}{{\mathbb R}}
\newcommand{\bC}{{\mathbb C}}
\newcommand{\bT}{{\mathbb T}}
\newcommand{\bN}{{\mathbb N}}
\newcommand{\bQ}{{\mathbb Q}}
\newcommand{\bF}{{\mathbb F}}

%Other macros
\renewcommand{\i}{\infty}

\begin{document}
\pagestyle{fancy}

\textbf{Dense subsets of $C_\bR(X)$.} Let $X$ be a compact space, and $L$ a subspace and sublattice of $C(X)$. If $L$ strongly separates points in $X$, then $L$ is dense in $C(X)$ (for the sup norm).

\textbf{Remark.} Strong seperation implies $X$ is necessarily Hausdorff.

\textbf{Proof gist.} Given $f \in C(X)$, we construct a $g \in L$ such that $\|f-g\| < \e$. To do this, we use strong separation to generate families of functions in $L$, then piece these functions together using compactness and the $\wedge$, $\vee$ operators.

\noindent\rule{\textwidth}{1pt}

\textbf{Real Stone-Weierstrass.} Same statement as before, but instead of $L$, we consider subalgebra $A$.

\textbf{Proof gist.} We prove $\bar{A}$ is a subspace and sublattice, then invoke our previous theorem.

\textbf{Key tricks.} There are two key tricks to show that $\bar{A}$ is sublattice. The first is to note
\[
    f + g = \frac{f + g + |f-g|}{2}, \quad |f| = \sqrt{f^2}.
\]
The second is to observe that $\sqrt{t}$ can be uniformly approximated on a compact interval by a polynomial $p(t)$. This uses some power series knowledge.

\textbf{Extension.} We can extend Stone-Weierstrass to the complex case if we force $A$ to be closed under complex conjugation.

\end{document}
