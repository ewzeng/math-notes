\documentclass[12pt, letterpaper]{article}
\usepackage[utf8]{inputenc}
\usepackage{amsmath,amssymb}
\usepackage{xcolor} %For colored text
\usepackage{parskip}

%Title and margin formatting
\usepackage[left=2cm,top=3cm,bottom=3cm,right=2cm]{geometry} %Customize margins.
\usepackage{fancyhdr}
\setlength{\headheight}{15pt} %To avoid compiler warnings (12pt too small)
\fancyhead[L]{Edward Zeng, SID: 3034036984}
\fancyhead[C]{Math 202B Notes}
\fancyhead[R]{\today}

%Inkscape
\usepackage{import}
\usepackage{pdfpages}
\usepackage{transparent}

\newcommand{\incfig}[2][1]{%
    \def\svgwidth{#1\columnwidth}
    \import{./figures/}{#2.pdf_tex}
}

%Macros for Greek Letters
\renewcommand{\a}{\alpha}
\renewcommand{\b}{\beta}
\renewcommand{\d}{\delta}
\newcommand{\D}{\Delta}
\newcommand{\e}{\varepsilon}
\newcommand{\g}{\gamma}
\newcommand{\G}{\Gamma}
\renewcommand{\l}{\lambda}
\renewcommand{\L}{\Lambda}
\newcommand{\s}{\sigma}
\renewcommand{\th}{\theta}
\renewcommand{\o}{\omega}
\renewcommand{\O}{\Omega}
\renewcommand{\S}{\Sigma}
\renewcommand{\t}{\tau}
\newcommand{\var}{\varphi}
\newcommand{\z}{\zeta}

%Macros for math cal letters
\newcommand{\cA}{{\mathcal A}}
\newcommand{\cB}{{\mathcal B}}
\newcommand{\cC}{{\mathcal C}}
\newcommand{\cD}{{\mathcal D}}
\newcommand{\cE}{{\mathcal E}}
\newcommand{\cF}{{\mathcal F}}
\newcommand{\cH}{{\mathcal H}}
\newcommand{\cI}{{\mathcal I}}
\newcommand{\cK}{{\mathcal K}}
\newcommand{\cL}{{\mathcal L}}
\newcommand{\cM}{{\mathcal M}}
\newcommand{\cN}{{\mathcal N}}
\newcommand{\cO}{{\mathcal O}}
\newcommand{\cP}{{\mathcal P}}
\newcommand{\cS}{{\mathcal S}}
\newcommand{\cT}{{\mathcal T}}
\newcommand{\cU}{{\mathcal U}}
\newcommand{\cV}{{\mathcal V}}
\newcommand{\cW}{{\mathcal W}}
\newcommand{\cY}{{\mathcal Y}}

%Macros for blackboard bold letters
\newcommand{\bZ}{{\mathbb Z}}
\newcommand{\bR}{{\mathbb R}}
\newcommand{\bC}{{\mathbb C}}
\newcommand{\bT}{{\mathbb T}}
\newcommand{\bN}{{\mathbb N}}
\newcommand{\bQ}{{\mathbb Q}}
\newcommand{\bF}{{\mathbb F}}

%Other macros
\renewcommand{\i}{\infty}

\begin{document}
\pagestyle{fancy}

\textbf{Lattices!} Here's a way to remember the two lattice operations: pretend that the shapes represent the set of \underline{bounds}.
\begin{itemize}
    \item
    The $\wedge$ shape looks like one point is greater than all the points, so this represents the greatest lower bound.

    \item
    Similarly, $\vee$ represents the least upper bound.

\end{itemize}
In order of increasing structure, we have: lattice ordered (abelian) group $ \rightarrow $ lattice ordered vector space $ \rightarrow $ lattice ordered normed vector space. A partial order is compatible with a normed vector space $V$ if
\begin{itemize}
    \item
    For $v,w \in V$, $v, w \geq 0 \implies v + w \geq 0$.
    
    \item
    For $v \in V$, $r \in \bR$, $v \geq 0, r \geq 0 \implies rv \geq 0$.

    \item
    For $v \in V$, $\|v\| = \| |v| \|$. (c.f. below for definition of $|\cdot|$).

    \item
    For $v,w \in V$, $0 \leq v \leq w \implies \|v\| \leq \|w\|$.
\end{itemize}
The definitions of all the structures mentioned can be interpolated from this.

Here are some key properties and definitions of lattices (with the appropriate structure):
\begin{itemize}
    \item
    We can impose an order structure in a group by defining a collection of elements to be positive. (This collection has to satisfy some additional properties, evidently not all collections work).

    \item
    We define $v^+ = v \wedge 0$, $v^- = (-v) \wedge 0$, $v^+ + v^- = |v|$.

    \item
    For function spaces, we usually use the order: $f \geq 0$ means $f(x) \geq 0$ for all $x$. For spaces of functionals, we usually use the order $f \geq 0$ means $f(x) \geq 0$ if $x \geq 0$ (for the function space order).
\end{itemize}


\end{document}
