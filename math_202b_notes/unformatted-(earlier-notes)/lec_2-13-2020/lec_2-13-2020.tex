\documentclass[12pt, letterpaper]{article}
\usepackage[utf8]{inputenc}
\usepackage{amsmath,amssymb}
\usepackage{xcolor} %For colored text
\usepackage{parskip}

%Title and margin formatting
\usepackage[left=2cm,top=3cm,bottom=3cm,right=2cm]{geometry} %Customize margins.
\usepackage{titling} %Customize the position of the title.
\setlength{\droptitle}{-50pt} %Raise the position of the title.

%Inkscape
\usepackage{import}
\usepackage{pdfpages}
\usepackage{transparent}

\newcommand{\incfig}[2][1]{%
    \def\svgwidth{#1\columnwidth}
    \import{./figures/}{#2.pdf_tex}
}

%Macros for Greek Letters
\renewcommand{\a}{\alpha}
\renewcommand{\b}{\beta}
\renewcommand{\d}{\delta}
\newcommand{\D}{\Delta}
\newcommand{\e}{\varepsilon}
\newcommand{\g}{\gamma}
\newcommand{\G}{\Gamma}
\renewcommand{\l}{\lambda}
\renewcommand{\L}{\Lambda}
\newcommand{\s}{\sigma}
\renewcommand{\th}{\theta}
\renewcommand{\o}{\omega}
\renewcommand{\O}{\Omega}
\renewcommand{\S}{\Sigma}
\renewcommand{\t}{\tau}
\newcommand{\var}{\varphi}
\newcommand{\z}{\zeta}

%Macros for math cal letters
\newcommand{\cA}{{\mathcal A}}
\newcommand{\cB}{{\mathcal B}}
\newcommand{\cC}{{\mathcal C}}
\newcommand{\cD}{{\mathcal D}}
\newcommand{\cE}{{\mathcal E}}
\newcommand{\cF}{{\mathcal F}}
\newcommand{\cH}{{\mathcal H}}
\newcommand{\cI}{{\mathcal I}}
\newcommand{\cK}{{\mathcal K}}
\newcommand{\cL}{{\mathcal L}}
\newcommand{\cM}{{\mathcal M}}
\newcommand{\cN}{{\mathcal N}}
\newcommand{\cO}{{\mathcal O}}
\newcommand{\cP}{{\mathcal P}}
\newcommand{\cS}{{\mathcal S}}
\newcommand{\cT}{{\mathcal T}}
\newcommand{\cU}{{\mathcal U}}
\newcommand{\cV}{{\mathcal V}}
\newcommand{\cW}{{\mathcal W}}
\newcommand{\cY}{{\mathcal Y}}

%Macros for blackboard bold letters
\newcommand{\bZ}{{\mathbb Z}}
\newcommand{\bR}{{\mathbb R}}
\newcommand{\bC}{{\mathbb C}}
\newcommand{\bT}{{\mathbb T}}
\newcommand{\bN}{{\mathbb N}}
\newcommand{\bQ}{{\mathbb Q}}
\newcommand{\bF}{{\mathbb F}}

%Other macros
\renewcommand{\i}{\infty}

\begin{document}

\textbf{Key Equation.} Suppose $\mu, \nu$ are two $\sigma$-finite measures. Observe that the map
\[
    \phi(f) = \int f d\nu
\]
is an element of the dual space of $\cL^1(\mu + \nu)$. Thus there exists an $h \in \cL^\i(\mu + \nu)$ such that
\[
    \int f d\nu = \int fh d(\mu + \nu) = \int fh d\mu + \int fh d\nu,
\]
i.e.
\[
    \int f(1-h) d\nu = \int fh d\mu.
\]

\noindent\rule{\textwidth}{1pt}

\textbf{Lebesgue Decomposition.} Observe $\|h\| \leq 1$. Define
\[
    E = \{x : h(x) = 1\}, \quad F = X\backslash E
\]
and use the equation to note that $\nu|_E$ is singular to $\mu$ and $\nu|_F$ is absolutely continuous to $\mu$.

\noindent\rule{\textwidth}{1pt}

\textbf{Radon-Nikodyn.} When the set $\{x : h(x) = 1\}$ has measure 0 (i.e. $\nu$ is absolutely continuous w/r/t $\mu$), we choose $f = \chi_G$ and note
\[
    \nu(G) = \int \chi_G d \nu = \int \frac{h\chi_G}{1 - h} d\mu.
\]

\newpage

\textbf{Theorem.} For $1 < p < \i$, $ \frac{1}{p} + \frac{1}{q} = 1$, and finite measure space $X$, there is an isometric bijection between the positive elements of $\big(\cL^p(\mu)\big)'$ and $\cL^{q}(\mu)$.

\textbf{Proof gist.} For positive $\phi \in \big(\cL^p(\mu)\big)'$, observe
\[
    \nu(E) = \int \phi(\chi_E) d\mu
\]
is a measure absolutely continuous to $\mu$. Then by Radon-Nikodyn, there exists a positive $g$ such that
\[
    \nu(E) = \int_E g d\mu.
\]
We show that $g \in \cL^{q}(\mu)$. The other direction comes from Holder and the following observation proved via simple functions:
\[
    \phi(f) = \int f d\nu = \int fg d\mu
\]

\textbf{Remark.} In the usual fashion, this can be extended to the $\s$-finite case.

\end{document}
