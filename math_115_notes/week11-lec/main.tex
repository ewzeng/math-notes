\documentclass[12pt]{article}
\usepackage[utf8]{inputenc}
\usepackage{amsmath}
\usepackage{amssymb}
\usepackage{amsthm}
\usepackage{tikz-cd}

\newtheorem{theorem}{Theorem}
\newtheorem{lemma}[theorem]{Lemma}
\newtheorem{exercise}{Exercise}
\newtheorem{definition}{Definition}
\newtheorem*{remark}{Remark}
\newtheorem*{problem}{Problem}

\newcommand{\legendre}[2]{\genfrac{(}{)}{}{}{#1}{#2}}
\newcommand{\jacobi}[2]{\genfrac{(}{)}{}{}{#1}{#2}}
\newcommand{\kronecker}[2]{\genfrac{(}{)}{}{}{#1}{#2}}

\title{Week 11 Lecture Notes: Extensions of the Legendre Symbol}
\author{Edward Zeng}
\date{April 2019}

\begin{document}
\maketitle

\subsubsection{Standing Assumptions}
Unless otherwise specified:
\begin{itemize}
    \item all variables are integers; further restrictions depend on context
    \item $p$ and $q$ \textbf{denote primes}, and sometimes, odd primes
\end{itemize}

\section{The Jacobi Symbol}
\subsubsection{}
The Legendre Symbol provides a easy method to compute the existence of solutions for quadratic congruences, provided that we are able to factor with ease. This is not always the case, but we introduce an extension of the Legendre symbol that will solve this difficulty.

\subsubsection{}
\begin{definition}
    For positive odd $b$, we define the Jacobi Symbol
    \begin{equation}
        \jacobi{a}{b} = \legendre{a}{p_{1}}\cdots\legendre{a}{p_{n}}
    \end{equation}
    where $p_{1}...p_{n}$ is the prime decomposition of $b$.
\end{definition}
\begin{remark}
    Note the LHS of (1) is the Jacobi Symbol and the right side of (1) is a product of Legendre Symbols. The distinction between the two will depend on context, as they have identical symbols. However, if $b$ is an odd prime, then the Jacobi Symbol is just the Legendre Symbol, and this distinction will not be significant.
\end{remark}

\subsubsection{Properties}
The Jacobi symbols inherits the following properties from Legendre:
\begin{enumerate}
    \item $\jacobi{a}{bc} = \jacobi{a}{b}\jacobi{a}{c}$
    \item $\jacobi{ab}{c} = \jacobi{a}{c}\jacobi{b}{c}$
    \item $\jacobi{a+nc}{c} = \jacobi{a}{c}$
    \item $\jacobi{-1}{b} = 1$ when $b \equiv 1 \pmod{4}$ and -1 otherwise
    \item $\jacobi{2}{b} = 1$ when $b \equiv 1, 7 \pmod{8}$ and -1 otherwise
    \item $\jacobi{a}{b}\jacobi{b}{a} = (-1)^{\frac{a-1}{2}\frac{b-1}{2}}$ for odd $a,b$.
\end{enumerate}
Properties 1, 2, 3 are self-evident. Properties 4, 5 require a little more argument (even and odd numbers of certain factors), and we present a proof of properties 6.

\subsubsection{The Reciprocity of the Jacobi Symbol}
Let $a = p_{1}\cdots p_{n}$ and $b = q_{1}\cdots q_{m}$. Then
\begin{equation}
    \jacobi{a}{b} = \prod_{i,j}\jacobi{p_{i}}{q_{j}}
\end{equation}
\begin{equation}
    \jacobi{a}{b} = \prod_{i,j}\jacobi{q_{j}}{q_{i}}.
\end{equation}
Hence
\begin{equation}
    \jacobi{a}{b}\jacobi{b}{a} = \prod_{i,j}(-1) \textrm{ if both $p_{i}$ and $q_{j}$ are }3\pmod{4}
\end{equation}
If $a$ or $b$ is equal to $1 \pmod{4}$, then there would an even number of $p_{i}$ or $q_{j}$ equal to $3 \pmod{4}$, respectively. Then (4) would evaluate to 1. The other case is simple.
\begin{remark}
    Note that the Jacobi Symbols does not indicate whether $a$ is a quadratic residue or nonresidue modulo $b$!
\end{remark}

\subsubsection{Computation of Quadratic Reciprocity}
Suppose we want to compute the Legendre Symbol
\begin{equation*}
    \legendre{-2002}{99991}.
\end{equation*}
We take out the factors of $2$ and $-1$ (as they ``cause trouble") to get
\begin{equation}
    \legendre{2}{99991}\legendre{-1}{99991}\legendre{1001}{99991}.
\end{equation}
We compute the third term by noting that it is equal to its corresponding Jacobi Symbol. Hence, in terms of Jacobi Symbols, we have
\begin{equation}
\begin{split}
    \jacobi{1001}{99991} &= \jacobi{99991}{1001}
    = \jacobi{892}{1001} = \jacobi{223}{1001} = \jacobi{1001}{223}\\
    &= \jacobi{223}{109}  = \jacobi{5}{109} = \jacobi{109}{5} = \jacobi{4}{5} = 1.  
\end{split}
\end{equation}
\begin{remark}
    The above is just the Euclidean algorithm (plus factoring out 2's and -1's) except we have to track of signs! The advantage of this is that we didn't have to factor 1001 to apply quadratic reciprocity, as we would have done with Legendre.
\end{remark}
\subsubsection{}
There is an alternate, more natural, definition of the Legendre Symbol for co-prime numbers $a$ and odd $b$ (as the Jacobi evaluates to zero is a common factor is shared). Because
\begin{equation}
    a, 2a, ..., (b-1)a \pmod{b}
\end{equation}
is a permutation of
\begin{equation}
    1, 2, ..., (b-1) \pmod{b}
\end{equation}
we define $\jacobi{a}{b}$ as the sign of the permutation. It takes some work to show is the same definition as the one we use, and this leads to another proof of the Law of Quadratic Reciprocity.

\subsubsection{Primality Testing}
We can apply the relationship between the Jacobi and Legendre Symbols in testing for primes. If the LHS denotes the Jacobi symbol and the RHS denotes Legendre, and if $b$ is an odd prime, then
\begin{equation}
    \jacobi{a}{b} = \legendre{a}{b}
\end{equation}
where we can compute Legendre by $a^{\frac{b-1}{2}}$.

\section{The Kronecker Symbol}
\subsubsection{}
There is a further extension of the Jacobi Symbol for all integers $a$, $b$, known as the Kronecker Symbol, which is defined as follows:
\begin{definition}
    The Kronecker Symbol $\kronecker{a}{b}$ is equal to
    \begin{enumerate}
        \item 0 unless $(a,b) = 1$ for nonzero $a$, $b$
        \item 1 if $a \geq 0$ and b = -1
        \item -1 if $a < 0$ and b = -1
        \item 1 if $a = \pm 1$ and b = 0; otherwise 0 if $b = 0$
        \item otherwise equal to the Jacobi Symbol for odd positive $b$
    \end{enumerate}
    We also define the following properties:
    \begin{enumerate}
        \item $\kronecker{a}{2} = \kronecker{2}{a}$ if $a = 1, 7 \pmod{8}$ and $-\kronecker{2}{a}$ otherwise
        \item $\kronecker{a}{b} = \kronecker{a}{\pm1} \kronecker{a}{p_{1}} \cdots \kronecker{a}{p_{n}}$ when $b = \pm p_{1}\cdots p_{n}$.
        \end{enumerate}
\end{definition}
\begin{remark}
    As it is a little messy to remember the properties of Kronecker (quadratic reciprocity requires more casework, etc), is is often easier to do most the Kronecker computation with the Jacobi.
\end{remark}
\subsubsection{Application of the Kronecker Symbol}
\begin{definition}
    Let the L-function of quadratic field of discriminant $d$ to be
    \begin{equation}
        L_{d}(s) = \sum_{n > 0}\frac{\kronecker{d}{n}}{n^{s}}
    \end{equation}
    where $\kronecker{d}{n}$ is the Kronecker Product.
\end{definition}
Recall the Riemann Zeta function
\begin{equation}
    \zeta(s) = \frac{1}{1^{s}} + \frac{1}{2^{s}} \dots
    = \frac{1}{1-2^{-s}} \cdot \frac{1}{1-3^{-s}} \cdots
\end{equation}
Note that $\kronecker{-4}{n} = 0$ for even $n$, $\kronecker{-4}{n} = -1$ for $n = 4m + 3$ and $\kronecker{-4}{n} = 1$ for $n = 4m + 1$. Thus
\begin{equation}
    L_{-4}(s) = \frac{1}{1^{s}} - \frac{1}{3^{s}} + \dots
            = \frac{1}{1+3^{-s}} \cdot \frac{1}{1-5^{-s}} \cdots.
\end{equation}
It takes a little work (proof was not shown in class) to show $L_{-4}(1) = \frac{\pi}{4}$, which is finite and nonzero. We note that $\zeta(1)$ and $L_{-4}(1)$ differ by a factor of
\begin{equation}
    \prod_{p = 4m + 3}\frac{1-p^{-1}}{1+p^{-1}}.
\end{equation}
As $\zeta(1)$ is infinite, then so is (13), and thus there are infinite primes $3 \pmod{4}$.

\subsubsection{}
Similarly, consider
\begin{equation}
    \zeta(s)L_{-4}(s) = \prod_{p = 4n+1}\frac{1}{(1-p^{-2})^{2}}\dot \prod_{p = 4m+1}\frac{1}{1-p^{-2}}.
\end{equation}
The second product is finite because it is less than the sum
\begin{equation}
    \sum \frac{1}{n^{2}}.
\end{equation}
But as the LHS of (14) is infinite, thus the first product of the RHS is infinite, and hence there are inifite primes $1 \pmod{4}$. This type of argument will be further refined for Dirichlet series.
\end{document}