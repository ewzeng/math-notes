\documentclass[12pt]{article}
\usepackage[utf8]{inputenc}
\usepackage{amsmath}
\usepackage{amssymb}
\usepackage{amsthm}

\newtheorem*{theorem}{Theorem}
\newtheorem*{definition}{Definition}
\newtheorem{claim}{Claim}
\newtheorem*{remark}{Remark}

\title{A Proof Sketch of the Prime Number Theorem (incomplete)}
\author{Edward Zeng}
\date{May 14, 2019}

\begin{document}
\maketitle
\section{Introduction and Overview}

\section{Analytic Continuation}
In complex analysis, certain \textit{nice} functions can be \textbf{continued}, that is, have their domain extended uniquely. Dirichlet series, and in particular the zeta function, can have their domains extended from the real line to the complex plane.
\begin{claim}
    If $f(s)$ is an extended Dirichlet series that converges somewhere, then there exists real $s_{0}$ such that if $\Re(s) > s_{0}$ then $f(s)$ converges and if $\Re(s) < s_{0}$ then $f(s)$ diverges. 
\end{claim}

\section{Continuation of the Zeta Function}
We will define a hand-wavy partial extension of the zeta function. Note for $s > 1$ we have
\begin{equation}
    \zeta(s) - 2\frac{\zeta(s)}{2^{s}} = 
    \frac{1}{1^{s}} - \frac{1}{2^{s}} + \frac{1}{3^{s}} - ...
\end{equation}
Also note the RHS converges by AST for all real $s > 0$. If we denote the RHS by $\mu(s)$, then we can define $\zeta(s)$ for all positive $s$ not equal to 1 by
\begin{equation}
    \zeta(s) = \frac{\mu(s)}{1 - 2^{1-s}}.
\end{equation}
We extend this definition to all complex numbers with $\Re(s) > 0$ such that
\begin{equation}
    1 - 2^{1-s} \neq 0.
\end{equation}
Observe (3) implies that
\begin{equation}
    s \neq 1 - \frac{2\pi k}{\log(2)}.
\end{equation}
To define $\zeta(s)$ for such $s$ other than $s = 1$, we can consider the equation
\begin{equation}
    \zeta(s) - 2\frac{\zeta(s)}{3^{s}} = 
    \frac{1}{1^{s}} + \frac{1}{2^{s}} - \frac{1}{3^{s}} + ...
\end{equation}
If we denote the RHS of (5) by $\mu_{3}(s)$, then we define
\begin{equation}
    \zeta(s) = \frac{\mu_{3}(s)}{1 - 2^{1-s}}.
\end{equation}
for all $s \neq 1$ that do not satisfy (4).
\begin{remark}
    These definitions require (2) and (6) to be convergent when their denominators are nonzero as $\Re(s) > 0$. We will assume this is true. The careful reader will also note that $\zeta(s)$ is undefined for $s = 1$. We simply define it to be $\infty$.
\end{remark}

\section{$\Re(s) > 1$}
From now on, $\zeta(s)$ will denote the analytic continuation of the original Dirichlet series. We would like to show that if $\Re(s) > 1$ then $\zeta(s) \neq 0$.

\begin{claim}
    The Euler product
    \begin{equation}
        \zeta(s) = (\frac{1}{1 - 2^{-s}})(\frac{1}{1 - 3^{-s}})...
    \end{equation}
    still converges and holds for complex $s$ if $\Re(s) > 1$.
\end{claim}
By taking log on both sides of (7), we get
\begin{equation}
    \log \zeta(s) = -\sum_{p} \log(1 - p^{-s}) \approx \sum_{p}p^{-s}
\end{equation}
by the Taylor expansion of log. As (we claim) the RHS is finite for complex $s$ if $\Re(s) > 1$, thus we reach our desired conclusion.

\section{$\Re(s) = 1$}
The more tricky part is this: we would like to show that if $\Re(s) = 1$, then $\zeta(s) \neq 0$. Consider the artificially constructed function
\begin{equation}
    f(s) = \zeta(s - 2it)\zeta(s - it)^{4}\zeta(s)^{6}
    \zeta(s + it)^{4}\zeta(s + 2it).
\end{equation}
By (once again) the Taylor expansion of log, we get
\begin{equation}
    \log \zeta(s) = \sum_{p, n}\frac{p^{ns}}{n}
\end{equation}
hence
\begin{equation}
    \log f(s)
    = 
    \sum_{p,n}\frac{p^{ns}}{n}
    (
    p^{-2nit} + 4p^{-nit} + 6 + 4p^{nit} + p^{2nit} 
    ).
\end{equation}
However, this is equivalent to
\begin{equation}
    \sum_{p,n}\frac{p^{ns}}{n}(p^{nit} + p^{-nit})^{4}
\end{equation}
by the binomial expansion! Hence (11) is always positive and $f(s) \geq 1$ whenever $s > 1$.

\subsection{Notion of Order}
In complex analysis, there is a notation of the \textbf{order} of a zero (basically the multiplicity of the root). We will give partial definition:
\begin{definition}
If $f(z)=0$ and $n$ is the smallest positive integer with $f^{(n)}(z)=0$, then the zero $z$ of $f(z)$ has order $n$.
\end{definition}
\begin{claim}
    Suppose $\zeta(1 + it) = 0$. Then $\zeta(1 - it) \neq 0$ and $f(1 + it)$ has zero of order $\geq 4 - 6 + 4 > 0$.
\end{claim}
By claim 3, thus $f(1 + it) = 0$. However, by continuity of $f$, we have
\begin{equation}
    \lim_{s \rightarrow 1}f(s + it) = 0
\end{equation}
which contradicts our previous conclusion as $f(s) \geq 1$ when $s > 1$.

\section{Newman's Tauberian Theorem}
\end{document}