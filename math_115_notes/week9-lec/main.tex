\documentclass{article}
\usepackage[utf8]{inputenc}
\usepackage{amsmath}
\usepackage{amssymb}
\usepackage{amsthm}

\newtheorem{theorem}{Theorem}
\newtheorem{lemma}[theorem]{Lemma}
\newtheorem{exercise}{Exercise}
\newtheorem{definition}{Definition}
\newtheorem*{remark}{Remark}
\newtheorem*{problem}{Problem}

\title{Week 9 Lecture Notes: Abstract Algebra}
\author{Edward Zeng}
\date{March 2019}

\begin{document}
\maketitle

\subsubsection{Standing Assumptions}
$G$ and $H$ denote groups, $R$, $I$ denote rings, and $p$, $q$ denote primes, and all other variables denote integers unless otherwise specified.

\section{Groups}
\subsubsection{}
\begin{definition}
    A group is a set $G$ with a binary operation $\cdot$ that has an unique identity, and is invertible (every element as an unique inverse), associative, and closed.
\end{definition}
In this course, we will be dealing with abelian (commutative) groups. The binary operation of a group is often denoted as either multiplication (as in this case) or addition, the choice often depends on context.
\begin{definition}
    A subgroup $(H, \cdot)$ is a subset of the group $(G, \cdot)$ that contains the identity and is closed.
\end{definition}
When the binary operation is obvious from context, we will often drop the symbol $\cdot$ and simply write $G$ for $(G, \cdot)$.
\begin{definition}
    $(Z/nZ)$ denotes the group of residue classes under $+$. $(Z/nZ)^{*}$ denotes the group of residue classes coprime to $n$ under $\cdot$.
\end{definition}
We now see a few familiar definitions, included here for completeness.
\begin{definition}
    If $a$ is an element of $G$, then the group $\{ ..., a^{-2}, a^{-1}, 1, a^{1}, ... \}$ is the subgroup generated by $a$.
\end{definition}
\begin{definition}
    The order of $a$ is the smallest $n > 0$ such that $a^{n} = 1$. We denote $n = \infty$ if $n$ does not exist.
\end{definition}
\begin{definition}
    The order of $G$ is its cardinality.
\end{definition}
Note the order of the subgroup generated by $a$ is equal to the order of $a$.

\subsubsection{}
\begin{theorem}[Lagrange]
    If $H$ is a subgroup of finite group $G$, then the order of $H$ divides the order of $G$.
\end{theorem}
This is the group version of Fermat and Euler's Theorems. In Week 3, we proved a version of this theorem that required commutativity.
\begin{definition}
    A coset of $H$ is a set obtained by multiplying every element of $H$ by an element $a$ in $G$.
\end{definition}
Now we are ready for a proof sketch:
\begin{proof}
We note
\begin{enumerate}
    \item Any 2 cosets of $H$ are identical or disjoint. Proof comes from showing if two cosets have a common element, then they are equal.
    \item All cosets of $H$ have the same size as $H$. This is because of invertibility.
    \item Every element of $G$ is in some coset.
\end{enumerate}
It follows that the order of $G$ is equal to the order of $H$ multiplied by the number of distinct cosets of $H$.
\end{proof}

\subsubsection{}
\begin{definition}
    A cyclic group is a group that consists of all powers of some generator $g$.
\end{definition}
\begin{definition}
    A morphism between two groups $(G, \cdot)$ and $(H, *)$ is a mapping $f$ such that
    \begin{equation}
        f(ab) = f(a) * f(b).
    \end{equation}
    A $f$ is bijective, we call $f$ an isomorphism.
\end{definition}
Note $(Z/pZ)*$ is isomorphic to $(Z/(p-1)Z)$ because we can define a morphism between the an element in $(Z/pZ)*$ and its corresponding power for some given primitive root.

\begin{definition}
    For abelian $G$, the quotient group $G/H$ consists of pairs of elements is the group where the elements of $H$ are consider to be trivial, i.e. we define $a = b$ if $ab^{-1} \in H$. 
\end{definition}
It takes some routine checks to show that the quotient group is well-defined with the inherited binary operation. If $G$ is not abelian, then we would require $H$ to be \textit{normal}.
\begin{definition}
    The product group $G \times H$ consists of pairs of elements $(g, h)$ where the binary operation is pairwise inherited. 
\end{definition}
\begin{remark}
    It is not always the case that $G/H \times H \cong G$. For example, take $G = Z$, $H = 2Z$.
\end{remark}

\subsubsection{}
\begin{theorem}[Chinese Remainder Theorem Group Theory Version]
    $(Z/mnZ)$ is isomorphic to $(Z/mZ) \times (Z/nZ)$ if $(m,n) = 1$.
\end{theorem}
We now state a theorem without proof (see Math 113!) that characterizes all finite abelian groups.
\begin{theorem}
    Any finite abelian group is isomorphic to a product of cyclic groups $(Z/p^{n}Z)$.
\end{theorem}

\subsubsection{}
\begin{exercise}
What is the smallest $n$ such that $x^{n} = 1 \pmod{10^6}$ for all $x$?
\end{exercise}
We note 
\begin{equation}
    (Z/10^{6}Z)^{*} = (Z/2^{6}Z)^{*} \times (Z/5^{6}Z)^{*}
\end{equation}
(an special version of the Chinese Remainder Theorem for reduced residues). By Theorem 3, we know that the last term on the RHS is the product of cyclic groups of the form $(Z/p^{n}Z)$. However, as
\begin{equation}
    (Z/5^{6}Z)^{*} \equiv (Z/4\cdot5^{5}Z)  
\end{equation}
hence $(Z/5^{6}Z)^{*}$ is a product of cyclic groups with orders of either $4$ or $5^{5}$ (why?). We also know
\begin{equation}
    (Z/2^{6}Z)^{*} = \{-1, 1\} \times \{\textrm{powers of 5 mod }2^{6}\}
\end{equation}
where the second term has order $2^{4}$ (see page 105 of Niven's book). Hence $(Z/10^{6}Z)^{*}$ is the product of cyclic groups with orders 4, $5^{5}$, 2, and $2^{4}$ (there can be cyclic groups with the same order). Thus $(Z/10^{6}Z)^{*}$ has a cycle of $2^{4} \cdot 5^{5}$.
\begin{remark}
    This exercise shows that Euler's Totient Function does not provide the smallest order $n$ for all $x$, and is a rather lazy estimate.
\end{remark}
\subsubsection{}
We end this section with a generalization of Wilson's Theorem: What is the product of all the elements of a finite abelian group $G$?
If all elements pair up with each other as inverses, then the product is obviously 1. If there exists exactly one element $c \neq 1$ that is the inverse of itself, then the product is $c$. Suppose these are more than two elements with an order of 2; note they form a group (!) which we will denote as $H$. Then $H$ is isomorphic of the product of groups:
\begin{equation}
    (Z/2Z) \times (Z/2Z) \times ... (Z/2Z).
\end{equation}
Multiplying all elements in $H$ together is equivalent to summing all the elements in (5), which is not hard to see to yield $$(0,0,0...0).$$ Hence the product of all elements in $H$ is 1. Therefore, we have our generalization:
\begin{theorem}[Generalization of Wilson's Theorem]
    The product of all elements in a finite abelian group is either 1 or $c$ where $c$ is the unique element other than the identity with an order of $2$. 
\end{theorem}

\subsection{Rings and Fields}
\subsubsection{}
\begin{definition}
    A ring $(R, +, \cdot)$ is a set of at least two elements such that $(R, +)$ is an abelian group, $(R, \cdot)$ is closed and associative and distributive property with respect to $+$. 
\end{definition}
\begin{definition}
    A field is a ring such that the nonzero elements form an abelian group under $\cdot$.
\end{definition}

\subsubsection{}
\begin{definition}
    Suppose $I$ is a subgroup of $R$ such that $xy \in I$ if either $x$ or $y$ is an element of $I$. Then $I$ is called an ideal.
\end{definition}
    If $I$ is an ideal of $R$, then $R/I$ is a well defined ring with the inherited binary operations and the equivalence relations (why?). This ring is known as the \textit{quotient} ring.
    
\subsubsection{}
Note that if $k$ is a field, then the polynomials with coefficients $k$ form a ring. In particular, the polynomials with coefficients in $(Z/pZ)$ form a ring. These polynomials, denoted $(Z/pZ)[x]$, will be a point of emphasis in these notes.

\subsubsection{}
In fact, $Z$ and $(Z/pZ)[x]$ share many (ring) properties:
\begin{itemize}
    \item division algorithm with remainder
    \item primes/irreducible polynomials
    \item unique factorization up to ordering and units (units are elements with inverses, e.g. $\pm 1$ for $Z$)
    \item if $f$ is irreducible, then $(Z/pZ)[x]/f$ is a (finite) field
    \item Sieve of Eratosthenes (for polynomials too!)
\end{itemize}

\subsubsection{}
We denote $R^{*}$ as the units of $R$, which forms a multiplicative group.
\begin{theorem}
    If $k$ is any field, and $G$ is a finite subgroup of $k^{*}$, then $G$ is cyclic.
\end{theorem}
The proof is similar to showing the existence of primitive roots in $(Z/pZ)$.
\begin{proof}
We give a sketch:
\begin{enumerate}
    \item Any polynomial of degree $n$ has $\leq n$ roots (duplicates not included) as $k$ is a field (by division algorithm)
    \item From (1), for any $n$, $G$ has $\leq n$ elements order $n$.
    \item From (2), show $G$ is cyclic.
\end{enumerate}
\end{proof}

\subsubsection{}
We have another generalization of Fermat:
\begin{theorem}
    If $k$ is any finite field with $p^{n}$ elements, then $x^{p^{n}} = x$.
\end{theorem}
\begin{proof}
This is clearly true if $x = 0$. If $x \neq 0$, then $x$ is an element of the group $k^{*}$, and hence by Lagrange, $x^{p^{n}} = x$.
\end{proof}
We give Theorem 6 more significance with the following remark:
\begin{remark}
    We will state but not prove the following three interesting results:
    \begin{itemize}
        \item The number of elements of any finite field is a power of $p$.
        \item For any $p^{i}$, there is a finite field of this order
        \item 2 finite fields with the same order are isomorphic
    \end{itemize}
\end{remark}

\subsubsection{Application}
\begin{problem}
    Show that a prime $p$ is of the form $m^{2} + n^{2}$ if $p = 1 \pmod{4}$.
\end{problem}
We use the ring of Gaussian integers to solve.
\begin{remark}
    One possible formal construction of the ring of Gaussian integers is the ring of polynomials $Z[i]/(i^{2}+1)$. This approach allows us to use properties of polynomial rings mentioned above.
\end{remark}
 We have the following (ring) properties for Gaussian integers:
\begin{itemize}
    \item There are Gaussian primes (irreducible polynomials).
    \item Gaussian integers have a unique factorization into Gaussian primes (Euclidean algorithm for polynomials).
\end{itemize}
Recall that $p = 4n + 1$ implies $p \mid m^{2} + 1$ for some $m$. As $p \nmid m - i$ and $p \nmid m + i$, thus $p$ is not a Gaussian prime and thus can be written as the product of two (non-unit) Gaussian integers $xy$. As $x$ and $y$ have imaginary parts, hence $x = a + bi$ and $y = a - bi$ for some $a$, $b$ and therefore $p = a^{2} + b^{2}$.
\end{document}