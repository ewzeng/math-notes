\documentclass[12pt, a4paper]{article}
\usepackage[utf8]{inputenc}
\usepackage{amsmath,amssymb}
\usepackage{xcolor} %For colored text
\usepackage{parskip}
\usepackage{tcolorbox}
\usepackage{fancyhdr}

%Title and margin formatting
\usepackage[left=2cm,top=3cm,bottom=3cm,right=2cm]{geometry} %Customize margins.
\usepackage{titling} %Customize the position of the title.
\setlength{\droptitle}{-50pt} %Raise the position of the title.

%Macros for Greek Letters
\renewcommand{\a}{\alpha}
\renewcommand{\b}{\beta}
\renewcommand{\d}{\delta}
\newcommand{\D}{\Delta}
\newcommand{\e}{\varepsilon}
\newcommand{\g}{\gamma}
\newcommand{\G}{\Gamma}
\renewcommand{\l}{\lambda}
\renewcommand{\L}{\Lambda}
\newcommand{\s}{\sigma}
\renewcommand{\th}{\theta}
\renewcommand{\o}{\omega}
\renewcommand{\O}{\Omega}
\renewcommand{\S}{\Sigma}
\renewcommand{\t}{\tau}
\newcommand{\var}{\varphi}
\newcommand{\z}{\zeta}

%Macros for math cal letters
\newcommand{\cA}{{\mathcal A}}
\newcommand{\cB}{{\mathcal B}}
\newcommand{\cC}{{\mathcal C}}
\newcommand{\cD}{{\mathcal D}}
\newcommand{\cE}{{\mathcal E}}
\newcommand{\cF}{{\mathcal F}}
\newcommand{\cH}{{\mathcal H}}
\newcommand{\cI}{{\mathcal I}}
\newcommand{\cK}{{\mathcal K}}
\newcommand{\cL}{{\mathcal L}}
\newcommand{\cM}{{\mathcal M}}
\newcommand{\cN}{{\mathcal N}}
\newcommand{\cO}{{\mathcal O}}
\newcommand{\cP}{{\mathcal P}}
\newcommand{\cS}{{\mathcal S}}
\newcommand{\cT}{{\mathcal T}}
\newcommand{\cU}{{\mathcal U}}
\newcommand{\cV}{{\mathcal V}}
\newcommand{\cW}{{\mathcal W}}
\newcommand{\cY}{{\mathcal Y}}

%Macros for blackboard bold letters
\newcommand{\bZ}{{\mathbb Z}}
\newcommand{\bR}{{\mathbb R}}
\newcommand{\bC}{{\mathbb C}}
\newcommand{\bT}{{\mathbb T}}
\newcommand{\bN}{{\mathbb N}}
\newcommand{\bQ}{{\mathbb Q}}
\newcommand{\bF}{{\mathbb F}}

%Other macros
\renewcommand{\i}{\infty}

\begin{document}

Recently, Terrence Tao posted a few proofs of the following theorem on his blog:
\begin{tcolorbox}[colback=blue!5!white,colframe=blue!75!black]
    Let $A$ be an $n \times n$ Hermitian matrix, with eigenvalues $\l_1(A), \l_2(A), \dots, \l_n(A)$. Let $v_i$ be a unit eigenvector corresponding to the eigenvalue $\l_i(A)$, and let $v_{i,j}$ be the $j$-th component of $v_i$. Then
    \[
        |v_{i,j}|^2 \prod_{k = 1; k \neq i}^n (\l_k(A) - \l_i(A))
        = \prod_{k = 1}^{n-1}(\l_k(M_j) - \l_i(A))
    \]
    where $M_j$ is the $(n-1) \times (n-1)$ Hermitian matrix formed by deleting the $j$-th row and column from $A$.
\end{tcolorbox}
Here, I will provide an alternate proof.

\textbf{Proof.} We start by making a few simplifying assumptions. Set $j = 1$ and fix $i$. Note that if $\l_i(A) \neq 0$, we can instead consider the Hermitian matrix $A - \l_i(A)I$, so we may suppose $\l_i(A) = 0$. Therefore, the identity becomes
\[
    |v_{i,1}|^2 \prod_{k = 1; k \neq i}^n \l_k(A)
    = \det(M_1). 
\]
Recall that every Hermitian operator has a basis of eigenvectors. Therefore, if 0 is an eigenvalue of multiplicity $>$ 1, then the kernel of $A$ has dimension $>$ 1. As a result, any principal minor has a kernel with dimension $>$ 0, and the above identity trivially holds. Hence we may assume $A$ has all nonzero eigenvalues except $\l_i(A)$.

Suppose $A = (a_{mn})$. Consider now the matrix
\[
    A_t = 
    \begin{pmatrix}
        a_{11} + t  & a_{12}    & \dots\\
        a_{21}      & a_{22}    & \dots\\
        \vdots      & \vdots    & \ddots
    \end{pmatrix}.
\]
Observe that $\det(A_t) = t\det(M_1)$. Thus, to prove the theorem, it suffices to show
\[
    \frac{\det(A_t)}{
    t|v_{i,1}|^2 \prod_{k = 1; k \neq i}^n \l_k(A)
    } \rightarrow 1
\]
as $t \rightarrow 0$. We do just that. (Remark: the denominator is nonzero. Why?).

From linear algebra, we know that every Hermitian matrix can be diagonalized by a change of basis represented by a unitary matrix. In particular, if $v_1, v_2, \dots, v_n$ form a basis of unit eigenvectors, then
\[
    \begin{pmatrix}
        \overline{v_{1,1}}   & \overline{v_{1,2}} & \dots\\
        \overline{v_{2,1}}   & \overline{v_{2,2}} & \dots\\
        \vdots      & \vdots    & \ddots
    \end{pmatrix}
    A
    \begin{pmatrix}
        v_{1,1} & v_{2,1} & \dots\\
        v_{1,2} & v_{2,2} & \dots\\
        \vdots      & \vdots    & \ddots
    \end{pmatrix}
    =
    \begin{pmatrix}
        \l_1(A) & & \\
        & \l_2(A) & \\
        & & \ddots
    \end{pmatrix}.
\]

Now note that
\[
    \begin{pmatrix}
        \overline{v_{1,1}}   & \overline{v_{1,2}} & \dots\\
        \overline{v_{2,1}}   & \overline{v_{2,2}} & \dots\\
        \vdots      & \vdots    & \ddots
    \end{pmatrix}
    A_t
    \begin{pmatrix}
        v_{1,1} & v_{2,1} & \dots\\
        v_{1,2} & v_{2,2} & \dots\\
        \vdots      & \vdots    & \ddots
    \end{pmatrix}
    =
    \begin{pmatrix}
        \l_1(A) & & \\
        & \l_2(A) & \\
        & & \ddots
    \end{pmatrix}
    +
    \begin{pmatrix}
        t\overline{v_{1,1}}v_{1,1}   & t\overline{v_{1,1}}v_{2,j} & \dots\\
        t\overline{v_{2,1}}v_{1,1}   & t\overline{v_{2,1}}v_{2,j} & \dots\\
        \vdots                  & \vdots                & \ddots
    \end{pmatrix}.
\]

We may rewrite the resulting matrix as
\[
    B_t = 
    \begin{pmatrix}
        t|v_{1,1}|^2 + \l_1(A)   & t\overline{v_{1,1}}v_{2,j} & \dots\\
        t\overline{v_{2,1}}v_{1,1}   & t|v_{2,j}|^2 + \l_2(A) & \dots\\
        \vdots                  & \vdots                & \ddots
    \end{pmatrix}.
\]
Keeping in mind that $\l_i(A) = 0$, we get
\[
    \det(B_t) = t|v_{1,i}|^2 \prod_{k = 1; k \neq i}^n \l_k(A)
    + t^2\Big( \cdots \cdots \Big).
\]
As $\det(A_t) = \det(B_t)$, thus we have shown what we set out to prove.
    


\end{document}
