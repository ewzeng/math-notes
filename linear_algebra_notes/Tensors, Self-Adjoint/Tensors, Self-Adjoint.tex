\documentclass[12pt, a4paper]{article}
\usepackage[utf8]{inputenc}
\usepackage{amsmath,amssymb}
\usepackage{xcolor} %For colored text
\usepackage{parskip}

%Title and margin formatting
\usepackage[left=2cm,top=3cm,bottom=3cm,right=2cm]{geometry} %Customize margins.
\usepackage{titling} %Customize the position of the title.
\setlength{\droptitle}{-50pt} %Raise the position of the title.

%Macros for Greek Letters
\renewcommand{\a}{\alpha}
\renewcommand{\b}{\beta}
\renewcommand{\d}{\delta}
\newcommand{\D}{\Delta}
\newcommand{\e}{\varepsilon}
\newcommand{\g}{\gamma}
\newcommand{\G}{\Gamma}
\renewcommand{\l}{\lambda}
\renewcommand{\L}{\Lambda}
\newcommand{\s}{\sigma}
\renewcommand{\th}{\theta}
\renewcommand{\o}{\omega}
\renewcommand{\O}{\Omega}
\renewcommand{\S}{\Sigma}
\renewcommand{\t}{\tau}
\newcommand{\var}{\varphi}
\newcommand{\z}{\zeta}

%Macros for math cal letters
\newcommand{\cA}{{\mathcal A}}
\newcommand{\cB}{{\mathcal B}}
\newcommand{\cC}{{\mathcal C}}
\newcommand{\cD}{{\mathcal D}}
\newcommand{\cE}{{\mathcal E}}
\newcommand{\cF}{{\mathcal F}}
\newcommand{\cH}{{\mathcal H}}
\newcommand{\cI}{{\mathcal I}}
\newcommand{\cK}{{\mathcal K}}
\newcommand{\cL}{{\mathcal L}}
\newcommand{\cM}{{\mathcal M}}
\newcommand{\cN}{{\mathcal N}}
\newcommand{\cO}{{\mathcal O}}
\newcommand{\cP}{{\mathcal P}}
\newcommand{\cS}{{\mathcal S}}
\newcommand{\cT}{{\mathcal T}}
\newcommand{\cU}{{\mathcal U}}
\newcommand{\cV}{{\mathcal V}}
\newcommand{\cW}{{\mathcal W}}
\newcommand{\cY}{{\mathcal Y}}

%Macros for blackboard bold letters
\newcommand{\bZ}{{\mathbb Z}}
\newcommand{\bR}{{\mathbb R}}
\newcommand{\bC}{{\mathbb C}}
\newcommand{\bT}{{\mathbb T}}
\newcommand{\bN}{{\mathbb N}}
\newcommand{\bQ}{{\mathbb Q}}
\newcommand{\bF}{{\mathbb F}}

%Other macros
\renewcommand{\i}{\infty}

\title{Tensors, Self Adjoint Linear Operators}
\author{Edward Zeng}
\date{October 19, 2019}

\begin{document}

\maketitle

Today, I decided to take a look at Halmos's book \textit{Finite-Dimensional Vector Spaces}. I was pleasantly surprised by the presentation of the material, and I would like to share a few observations and notes.

\section*{Tensors}

Recall the construction of the tensor product of vector spaces $U$ and $V$ goes something like this:
\[
    U \times V \rightarrow \bF \langle U \times V \rangle \rightarrow
    U \otimes V
\]
where the last step involves quotienting out a subspace. (Here, $\bF \langle . \rangle$ denotes the free space.) I've always wanted see alternate contruction, which is why I was excited to see Halmos's simple (and obvious!) solution.

\textbf{Definiton.} $U \otimes V$ is the space of bilinear forms $f: U \oplus V \rightarrow \bF$.

Similarly, we can define the $k$-fold exterior algebra of $U$ as the space of alternating multilinear forms on $U \oplus U \oplus \dots \oplus U$. Personally, I prefer this simpler construction over the one I was taught. 

\section*{Self-Adjoint Linear Operators}
Recall now that a linear map $A: U \rightarrow V$ induces the \textbf{dual map} $A': V' \rightarrow U'$. Halmos calls this the \textbf{adjoint}. The key result here is that the matrix of $A$ with respect with some choice of bases for $U$ and $V$ is the transpose of the matrix $A'$ with respect to the corresponding dual bases.

Now, if $U$ is a (finite-dimensional) inner product space, there is a natural bijection between $U$ and $U'$. More specifically:

\textbf{Theorem.} Suppose $y' \in U'$, where $U'$ is the dual space of a (finite-dimensional) inner product space $U$. There there exists a unique vector $y \in U$ such that
\[
    y'(x) = (x, y), \quad \forall x \in U.
\]
(Here, (.,.) denotes the inner product.)

In fact, the bijection between $U$ and $U'$ is a conjugate isomorphism, i.e. it happens that
\[
    (ay')(x) = (x, \bar{a}y).
\]
It also happens that this bijection induces an inner product on $U'$, so we will often denote the dual space $U'$ as $U^*$ instead to emphasize this. We observe now that for any linear operator $A: U \rightarrow U$, our usual definition of the adjoint $A': U^*\rightarrow U^*$ induces a linear map $A^*: U \rightarrow U$ via the identification between $U$ and $U^*$. (Yes, the notation is slightly off. $A'$ is an operator on $U^*$. $A^*$ is an operator on $U$.)

Unlike before, the matrix of $A^*$ is the \textit{conjugate} transpose of the matrix of $A$ (with respect to corresponding bases). This is where the idea of conjugate transpose comes from! (To be sure, the matrix of $A'$ is the transpose of $A$, but the matrix of $A^*$ is the conjugate transpose of $A$, even though $A'$ naturally induces $A^*$.)

Halmos makes an important heuristic observation here: 
\[
    \text{\textit{the space of linear operators on $U$ acts like the field of complex numbers.}}
\]
For instance, $A \mapsto A^*$ is analogous to conjugation. From this, we deduce that the set of self-adjoint operators (i.e. \textbf{Hermitian matrices}) is analogous to the set of real numbers! (Similarly, the unitary matrices are analogous to the unit circle, the skew-symmetric matrices are analogous to imaginary numbers, etc.)

From this, I am starting to understand the reason why Hermitian matrices were so emphasized in the study of Toeplitz forms. Halmos's heuristic analogy isn't perfect (division breaks down a little), but does hold some fundamental truths. Hint: eigenvalues of Hermitian matrices are always real.

\end{document}
