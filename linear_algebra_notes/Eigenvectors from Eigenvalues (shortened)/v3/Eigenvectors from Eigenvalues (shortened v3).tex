\documentclass[12pt, letterpaper]{article}
\usepackage[utf8]{inputenc}
\usepackage{amsmath,amssymb}
\newtheorem{lemma}{Lemma}

\begin{document}

\textbf{Proof.} Set $j = 1$. WLOG assume $\lambda_i(A) = 0$. Then the formula to be proved is
\begin{equation}
    |v_{i,1}|^2 \prod_{k = 1; k \neq i}^n \lambda_k(A)
    = \det(M_1). 
\end{equation}

First we note any Hermitian matrix $A$ can be diagonalized by a unitary transformation
\[
    T^*AT = 
    \begin{pmatrix}
        \lambda_1(A) & & \\
        & \lambda_2(A) & \\
        & & \ddots
    \end{pmatrix}
\]
where the column vectors of $T$ form a basis of unit eigenvectors,
\[
    T = (\vec{v}_1, \vec{v}_2, \dots, \vec{v}_n).
\]
Now observe the following lemma, which follows directly from how determinants are computed.

\begin{lemma}
Let
\[
    A_t = A +
    \begin{pmatrix}
        t  & 0    & \dots\\
        0      & 0    & \dots\\
        \vdots      & \vdots    & \ddots
    \end{pmatrix}
    =
    \begin{pmatrix}
        a_{11} + t  & a_{12}    & \dots\\
        a_{21}      & a_{22}    & \dots\\
        \vdots      & \vdots    & \ddots
    \end{pmatrix}.
\]
Then $\det(A_t) = \det(A) + t\det(M_1)$.
\end{lemma}

Applying $T$-transformation to $A_t$, we obtain
\[
    T^*A_tT =
    \begin{pmatrix}
        \lambda_1(A) & & \\
        & \lambda_2(A) & \\
        & & \ddots
    \end{pmatrix}
    +
    \begin{pmatrix}
        t\overline{v_{1,1}}v_{1,1}   & t\overline{v_{1,1}}v_{2,j} & \dots\\
        t\overline{v_{2,1}}v_{1,1}   & t\overline{v_{2,1}}v_{2,j} & \dots\\
        \vdots                  & \vdots                & \ddots
    \end{pmatrix}.
\]
When we take the determinant on both sides (using Lemma 1 on the LHS) and collect the $t$-linear terms, we obtain (1).

\end{document}
