\section{Week 4}

\subsection{Localization}

Let $R$ be a ring and $S \subset R$ a multiplicative subset. There are two ways to see the localization $R[S^{-1}]$:
\begin{itemize}
    \item $R[S^{-1}]$ is the ring of fractions with denominators in $S$ (i.e. a partial construction of the field of fractions). This is the way I have always viewed it. Recall to make the construction simple, we assume $S$ has no zero divisors (alternatively, we can assume $R$ is an integral domain). Else the equivalence relation when defining fractions gets messy.
    \item $R[S^{-1}]$ is the ring that results when we try to make the elements in $S$ invertible. (Indeed, we can make an universal property out of this.)
\end{itemize}
The natural map $R \to R[S^{-1}]$ is injective if there are no zero divisors in $S$. If $S$ contains zero divisors, the direct construction gets messier, but it happens we can do an indirect 2-step construction:
\[
    R[S^{-1}] = \left( \frac{R}{I} \right)[S^{-1}], \quad I = \{a: as = 0 \text{ for some } s \in S\}.
\]
Localization is an extremely well-behaved operation (keeps many properties of the original ring) and thus a favored tool among algebraists.

\subsection{Localization and Spectrum}
Consider the natural map $f: R \to R[S^{-1}]$. Observe the following maps between the ideals of $R$ and $R[S^{-1}]$:
\begin{align*}
    I &\mapsto \text{the ideal generated by } f(I)\\
    J &\mapsto f^{-1}(J)
\end{align*}
It happens that the first map is the left inverse of the second, so the $\Spec(R[S^{-1}])$ can be seen as a subset of $\Spec(R)$. In fact, it is not diffcult to check that $\Spec(R[S^{-1}])$ is homeomorphic to that subset of $\Spec(R)$, and thus a subspace of $\Spec(R)$!

Geometrically, if we view $R$ (roughly) as the space of continuous functions on $\Spec(R)$, then
\[
    \Spec(R[S^{-1}]) = \Spec(R) - \{ \text{all the zeros of functions } f \in S \}.
\]
For finite $S$, this can be visualized as taking out a bunch of lines in a plane (or removing a bunch of points in a line).

The most important application of localization is when $S = R - \mathfrak p$, where $\mathfrak p$ is a prime ideal. We denote $R[S^{-1}] = R_{\mathfrak p}$ and call this ``the localization at $\mathfrak p$.'' In this case,
\[
    \begin{split}
        \Spec(R_{\mathfrak p}) & = \{\text{all points in $\Spec(R)$ whose closure contains } \mathfrak p\} \\
                               & = \{\text{$\mathfrak p$ and all points outside but nearby}\}
    \end{split}
\]
Thus localization turns $\mathfrak p$ into a 0-dimensional point (as $\mathfrak p$ becomes a maximal ideal), and allows us to study the neighborhood of this point. (We study the ``local'' area around this point in the spectrum. This is where the name comes from.)

Note we are studying the stuff around $\mathfrak p$ but not looking ``within'' $\mathfrak p$ itself. The spectrum of $R/\mathfrak p$ (which is the closure of $\mathfrak p$ in $\Spec(R)$) does exactly that. (In this case, $\mathfrak p$ becomes minimal and everything is contained in the closure.)

\subsection{Affine Schemes}
A scheme is a topological space that assigns each open set (or basis) a commutative ring. The affine scheme of the ring $R$ is $\Spec(R)$ equipped with the following map from basis open sets to local rings:
\[
    U(f) \mapsto O(U(f)) := R[f^{-1}]
\]
This mapping allows us to do operations to ``piece things together.'' More precisely, we have the following three properties:
\begin{enumerate}
    \item Restriction: if $U(f) \subset U(g)$, there is a natural ``restriction'' map $O(U(g)) \to O(U(f))$.
    \item Presheaf: if $U(f)$ is covered by $U(f_i)$'s and $g \in O(U(f))$ is 0 in the $O(U(f_i))$'s, then $g = 0$.
    \item Sheaf: if $U(f)$ is covered by $U(f_i)$'s, and there exists $r_i \in O(U(f_i))$ such that $r_i = r_j$ when restricted to $U(f_i) \cap U(f_j)$, then there exists $r \in U(f)$ such that $r = r_i$ when restricted to each $U(f_i)$.
\end{enumerate}
Intuitively, we are assigning a ring of functions to each (basis) open set. This ring of functions can be thought as defined on the corresponding open set. When we restrict/piece together these open sets together, we also restrict/piece together these rings of functions.



%%% Local Variables:
%%% TeX-master: "main"
%%% End: