\section{Week 4}

\subsection{Localization}

Let $R$ be a ring and $S \subset R$ a multiplicative subset. There are two ways to see the localization $R[S^{-1}]$:
\begin{itemize}
    \item $R[S^{-1}]$ is the ring of fractions with denominators in $S$ (i.e. a partial construction of the field of fractions). This is the way I have always viewed it. Recall to make the construction simple, we assume $S$ has no zero divisors (alternatively, we can assume $R$ is an integral domain). Else the equivalence relation when defining fractions gets messy.
    \item $R[S^{-1}]$ is the ring that results when we try to make the elements in $S$ invertible. (Indeed, we can make an universal property out of this.)
\end{itemize}
The natural map $R \to R[S^{-1}]$ is injective if there are no zero divisors in $S$. If $S$ contains zero divisors, the direct construction gets messier, but it happens we can do an indirect 2-step construction:
\[
    R[S^{-1}] = \left( \frac{R}{I} \right)[S^{-1}], \quad I = \{a: as = 0 \text{ for some } s \in S\}.
\]
Localization is an extremely well-behaved operation (keeps many properties of the original ring) and thus a favored tool among algebraists.

\subsection{Localization and Spectrum}
Consider the natural map $f: R \to R[S^{-1}]$. Observe the following maps between the ideals of $R$ and $R[S^{-1}]$:
\begin{align*}
    I &\mapsto \text{the ideal generated by } f(I)\\
    J &\mapsto f^{-1}(J)
\end{align*}
It happens that the first map is the left inverse of the second, so the $\Spec(R[S^{-1}])$ can be seen as a subset of $\Spec(R)$. In fact, it is not diffcult to check that $\Spec(R[S^{-1}])$ is homeomorphic to that subset of $\Spec(R)$, and thus a subspace of $\Spec(R)$!

Geometrically, if we view $R$ (roughly) as the space of continuous functions on $\Spec(R)$, then
\[
    \Spec(R[S^{-1}]) = \Spec(R) - \{ \text{all the zeros of functions } f \in S \}.
\]
For finite $S$, this can be visualized as taking out a bunch of lines in a plane (or removing a bunch of points in a line).

The most important application of localization is when $S = R - \mathfrak p$, where $\mathfrak p$ is a prime ideal. We denote $R[S^{-1}] = R_{\mathfrak p}$ and call this ``the localization at $\mathfrak p$.'' In this case,
\[
    \begin{split}
        \Spec(R_{\mathfrak p}) & = \{\text{all points in $\Spec(R)$ whose closure contains } \mathfrak p\} \\
                               & = \{\text{$\mathfrak p$ and all points outside but nearby}\}
    \end{split}
\]
Thus localization turns $\mathfrak p$ into a 0-dimensional point (as $\mathfrak p$ becomes a maximal ideal), and allows us to study the neighborhood of this point. (We study the ``local'' area around this point in the spectrum. This is where the name comes from.)

Note we are studying the stuff around $\mathfrak p$ but not looking ``within'' $\mathfrak p$ itself. The spectrum of $R/\mathfrak p$ (which is the closure of $\mathfrak p$ in $\Spec(R)$) does exactly that. (In this case, $\mathfrak p$ becomes minimal and everything is contained in the closure.)

\subsection{Affine Schemes}
Given a (compact Hausdorff) topological space $X$, there is natual map assigning each open set $U \subset X$ to the set $C(U)$, the continuous functions supported on $U$. Moreover, we have three (obvious) properties:
\begin{itemize}
    \item Restriction: if $U \subset V$, then there exists a restriction map $C(V) \to C(U)$.
    \item Presheaf: if $U$ is covered by the $U_i$'s, and $g \in U$ is 0 over the $U_i$'s, then $g = 0$ over $U$.
    \item Gluing: if $U$ is covered by the $U_i$'s, and there exists $r_i \in C(U_i)$ such that $r_i = r_j$ when restricted to $U_i \cap U_j$, then there exists $r \in C(U)$ such that $r = r_i$ when restricted to each $U_i$.
\end{itemize}

A sheaf of a topological space is an attempt to generalize this notion, assigning some data to each open set (or basis of open sets) such that when we restrict/glue together open sets, we also restrict/glue together this data. A scheme is a topological space with a sheaf that assigns each open set (or basis) a commutative ring. In particular, the affine scheme of the ring $R$ is $\Spec(R)$ equipped with the following map from basis open sets to local rings:
\[
    U(f) \mapsto O(U(f)) := R[f^{-1}]
\]
It takes some (boring) work to check this is indeed a sheaf.

\subsection{Tensor Products Review}
Let $M, N$ be modules over ring $R$. Recall there are two ways to view the tensor product $M \otimes N$:
\begin{itemize}
    \item $M \otimes N$ is the universal module for bilinear maps. That is, every bilinear map $M \times N \to X$ factors through a unique linear map $M \otimes N \to X$.
    \item From a practical point of view, $M \otimes N$ can be visualized as the $R$-module generated by the simple tensors with certain addition/multiplication rules. (This also reflects the construction of the tensor product as a quotient of a free $R$-module.)
\end{itemize}
Two important properties of tensor products are: $R \otimes M \cong M$ and $(M_1 \oplus M_2) \otimes N = (M_1 \otimes N) \oplus (M_2 \otimes N)$. The latter is not completely obvious.

\subsection{Tensor Products and Exactness}
The functor arising from tensor products is not exact. (This is a major problem in commutative algebra, and even has its own field, homological algebra, to deal with it.) To see this, consider the exact sequence of $\bZ$-modules
\[
    0 \to \bZ \xrightarrow{\times 2} \bZ \to \bZ/2\bZ \to 0
\]
and tensor it with $\bZ/2\bZ$. (Borcherds calls this exact sequence the ``universal counterexample.'')

However, tensor products are right exact. The proof is to take a right exact sequence, apply the $\Hom(\cdot, X)$ and $\Hom(M, \cdot)$ functors, notice
\[
    \Hom(M, \Hom(A,X)) = \Hom(M \otimes A, X),
\]
recall the exactness properties of the $\Hom$ functors, and then make some observations.

\subsection{Computing Tensor Products}
The exactness properties of tensor products gives us a trick to compute them. To compute $A \otimes_R M$ for some finitely generately $A$, note we can write
\[
    R^m \to R^n \to A \to 0
\]
and then apply the tensor product functor to get
\[
    M^m \to M^n \to A \otimes M \to 0.
\]
This allows us to rewrite $A \otimes M$ as a quotient.

Another way to compute tensor products is to take advantage of the property that tensor products commute with direct limits. That is,
\[
    \left( \varinjlim A \right) \otimes M = \varinjlim (A \otimes M).
\]


%%% Local Variables:
%%% TeX-master: "main"
%%% End: