\section{Week 5}

\subsection{Localization of Modules}
We can define localization of modules in the same manner as localization of rings: given a $R$-module $M$ and a multiplicative subset $S \subset R$, define $M[S^{-1}]$ to be the $R[S^{-1}]$-module of fractions $m/s$ where $m \in M$, $s \in S$, and $m_1/s_1 = m_2/s_2$ if $s(m_1s_2-s_1m_2) = 0$ for some $s \in S$.

One key result (that is not difficult to check) is that localization is an exact functor. As a natural consequence, localization preserves (i.e. commutes with) quotients, kernels, and direct sums.

\subsection{Flatness}
Call a $R$-module $M$ flat if the tensor product with $M$ is an exact functor. (Flat modules are very important in commutative algebra and algebraic geometry and behave very well.) By tensoring the exact sequence
\[
    0 \to R \xrightarrow{\times a} R \to R/aR \to 0,
\]
with $M$, one can conclude that flat modules are torsion-free (at least when $R$ is an integral domain).

The key link between tensor products and localization is the natural isomorphism
\[
    M[S^{-1}] \cong M \otimes_R R[S^{-1}], \quad ms^{-1} \mapsto m \otimes s^{-1}.
\]
Because localization is exact, we can conclude that $R[S^{-1}]$ (as a $R$-module) is flat.

\subsection{Local Properties}
Often times, global properties can be verified by checking them locally. We present three examples:
\begin{enumerate}
    \item $M = 0 \iff M_{\mathfrak p} = 0$ for all $\mathfrak p \in \Spec(R)$.
    \item $0 \to A \to B \to C \to 0$ exact $\iff$ $0 \to A_{\mathfrak p} \to B_{\mathfrak p} \to C_{\mathfrak p} \to 0$ exact for all $\mathfrak p \in \Spec(R)$. (Proof: use 1. and the fact that localization is exact).
    \item $M$ flat $\iff$ $M_{\mathfrak p}$ flat for all $\mathfrak p \in \Spec(R)$.
\end{enumerate}

\subsection{Flat Extensions}
Let $S$ be an $R$-algebra. For a $R$-module $M$, note the $R$-module $S \otimes_R M$ can also be viewed as an $S$-module (with  $a(s \otimes m) = (as) \otimes m$). This gives us a natural way to turn $R$-modules into $S$-modules.

On further study, one notes that for $R$-modules $M, N$, there is a natural map of $S$-modules
\[
    \a: S \otimes_R \Hom(M, N) \to \Hom_S(S \otimes_R M, S \otimes_R N).
\]
It happens that if $S$ is flat and $M$ is finitely presented, then $\a$ is an isomorphism. The proof consists of 3 steps:
\begin{enumerate}
    \item Prove $\a$ is an isomorphism when $M = R$. Extend to the case $M = R^m$ via direct sums.
    \item As $M$ is finitely presented, there exists an exact sequence
    \[
        R^m \to R^n \to M \to 0.
    \]
    \item Apply the tensor and Hom functors to the exact sequence in different ways to get two different exact sequences. Then use the result from 1 to relate the two exact sequences, and apply the Fives Lemma (or equivalently, diagram chase).
\end{enumerate}
A consequence of this result is that by setting $S = R[U^{-1}]$, we get the isomorphism
\[
    \Hom_{R[U^{-1}]}(M[U^{-1}], N[U^{-1}]) \cong \Hom_R(M,N)[U^{-1}].
\]
In other words, homomorphisms between localizations of nice modules come from homomorphisms between the original modules.

\subsection{Artinian Modules and Rings}
Call a module $M$ Artinian if every decreasing chain of modules stabilizes (equivalently, every set of modules has a minimal element). Call a ring $R$ Artinian if it is an Artinian $R$-module (just like the relationship between Noetherian rings and modules). Like Noetherian modules, Artinian modules also satisfy the exact sequence property.

\subsection{Noetherian and Artinian Modules}
Call a module simple if there are no nontrivial submodules (analog of 1-dim vector subspaces). Define a module $M$ to be of finite length if there exists a finite ascending chain of modules
\[
    0 \subset M_1 \subset M_2 \subset \dots \subset M_n = M
\]
where every $M_i/M_{i-1}$ is simple.

Let $M$ be both a Noetherian and Artinian $R$-module. By choosing a minimal submodule $M_1 \neq 0$, and then choosing a minimal submodule $M_2 \supset M_1$ and so on, we can show that $M$ is finite length. Conversely, if $M$ is of finite length, we can apply the exact sequence properties of Artinian and Noetherian modules to the ascending chain of modules iteratively to show that $M$ is Artinian and Noetherian. Thus, \textit{being both Noetherian and Artinian is equivalent to the finite length condition}.

In fact, if a module has finite length, then all maximal ascending chains have the same length (Borcherd's proof uses a grid argument). This is somewhat analogous to Jordan Holder theorem for groups and allows us to have a well-defined notion of ``dimension'' for modules of finite length. Furthermore, this ``dimension'' is additive on exact sequences (not hard to show). Thus, modules that are both Noetherian and Artinian are kind of like vector spaces.

\subsection{Artinian Rings}
It is an amazing fact that Artinian rings are automatically Noetherian. The proof consists of two main steps:
\begin{enumerate}
    \item We show for Artinian ring $R$, there exists maximal ideals such that
    \[
        \mathfrak m_1 \mathfrak m_2 \cdots \mathfrak m_n = 0.
    \]
    This proof is short and elementary but extremely hairy (probably the most hairy of this semester).
    \item From step 1, we can construct the following ascending chain
    \[
        0 = \mathfrak m_1 \mathfrak m_2 \cdots \mathfrak m_n \subset 
        \mathfrak m_1 \mathfrak m_2 \cdots \mathfrak m_{n-1} \subset \dots \subset \mathfrak m_1 \subset R.
    \]
    Observing that the quotients of consecutive elements are finite dimensional (by the Artinian condition), this allows us to construct a chain where the quotients are simple. Thus $R$ is of finite length and hence Noetherian.
\end{enumerate}

Suppose for an Artinian ring $R$, there exists maximal ideals $\mathfrak m_i$ such that
\[
    0 = \mathfrak m_1^{k_1}\cdots\mathfrak m_j^{k_j}
\]
Then by CRT, we have
\[
    R = \frac{R}{\{0\}} \cong \prod_i \frac{R}{\mathfrak m_i^{k_i}}
\]
As a consequence, we conclude that every Artinian ring is a product of local Artinian rings.

%%% Local Variables:
%%% TeX-master: "main"
%%% End: