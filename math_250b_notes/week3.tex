\section{Week 3}

\subsection{Why Called Spectrum?}
Consider the ring $\bC[A]$ generated as $\bC$-algebra by the matrix $A$. By Cayley-Hamilton,
\[
    \bC[A] \cong \frac{\bC[x]}{\text{characteristic polynomial of } A} =: R.
\]
After some work, we can show $\Spec(R)$ consists of the ideals generated by $(x - \a)$, where $\a$ is a eigenvalue. Thus $\Spec(\bC(A)) = \Spec(R)$ corresponds to the (matrix) spectrum of $A$.

\subsection{Spectrum Topology}
The spectrum topology defined earlier is called the Zariski topology. Here are some basic properties which can be easily verified:
\begin{itemize}
    \item $\Spec(R)$ is compact (sometimes called quasi-compact to emphasize it is non-Hausdorff).
    \item $\Spec(R)$ is connected if $R$ is an integral domain. (Proof: if $R$ is an integral domain, the $(0)$ is a prime ideal whose closure is the whole spectrum.)
    \item $\Spec(R \times S) = \Spec(R) \sqcup \Spec(S)$.
    \item Called a set \textbf{irreducible} if it is not the union of two proper closed sets. (Equivalently, any two nonempty open sets intersect, i.e. super non-Hausdorff.) As we will see, the spectrum is often a finite union of irreducible components.
\end{itemize}

\subsection{Closed Irreducible Subsets}
We can classify all closed irreducible subsets with the following result: the closed irreducible subsets of $\Spec(R)$ are precisely the sets $\bar{x}$ for $x \in \Spec(R)$. The proof consists of two main steps:
\begin{enumerate}
    \item Given a closed irreducible subset $Z(I)$, we define the radical
    \[
        \sqrt{I} = \{a | a^m \in I\}
    \]
    and show $\sqrt{I}$ is prime. This requires some fiddling with primality and irreducibility as well as a Zorn's Lemma contradiction.
    \item Then we show $Z(I) = \overline{\sqrt{I}}$.
\end{enumerate}

\subsection{Looking Back}
We motivated the definition of the spectrum with maximal ideals, and then changed to prime ideals to make things nicer. But how does our new definition work with our original motivation?

That is, given the ring of continuous functions $R$ on a compact Hausdorff space $X$, what is $\Spec(R)$? It happens that:
\begin{itemize}
    \item Each prime ideal is contained in exactly one maximal ideal.
    \item Closed prime ideals are maximal.
    \item Nonclosed prime ideals are weird. In fact, to prove they exist, we need Zorn's Lemma! In particular, the argument involves picking a maximal ideal disjoint from a certain subset. (This same argument was used in the proof that $Z(I)$ irreducible $\implies$ $\sqrt{I}$ prime.)
\end{itemize}
Thus, $\Spec(R)$ looks very similar to $\Spec_m(R) = X$. If we do not assume AOC, then $\Spec(R) = \Spec_m(R)$!

\subsection{Noetherian Spaces}
Call a topological space Noetherian if every ascending chain of open sets stabilizes. Equivalently: 1. every family of open sets has a maximal element, 2. every descending chain of closed sets of stabilizes, 3. every family of closed sets has a minimal element, 4. every open set is compact. (Note the fourth condition implies Noetherian spaces are super non-Hausdorff.)

Because there is a order-reserving relation between ideals $I$ and closed sets $Z(I)$, if $R$ is a Noetherian ring, then $\Spec(R)$ is a Noetherian topology. Noetherian induction works on Noetherian topologies in the following way:
\begin{enumerate}
    \item Suppose to show a property $P$ is true for all closed sets $C$, it suffices to show it is true for all proper closed sets of $C$.
    \item Then property $P$ is true for all closed sets $C$. (Proof: take a minimal closed set for which $P$ does not hold and derive a contradiction.)
\end{enumerate}
Using Noetherian induction, one can prove that every closed set in a Noetherian topology is the union of a finite number of closed irreducible subsets. Combined with the previous characterization of closed irreducible subsets, this allows us to completely characterize the closed sets in $\Spec(R)$ for $R$ Noetherian.

%%% Local Variables:
%%% TeX-master: "main"
%%% End: