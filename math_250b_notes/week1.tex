\section{Week 1}
Commutative algebra is a core subject intended to be used in fields like algebraic geometry, number theory, and (the somewhat out of fashion) invariant theory. In the old days, commutative algebra was the study of polynomial rings over a field. Over time, commutative algebra evolved into the study of commutative rings and related structures.

\subsection{Definitions}
The reader is assumed to familiar with basic algebraic concepts, like groups, rings, ideals, modules, homomorphisms, PIDs, UFDs, Galois theory, etc.

In these notes, rings are always commutative and have identity. The reason why people study rings without identity is because these structures often arise from analysis. In fact, rings with identity are analogous to compact spaces, and rings without identity are analogous to locally compact spaces.

\subsection{Invariant Rings}
Suppose group $G$ acts on space $X$. Then $G$ acts naturally on functions $f: X \to Y$ by defining
\[
    (gf)(x) = f(g^{-1}x).
\]
If $f = gf$, we say $f$ is an invariant.

Consider $G = \bZ/3\bZ$ acting on $\bC^2$ by $60^{\circ}$ rotation in each component. After doing some work, we can conclude that the ring of invariant polynomials over $\bC^2$ is the $\bC$-algebra generated by $z_0 = x^3, z_1 = x^2y, z_2 = xy^2, z_3 = y^3$.
\begin{itemize}
    \item Note we have the following relations (called 1st order syzygies):
    \[
        \begin{split}
            a_1 &= z_2^2 - z_1z_3 = 0\\
            a_2 &= z_0z_3 - z_1z_2 = 0\\
            a_3 &= z_1^2 - z_0z_2 = 0\\
        \end{split}
    \]
    These three equations entirely describe the relations between $z_0, z_1, z_2, z_3$.
    \item Treating $z_0, z_1, z_2, z_3$ as symbols, we also have the following equation (a 2nd order syzygy) that completes described the relation between $a_1, a_2, a_3$:
    \[
        b = z_1a_1 + z_2a_2 + z_3a_3 = 0.
    \]
\end{itemize}

Pictorally, we can describe our previous analysis of syzygies with the following exact sequence of $R$-modules:
\[
    0 \to R \to R^3 \to R \to P \to 0
\]
where
\begin{itemize}
    \item $R$ is the ring of polynomials $\bC[z_0, z_1, z_2, z_3]$ and $P$ is the ring of invariant polynomials over $\bC^2$ (the invariant ring).
    \item $R \to P$ is the natural map quotienting out the ideal generated by $a_1, a_2, a_3$. This is the ideal of 1st order syzygies.
    \item $R^3 \to R$ is the map sending $(1,0,0) \mapsto a_1$, $(0,1,0) \mapsto a_2$, $(0,0,1) \mapsto a_3$. The kernel is the ideal of 2nd order syzygies.
    \item $R \to R^3$ is the map sending $1 \mapsto (z_1, z_2, z_3)$.
\end{itemize}

\subsection{Classical Invariant Theory}
In the 19th century, a lot of effort was made to explicitly find generators of invariant rings for certain group actions. (Some generators would be polynomials spanning multiple pages. Being able to find ugly generators was a form of street cred among mathematicians in that era.)

In these complicated scenarios, the exact sequences obtained would look like
\[
    \dots \to R^* \to R^n \to R^m \to R \to P \to 0.
\]
Questions that were often asked were:
\begin{enumerate}
    \item Is $R$ finitely generated as a $\bC$-algebra? That is, does there exist a finite set of generators for $R$, and thus $P$? (Mathematicians of the 19th century would prove this by explicitly constructing generators, as described above.)
    \item Are the 1st order syzygies finitely generated (as an ideal)? What about $n$-th order syzygies (if they exist)?
    \item Is the chain of modules finite?
\end{enumerate}
My guess why people were interested in the first question is that people wanted to find analogs of the statement ``all symmetric polynomials are generated by the elementary symmetric polynomials,'' but for more complicated group actions. (The group action corresponding to symmetric polynomials is $S_n$.)

As we will see, Hilbert killed off the whole theory by answering these questions in a general way (without explicit construction of generators).

\subsection{Noetherian Rings}
Recall there are 3 equivalent definitions of Noetherian rings: 1. every ideal is finitely generated, 2. every ascending chain of ideals stabilizes, and 3. every set of ideals has a maximal element.

Being Noetherian is a nice but subtle property. Subrings of Noetherian rings are not necessarily Noetherian, but quotient rings are (by correspondence theorem for ideals).

\subsection{Hilbert's Basis Theorem}
Hilbert showed that if $R$ is finitely generated as a $\bC$-algebra, then the 1st order syzygies are finitely generated. More generally, he showed every polynomial ring (with a finite number of variables) over a field is Noetherian.

The proof is to show $S$ Noetherian $\implies$ $S[x]$ Noetherian, and iteratively apply this fact. To show $S[x]$ is Noetherian, we show any arbitrary ideal $I \subset S[x]$ is finitely generated. Define
\[
    \begin{split}
        I_n &= \{\text{the ideal of all leading coef of $n$-degree poly in } I \},\\
        S_n &= \{\text{a finite set of $n$-degree poly in $I$ whose leading coef generate } I_n \},
    \end{split}
\]
and observe the chain $I_0 \subset I_1 \subset \dots$ stabilizes as $S$ is Noetherian (say at index $m$). Then it is easy to see
\[
    \bigcup_{i < m} S_i
\]
generates $I$.

\subsection{Finite Generation of Invariants}
Hilbert also proved the following: if $G$ is finite group acting on a vector space $V/k$, then the corresponding invariant ring $P$ is finitely generated as a $k$-algebra.

The key idea is to introduce the Reynold's operator:
\[
    p(f) = \frac{1}{|G|} \sum_{g \in G} g(f).
\]
Observe
\begin{itemize}
    \item $p(f)$ is the average of $f$ under $G$. Furthermore, $p(f) \in P$.
    \item For $1/|G|$ to make sense, we assume $\text{char}(k)$ does not divide $|G|$.
    \item If $f \in P$, then $f = p(f)$.
\end{itemize}
To prove $P$ is finitely generated, we use Hilbert's basis theorem to find elements $a_1, \dots a_k$ that generate the ideal of polynomials in $P$ with constant term $ = 0$ (generate as an $R$-module, not $k$-algebra). Then for any $f \in P$, we note
\[
    f = p(f) = p \circ p(f) = \dots
\]
At some point, we find $f = p^m(f)$ is a polynomial of the $a_i$'s (over $k$). Thus we conclude $P$ is generated by the $a_i$'s as a $k$-algebra.

Hilbert's proof can be generalized to compact groups (define Reynold's operator with an integral). Using Weyl's unitarian trick of Lie groups, it can be generalized further.

%%% Local Variables:
%%% TeX-master: "main"
%%% End:
