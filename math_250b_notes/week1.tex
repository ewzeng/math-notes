\section{Week 1}
Commutative algebra is a surface subject intended to be used in fields like algebraic geometry, number theory, and (the somewhat out of fashion) invariant theory. In the old days, commutative algebra was the study of the ring of polynomials over a field. Over time, commutative algebra evolved into the study of commutative rings and related structures.

\subsection{Defintions}
The reader is assumed to familiar with basic algebraic concepts, like groups, rings, ideals, modules, homomorphisms, UFDs, etc.

In these notes, rings are always commutative and have identity. The reason why people study rings without identity is because these structures often arise from analysis. In fact, rings with identity are analogous to compact spaces, and rings without identity are analogous to locally compact spaces.

\subsection{Invariant Rings}
Suppose group $G$ acts on space $X$. Then $G$ acts naturally on functions $f: X \to Y$ by defining
\[
    (gf)(x) = f(g^{-1}x).
\]
If $f = gf$, we say $f$ is an invariant.

Consider $G = \bZ/3\bZ$ acting on $\bC^2$ by $60^{\circ}$ rotation in each component. After doing some work, we can conclude that the ring of invariant polynomials over $\bC^2$ is the $\bC$-algebra generated by $z_0 = x^3, z_1 = x^2y, z_2 = xy^2, z_3 = y^3$.
\begin{itemize}
    \item Note we have the following relations (called 1st order syzygies):
    \[
        \begin{split}
            a_1 &= z_2^2 - z_1z_3 = 0\\
            a_2 &= z_0z_3 - z_1z_2 = 0\\
            a_3 &= z_1^2 - z_0z_2 = 0\\
        \end{split}
    \]
    These three equations entirely describe the relations between $z_0, z_1, z_2, z_3$.
    \item Treating $z_0, z_1, z_2, z_3$ as symbols, we also have the following relation between $a_1, a_2, a_3$ (the 2nd order syzygy):
    \[
        b = z_1a_1 + z_2a_2 + z_3a_3 = 0.
    \]
\end{itemize}

Pictorally, we can describe our previous analysis of syzygies with the following exact sequence of $R$-modules:
\[
    0 \to R \to R^3 \to R \to P \to 0
\]
where:
\begin{itemize}
    \item $R$ is the ring of polynomials $\bC[z_0, z_1, z_2, z_3]$ and $P$ is the ring of invariant polynomials over $\bC^2$ (the invariant ring).
    \item $R \to P$ is the natural map quotienting out $a_1, a_2, a_3$.
    \item $R^3 \to R$ is the map sending $(1,0,0) \mapsto a_1$, $(0,1,0) \mapsto a_2$, $(0,0,1) \mapsto a_3$.
    \item $R \to R^3$ is the map sending $1 \mapsto (z_1, z_2, z_3)$.
\end{itemize}

\subsection{Classical Invariant Theory}
In the 19th century, a lot of effort was made to explicitly find generators of invariant rings for certain group actions. (Some generators would be polynomials spanning multiple pages. Being able to find ugly generators was a form of street cred among mathematicians in that era.)

In these complicated scenarios, the exact sequences obtained would look like
\[
    \dots \to R^* \to R^n \to R^m \to R \to P \to 0.
\]
Questions that were often asked were:
\begin{enumerate}
    \item Is $R$ finitely generated as a $\bC$-algebra? That is, does there exist a finite set of generators for $R$, and thus $P$? (Mathematicians of the 19th century would prove this by explicitly constructing generators, as described above.)
    \item Is $R^m$ finitely generated as a $R$-module? That is, is there is finite set of 1st order syzygies? 2nd order syzygies?
    \item Is the chain of modules finite?
\end{enumerate}
Hilbert killed off the whole theory by answering these questions in a general way (without explicit construction).