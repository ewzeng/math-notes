\section{Another View of the Spectrum}

Eisenbud develops the spectrum of a ring from a slightly different perspective that well complements Borcherd's lectures, so we also present it here.

Let $R = k[x_1, \dots, x_n]$ be the ring of polynomials defined on $k^n$. We make the following definitions:
\begin{itemize}
    \item For ideal $I \subset R$ define $Z(I) \subset k^n$ as the set of common zeros of polynomials in $I$. These sets are called \textbf{(affine) algebraic subsets}.
    \item Call an algebraic subset an \textbf{algebraic variety} if it is not the union of smaller algebraic subsets. (In other words, an irreducible algebraic subset.)
    \item For any $X \subset k^n$, define $I(X)$ as the set of polynomials vanishing on $X$.
\end{itemize}
Call an ideal $I$ radical if $I = \sqrt{I}$. Hilbert's Nullstellensatz states that for algebraically closed $k$, there is an (order-reserving) correspondence between radical ideals and algebraic subsets given by the maps
\[
    I \to Z(I), \quad X \to I(X).
\]
On further analysis, one can show:
\begin{itemize}
    \item Maximal ideals $\leftrightarrow$ points.
    \item Prime ideals $\leftrightarrow$ algebraic varieties.
\end{itemize}
Thus, $\Spec_m(R) = k^n$ and $\Spec(R)$ is the space of algebraic varieties in $k^n$. From another perspective, $\Spec(R)$ is an enhancement of $\Spec_m(R)$, introducing an extra element for every algebraic variety $V$ (that is not a point) to keep ``track'' of that variety. One thinks of this extra element as the generic point for $V$; the closure of this point consists of $V$ and all extra points corresponding to subvarieties of $V$.

Personally, I prefer this concrete example over the handwavy ``$R$ can roughly be seen as the space of continuous functions over $\Spec(R)$.''

As another cool fact, the quotient space
\[
    A(X) = \frac{k[x_1, \dots, x_n]}{I(X)}
\]
can be indentified as the space of polynomial functions defined on $X \subset k^n$ (as our quotient is an equivalence relation between polynomials taking the same values on $X$). $A$ is often called the \textbf{coordinate ring} over $X$.

\paragraph{Aside: Projective Varieties.} We can tweak our above example and only consider ideals $I$ generated by \textit{homogeneous} polynomials. Then our algebraic subsets are unions of 1-dim subspaces and therefore can be naturally embedded into the projective space $\bP^n_k$. Consequently, we have projective algebraic subsets and projective varieties. Working in the projective space often provides advantages (e.g. compactness) and is used alot in algebraic geometry.


%%% Local Variables:
%%% TeX-master: "main"
%%% End: