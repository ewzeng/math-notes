\section{Week 7}

\subsection{Nullstellensatz}

We now provide a proof of Hilbert's Nullstellensatz introduced earlier. The proof consists of two main steps, each with several substeps.
\begin{enumerate}
    \item We show that for algebraically closed $k$, the maximal ideals of $k[x_1, \dots, x_n]$ are precisely the ideals $(x_1 - a_1, x_2 - a_2, \dots)$. This is called the ``weak'' Nullstellensatz.
    \begin{itemize}
        \item[(a)] For any maximal $\mathfrak m$, we can consider the field extension
        \[
            \frac{k[x_1, \dots, x_n]}{\mathfrak m}
        \]
        This is a f.g. algebra, and thus by Zariski's lemma [proved below], it is a finite dimensional extension over $k$. Hence it is an algebraic extension of $k$ (recall transcendental extensions are infinite dimensional). But as $k$ is algebraically closed, this implies
        \[
            k = \frac{k[x_1, \dots, x_n]}{\mathfrak m}
        \]
        and from this we deduce the form of $\mathfrak m$.
    \end{itemize}
    \item Use step 1 to prove the ``strong'' Nullstellensatz. That is, for ideal $I \subset k[x_1, \dots, x_n]$, we want to show if $f = Z(I)$ then $f^m \in I$ (the other direction is easy). To do this, we do something called Rabinowitsch's trick:
    \begin{itemize}
        \item[(a)] We introduce a dummy variable $x_0$ and consider the ideal $(I, 1 - x_0f)$ in $k[x_0, \dots, x_k]$. As it has no common roots, it is not contained in any maximal ideal by step 1, so it is the whole ring $k[x_0, \dots, x_k]$ and we may write
        \[
            1 = a_1b_1 + \dots a_nb_n + a(1 - x_0f), \quad a_i \in k[x_0, \dots, x_n], b_* \in I.
        \]
        \item[(b)] We now quotient the ring $k[x_0, \dots, x_n]$ by $(1 - x_0f)$ (note we are basically localizing $k[x_1, \dots, x_n]$ at $f$ and writing $x_0 = \frac{1}{f}$). Then we have
        \[
            1 = a_1b_1 + \dots a_nb_n, \quad a_i \in k[x_1, \dots, x_n, \frac{1}{f}], b_* \in I.
        \]
        After clearing denominators, we get
        \[
            f^m = c_1b_1 + \dots c_nb_n \in I.
        \]
        [Caution: in this proof, be careful of $k[x_1, \dots, x_n]$ vs. $k[x_0, \dots, x_n]$.]
    \end{itemize}
\end{enumerate}

Zariski's lemma states that if the field extension $K/k$ is a f.g. algebra over $k$, then $K/k$ is a finite dimensional extension. The proof of Zariski's lemma is a little tricky, so we first present a simple case:
\begin{itemize}
    \item Let $K = k(x_1, \dots, x_m)$, the space of rational functions, which is an infinite dimensional extension. Suppose the generators of $K$ as a $k$-algebra are
    \[
        \left\{\frac{f_i(x_1,\dots x_m)}{g_i(x_1, \dots, x_m)} \Bigm\vert i \in I \right\}.
    \]
    Note that every irreducible polynomial must be a factor of one the $g_i$'s. However as $k(x_1, \dots, x_m)$ has infinite number of irreducible polynomials (copy Euclid's proof for infinite number of primes), we conclude $I$ is an infinite index set.
\end{itemize}
For general $K$, we write
\[
    K = k(\underbrace{x_1, \dots, x_m}_{\text{alg. indep.}}, \underbrace{x_{m+1}, \dots, x_n}_{\text{alg. over } k[x_1, \dots, x_m]} )
\]
and try to copy the proof above, but do technical stuff to deal with $x_{m+1}, \dots, x_n$.

\subsection{Integral Elements}

We now go on a slight tangent to discuss ring extensions. Let $R$ be a ring and $S$ a $R$-algebra. Call an element $s \in S$ integral if it is a root of a monic polynomial with coefficents in $R$. As expected from the theory of field extensions, if $s$ is integral, then $R[s]$ is f.g. over $R$.

The Cayley-Hamilton theorem for modules states any endomorphism (i.e. square matrix) of f.g. $R$-module $M$ satisfies its own characteristic equation, and as a consequence, every endomorphism of $M$ is integral.

\subsection{Normalization}

For an integral domain $R$, let $K$ be its field of fractions. We call the normalization of $R$ to be the integral closure of $R$ in $K$ (i.e. find all integral elements over $R$). If $R$ is equal to its integral closure, we say $R$ is normal.

Geometrically, normalization can be interpreted as singularity removal in the spectrum. For instance, the ring
\[
    k[t^2, t^3] \cong \frac{k[x,y]}{y^2-x^3}
\]
represents the polynomials defined on the curve $y^2 = x^3$, which has a singularity at the origin. The normalization of $k[t^2, t^3]$ is $k[t]$, whose spectrum is a straight line (no singularities).

In fact, Serre proved that under mild assumptions (that are usually met), normal $\iff$ singularities of spectrum have codim $\ge 2$. A common trick in algebraic geometry is to normalize (remove singularities of codim 1), blowup (remove other singularities but may create singularities of codim 1), and repeat.

%%% Local Variables:
%%% TeX-master: "main"
%%% End: