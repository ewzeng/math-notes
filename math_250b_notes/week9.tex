\section{Week 9 (Homological Algebra)}

\subsection{Definition and Motivation of Tor}
The motivation behind $\Tor$ comes from homology in algebraic topology. For a triangulated manifold $M$, if we let $C_q$ be the group generated by the oriented $q$-simplicies, recall we can form the chain complex
\[
    \dots \xrightarrow{\partial} C_q \xrightarrow{\partial} C_{q-1} \xrightarrow{\partial} \dots \xrightarrow{\partial} C_0 \xrightarrow{\partial} 0.
\]
(In a chain complex, the image of a map is a subset of the kernel of the next map, so it is \textit{not necessarily exact}.) Furthermore, recall the $q$-th homology group is defined as
\[
    H_q(M, \bZ) = \frac{\ker(\partial: C_q \to C_{q-1})}{\Ima(\partial: C_{q+1} \to C_q)}.
\]
More generally, recall for any abelian $G$ we can define
\[
    H_q(M, G) = \frac{\ker(\partial: C_q \otimes G \to C_{q-1} \otimes G)}{\Ima(\partial: C_{q+1} \otimes G \to C_q \otimes G)}.
\]
For instance, $G = \bQ$ gives the rational homology.

$\Tor$ is a function that takes abelian $\times$ abelian $\to$ abelian. $\Tor(A, B)$ is defined as taking a finite free resolution of $A$,
\[
    0 \to Z^m \to Z^n \to A \to 0,
\]
turning it into a \textit{chain complex} by dropping the $A$,
\[
    0 \to Z^m \to Z^n \to 0,
\]
and taking the homology group
\[
    \frac{\ker(Z^m \otimes B \to Z^n \otimes B)}{\Ima(0 \otimes B \to Z^m \otimes B)} = \ker(B^m \to B^n).
\]
Unsurprisingly, $\Tor$ shows up when computing homologies (e.g. the universal coefficent theorem). The reason $\Tor$ is independent of choice of resolution is the same reason why homology groups are independent of the triangulation of the underlying manifold. Although the details are rather tedious, the argument is (roughly) as follows:
\begin{enumerate}
    \item We show chain maps between two resolutions (of the same group) are ``induced by homotopies on the underlying manifold'' and thus give rise to the same (canonical) map between the homologies, and thus $\Tor$.
    \item We show the canonical maps going in opposite directions are inverses.
\end{enumerate}

Another way to view $\Tor(A,B)$ is the group that makes
\[
    0 \to \Tor(A,B) \to Z^m \otimes B \to Z^n \otimes B \to A \otimes B \to 0
\]
exact. This suggests that $\Tor$ is a measure of how much $\cdot \otimes B$ preserves exactness, or equivalently, how far $B$ is from being flat.

As flat is torsion-free, this suggests that $\Tor(\cdot, B)$ only depends on the torsion subgroup of $B$ (i.e. the possibly non-flat part), and thus gives insight to why $\Tor$ is named $\Tor$.

\subsection{Properties of Tor}
Here are some properties of $\Tor$:
\begin{itemize}
    \item $\Tor(A\oplus B, C) = \Tor(A, C) \oplus \Tor(B,C)$. (Easy exercise.)
    \item $\Tor(A,B) = \Tor(B,A)$. (Make a grid and do a zigzag diagram chase.)
    \item $\Tor(\cdot, B)$ is a functor.
    \item If $0 \to A \to B \to C \to 0$ is exact, by making a grid and then applying the Snake Lemma (whose proof is also a zigzag diagram chase), we get the long exact sequence
    \[
        0 \to \Tor(A,G) \to \Tor(B,G) \to \Tor(C,G) \to A \otimes G \to B \otimes G \to C \otimes G \to 0.
    \]
    This further validates the view that $\Tor$ is a measure of the failure of $\cdot \otimes G$ to be exact.
\end{itemize}

Additionally, $\Tor$ is nice because it is often easy to calculate. [Random: the best introduction to homological algebra is still Cartan and Eilenberg's original book introducing it.]

\subsection{Tor Over Rings}
We can naturally extend $\Tor$ to module $\times$ module $\to$ abelian. For a $R$-modules $M$, $N$, take a resolution
\[
    \dots \to R^{n_1} \to R^{n_0} \to M \to 0
\]
form a chain complex (like before), and define $\Tor_i^R(M,N)$ as the $i$-th homology group, i.e.
\[
    \Tor_i^R(M,N) \to \frac{\ker(R^{n_i} \otimes N \to R^{n_{i-1}} \otimes N)}{\Ima(R^{n_{i+1}} \otimes N \to R^{n_i} \otimes N)}.
\]
(Note that our previous definition of $\Tor$ becomes $\Tor_1^\bZ$.) The proof that $\Tor$ is independent of choice of resolution is exactly the same as before. Additionally, the properties of $\Tor$ previously listed are also kept (with the same proofs), except the long exact sequence is now
\[
    \begin{split}
        \dots & \to \Tor^R_2(A,N) \to \Tor^R_2(B,N) \to \Tor^R_2(C,N) \to \Tor^R_1(A,N) \\
              & \to \Tor^R_1(B,N) \to \Tor^R_1(C,N) \to A \otimes N \to B \otimes N \to C \otimes N \to 0.
    \end{split}
\]

\subsection{Derived Functors}
The construction of $\Tor$ is a specific example of the construction of a derived functor (which ``measures the failure'' of a functor to be exact).

Given any right-exact functor $F$, the left-derived functors $L_i$ are formed by taking a projective resolution, turning it into a chain map, applying $F$, and taking the $i$-th homology group. In the case of $\Tor$, $F$ is the tensor product. On the other hand, given any left-exact functor $F$, we can form the right-derived functors $R^i$ by using an injective resolution instead of a projective one, but keeping everything else the same. In both cases, the choice of resolution does not matter, and we get a long exact sequence property (like before).

\subsection{Ext}
As $\Hom_R(X, \cdot)$ is a left-exact functor on the category of $R$-modules, we may consider its right-derived functors $\Ext^i(X, \cdot)$.

Given modules $X, A$, an extension of $X$ by $A$ is a module $B$ such that
\[
    0 \to A \to B \to X \to 0.
\]
Call two extensions $B, B'$ equivalent if there exists an isomorphism between them that plays well with the other maps in the exact sequences. The reason why $\Ext$ is called $\Ext$ is because $\Ext^1(X,A)$ classifies all extensions of $X$ by $A$ (up to equivalence). To see this, we take the long exact sequence
\[
    0 \to \Hom(X, A) \to \Hom(X, B) \to \Hom(X, X) \to \Ext^1(X, A) \to \dots
\]
It happens that different extensions will have different images of $1 \in \Hom(X,X)$ in $\Ext^1(X,A)$. With a little more argument, one can show that each element of $\Ext^1(X,A)$ corresponds uniquely to one extension, thus proving our claim. (Cool fact: the split extension $B = X \oplus A$ corresponds to $0 \in \Ext^1(X,A)$.) 

\subsection{Injective Resolutions}
Recall we define module $I$ to be injective if for any $A \subset B$, any map $A \to I$ can be extended to a map $B \to I$.
\[
    \begin{tikzcd}
    0 \arrow[r] & A \arrow[r] \arrow[d] & B \arrow[ld, dashed] \\
                & I                     &                     
    \end{tikzcd}.
\]
To construct right-derived functors, we need to be able to construct an injective resolution of any module $A$
\[
    0 \to A \to I_0 \to I_1 \to \dots.
\]
It is not too hard to show that for injective resolutions to exist, it suffices to show every module can be embedded as a submodule of an injective module.

To do this, we first define an extension $A \subset B$ to be essential if every nonzero submodule of $B$ has nonzero intersection with $A$. Then we observe that if every essential extension of a module $A$ is trivial, then $A$ is injective. From this, it is not hard to see that the maximal essential extension of $A$ is injective, proving the desired claim.

[Remark: the constructed injective module is called the injective envelope, and it is a ``minimal'' injective module that $A$ can be embedded in.]

%%% Local Variables:
%%% TeX-master: "main"
%%% End: