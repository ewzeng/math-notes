\section{Week 13}
We have the following classification of Noetherian local rings:
\[
    \begin{split}
        \text{Regular local} & \subset \text{Local complete intersection} \\
                             & \subset \text{Gorenstein}                  \\
                             & \subset \text{Cohen-Macaulay}              \\
                             & \subset \text{Noetherian local}.
    \end{split}
\]
At a high level, these terms mean the following:
\begin{itemize}
    \item Regular local: spectrum nonsingular.
    \item Local complete intersection: defined by the minimum number of relations.
    \item Gorenstein: nice duality properties.
    \item Cohen-Macaulay: no ``mixing'' of dimensions in the spectrum (i.e. the spectrum can't be the intersection point of a plane and a line).
\end{itemize}

\subsection{Regular Local Rings}
We say a Noetherian local ring $(R, \mathfrak m)$ is regular if the dimension of $\mathfrak m/\mathfrak m^2$ (as a $R/\mathfrak m$-v.s.) is equal to the dimension of $R$. That is, the dimension of the cotangent space (= dimension of the tangent space) at $\mathfrak m$ is equal to the dimension of $\Spec(R)$. Intuitively, this means the spectrum is nonsingular at $\mathfrak m$.

In general, we call a Noetherian ring regular if the localization at every prime ideal is regular.

\subsection{Cohen-Macaulay Rings}
Given a Noetherian local ring $(R, \mathfrak m)$, recall if $x_1$ is not a zero divisor or a unit, then
\[
    \dim\left( \frac{R}{(x_1)} \right) = \dim(R) - 1.
\]
We can repeat this construction by choosing $x_i$ that is not a unit or a zero divisor in $R/(x_1,\dots,x_{i-1})$, and so on, until we decide to stop, get to dimension 0, or all remaining choices are units/zero divisors.

Call the resulting sequence $x_1, \dots, x_n$ a \textbf{regular sequence}. Note $n \le \dim(R)$. Define the depth of $R$ to be the length of the longest regular sequence, and call a ring Cohen-Macaulay if the depth is equal to the dimension. (Using homological algebra (specifically, $\Ext$), one can show that any maximal regular sequence has length equal to the depth.)

The spectrum of a C-M ring is equidimensional (a consequence of the unmixedness theorem), but the converse is not completely true. However, an equidimensional spectrum is still the best intuitive description of C-M rings.

In general, we call a Noetherian ring Cohen-Macaulay if the localization at every prime ideal is C-M.

\subsection{Koszul Complex}
Given a sequence $x_1, \dots, x_n \in R$, we define the Koszul complex $K(x_1, \dots, x_n)$ to be the chain complex
\[
    0 \to \bigwedge^n R^n \xrightarrow{d_n} \bigwedge^{n-1} R^n \to \cdots \xrightarrow{d_3}  \bigwedge^{2} R^n \xrightarrow{d_2} R^n \to R \to \frac{R}{(x_1, \dots, x_n)} \to 0
\]
where
\[
    d_k(e_1 \wedge \cdots \wedge e_k) = \sum_{i = 1}^k (-1)^{i+1}x_ie_1\wedge \cdots \wedge \hat{e_i} \wedge \cdots \wedge e_k.
\]
[Here, $\hat{\cdot}$ denotes omission.] Note the Koszul complex looks very ``homological.''

When $x_1, \dots, x_n$ is a regular sequence, one can show that $K(x_1, \dots, x_n)$ is exact. (In other words, we can find a finite free resolution of $R/(x_1, \dots, x_n)$.) Moreover, one can show that when $R$ is local, exactness of the Koszul complex implies regularity. From this, one can show that when $R$ is local, the permutation of any regular sequence is also regular.

\subsection{Gorenstein Rings}
For Noetherian local $(R, \mathfrak m)$, we are interested in a contravariant functor $D: \text{Module}_R \to \text{Module}_R$ that is exact and has $D^2 = 1$. The naive choice $D = \Hom_R(\cdot, R)$ unfortunately isn't always nice; instead we have $D = \Hom_R(\cdot, \o_R)$, where $\o_R$ is called the canonical bundle.

We call $R$ Gorenstein if $R = \o_R$. Denoting $k = R/\mathfrak m$, we have two other equivalent characterizations:
\begin{enumerate}
    \item $R$ is Gorenstein if $\Ext^i(k,R) = 0$ for $i \neq d$, and $\Ext^d(k,R) \cong k$.
    \item $R$ is Gorenstein if $R/(x)$ is Gorenstein for $x$ not a unit/zero divisor. Base case: when $\dim(R) = 0$, we say $R$ is Gorenstein if $\Hom_R(k, R) \cong k$. (Note: to be able to reach the base case, need $R$ to be Cohen-Macaulay.)
\end{enumerate}
Being a Gorenstein ring is a subtle property: a Gorenstein and a non-Gorenstein ring can be very similar.

\subsection{Fitting Ideals}
Let $M$ be a f.g. $R$-module generated by $g_1, \dots, g_n$. Furthermore, suppose the relations between the generators are given by
\[
    \begin{split}
        r_{11}g_1 + r_{12}g_2 + \dots + r_{1n}g_n & = 0             \\
        r_{21}g_1 + r_{22}g_2 + \dots + r_{2n}g_n & = 0             \\
                                                  & \vdots
    \end{split}
\]
We can then consider the (possibly infinite) matrix
\[
    A = \begin{pmatrix}
        r_{11}                                    & \dots  & r_{1n} \\
        r_{21}                                    & \dots  & r_{2n} \\
        \vdots                                    & \vdots & \vdots
    \end{pmatrix}
\]
and define
\[
    \Fitt_k(M) = \text{ideal generated by dets of all $(n - k) \times (n - k)$ minors.}
\]
[Note we if take a presentation $R^* \to R \to M \to 0$, then $A^T$ represents the $R$-linear map from $R^* \to R$.] One can show that $\Fitt_k$ does not depend on the presentation of $M$ by showing that $\Fitt_k$ is invariant under adding/removing a generator + relation, and adding/removing a relation that is a linear combination of known relations.

$\Fitt_k(M)$ measures the obstruction of $M$ being generated by $k$ elements. More specifically, $R/\Fitt_k(M)$ can be interpreted as the ``minimum size'' of what is left over after killing $k$ elements. One can check that if $M$ is generated by $k$ elements, then $R/\Fitt_k(M) = 0$.

\subsection{Local Complete Intersection Rings}
Define a ring to be L.C.I. if it can be written as
\[
    \frac{R}{(x_1, \dots, x_m)}
\]
where $R$ is regular local and $x_1, \dots, x_m$ is a regular sequence.

Geometrically, if $S$ is L.C.I., then the (local) codimension of $\Spec(R)$ as a subset of the affine space $\bA^n$ is equal to the number of equations needed to define it.

We have the following criteria to determine if a ring $(S, \mathfrak m)$ is L.C.I.:
\begin{itemize}
    \item If $\dim(S) = 0$, then $S$ is L.C.I. $\iff$ $\Fitt_0(\mathfrak m) \neq 0$.
    \item If $\dim(S) > 0$, then $S$ is L.C.I. if $S/(x)$ is L.C.I. for $x$ not a unit/zero divisor.
\end{itemize}

%%% Local Variables:
%%% TeX-master: "main"
%%% End: