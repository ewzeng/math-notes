\section{Week 6}

\subsection{Finite Length Revisited}

If $N$ is a simple module over $R$, then $N$ is generated by a nonzero element. Hence there is a natural map $R \to N$ from which we conclude that $N \cong R/\mathfrak m$ for some maximal ideal $\mathfrak m$.

Consequently (and trivially), if $M$ is of finite length over $R$, then there exists a chain
\[
    0 \subset M_1 \subset M_2 \subset \dots \subset M_n = M
\]
where each $M_i/M_{i-1} \cong R/\mathfrak m_i$ for some maximal ideals $\mathfrak m_i$.

If $M$ is not of finite length, but it is finitely generated over a Noetherian $R$, it happens we can say something similar! For such $M$, we can show it contains a submodule isomorphic to $R/\mathfrak p$ for some prime $\mathfrak p$. (Proof: pick maximal ideal $I$ such that $M$ contains a submodule isomorphic to $R/I$). Applying this fact and taking quotients repeatedly, we can get a chain
\[
    0 \subset M_1 \subset M_2 \subset \dots \subset M_n = M
\]
where each $M_i/M_{i-1} \cong R/\mathfrak p_i$ for some prime ideals $\mathfrak p_i$. As $R$ is Noetherian (and thus so is $M$), this chain has to stop.

However, this generalization lacks the following useful property: for two such chains, the multiplicities of the quotients $R/\mathfrak p_i$ are not necessarily the same, and one quotient may even be missing in the other! Consequently, we no longer have the additive exact sequence property of multiplicities. We try to handle this by introducing associated primes.

\subsection{Associated Primes}
For f.g. $M$ over Noetherian $R$, define
\[
    \begin{split}
        \Ass(M) & = \left\{\mathfrak p \in \Spec(R) \mid \frac{R}{\mathfrak p} \cong \text{ submodule of } M\right\} \\
                & = \{\mathfrak p \in \Spec(R) \mid \mathfrak p = \Ann(a), \ a \in M\}
    \end{split}
\]

For an exact sequence $0 \to A \to B \to C \to 0$ of $R$-modules, it is not hard to show
\[
    \Ass(A) \subset \Ass(B) \subset \Ass(B) \cup \Ass(C).
\]
Consequently, it is not hard to conclude if $\mathfrak p \in \Ass(M)$, then $R/\mathfrak p$ will definitely occur as a quotient in a chain decomp of $M$.

We can also define the support of a module $M$ to be
\[
    \begin{split}
        \Supp(M) & = \{\mathfrak p \in \Spec(R) \mid M_{\mathfrak p} = 0\} \\
                 & = \{\mathfrak p \mid \mathfrak p \supset \Ann(M)\}
    \end{split}
\]
One can show that $\Supp(M) = \overline{\Ass(M)}$ (note both have something to do with annilators).

\subsection{Coprimary Decomposition}
Associated primes allow us to state a result similar to the structure theorem of abelian groups: every f.g. module $M$ over Noetherian $R$ is a submodule of
\[
    \bigoplus M_i
\]
where each $M_i$ has at most 1 associated prime. It happens that having at most 1 associated prime is equivalent to the notion of being coprimary, and with some work, one can show the above result is equivalent to Noether's theorem on primary decomposition.

The proof of coprimary decomposition is as follows:
\begin{enumerate}
    \item Assume not. WLOG assume if $N \subset M$ then $M/N$ is a submodule of a product of coprimary modules (else we first quotient out a maximal submodule that is not).
    \item As $M$ is not coprimary, there exists submodules $M_1, M_2$ such that
    \[
        M_1 \cong \frac{M}{\mathfrak p}, \quad M_2 \cong \frac{M}{\mathfrak q}, \quad \mathfrak p, \mathfrak q \text{ distinct primes.}
    \]
    \item If $x \in M_1 \cap M_2$ and $x \neq 0$, then $\Ann(x) = \mathfrak p$ and $\Ann(x) = \mathfrak q$. Thus $x = 0$. Then the natural exact sequence
    \[
        0 = M_1 \cap M_2 \to M \to \frac{M}{M_1} \oplus \frac{M}{M_2} \subset \text{ product of coprimary modules}
    \]
    produces a contradiction.
\end{enumerate}

%%% Local Variables:
%%% TeX-master: "main"
%%% End: