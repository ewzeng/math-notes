\section{Week 12}

We define the dimension of a ring to be the dimension of its spectrum.

\subsection{Dimension of Topological Spaces}
The dimension of topological spaces is a quantity that is intuitive to grasp but difficult to rigorously define (e.g. how do you deal with the space-filling curve?). Here we present a few definitions that people have come up with over the years.

\paragraph{Brouwer-Urysohn-Menger.} B-U-M is the classical definition of dimension. We say a topological space has dimension $n$ if all points have arbitrarily small neighborhoods with boundaries of dimension $< n$. (Base case: empty set has dimension $-1$).

\paragraph{Krull Dimension.} A topological space has a Krull dimension $n$ if the longest chain of irreducible subsets
\[
    Z_0 \subsetneq Z_1 \subsetneq \dots \subsetneq Z_n
\]
has length $n+1$.

This definition works well with Noetherian spaces, but works poorly for Hausdorff spaces. However, for Noetherian spaces, this definition happens to coincide with B-U-M. Nevertheless, because of historical reasons, the Krull definition is used more often than B-U-M.

\paragraph{Hausdorff Dimension.} To define the H. dimension of a \text{metric} space, we cover the space with $\e$-balls and study the asymptotic behavior of the number of balls needed as $\e \to 0$. The H. dimension is not necessarily an integer.

\paragraph{Hilbert Polynomials.} Suppose the longest strictly increasing chain of submodules of module $M$ is
\[
    0 = M_0 \subsetneq M_1 \subsetneq \dots \subsetneq M_n = M.
\]
Then we say that $M$ has length $n$.

Now we define the Hilbert polynomial associated with the local ring $(R, \mathfrak m)$ to be $p(x) \in \bQ[x]$ such that
\[
    p(k) = \text{length}\left( \frac{\mathfrak m^{k-1}}{\mathfrak m^k} \right)
\]
for large enough $k \in \bZ^+$. We then define the dimension of $\Spec(R)$ to be $1 + \deg(p)$. (In other words, take the associated graded ring, and see how fast the number of generators in each grade increases.)

The advantage of Hilbert polynomials is that it allows us to define dimension locally around any point in the spectrum by first localizing at that point. (The dimension of the entire spectrum would then be the $\sup$ of the local dimensions.)

\paragraph{Tangent Spaces.} We can also define the dimension at a point to be the dimension of the corresponding tangent space (as in the case of manifolds). For a local ring $R$ with maximal ideal $\mathfrak m$, we can compute the tangent space at the point $x$ in the spectrum corresponding to $\mathfrak m$ by computing the cotangent space (the dual) at $x$. The key is that the cotangent space can also be defined as the space of functions vanishing at $x$, identifying two functions to be the same if they have the same first-order behavior at $x$. From this, one can show that the cotangent space is $\mathfrak m/\mathfrak m^2$ over the field $R/\mathfrak m$. One can then show that the dimension of the cotangent space (and thus the tangent space) is the number of generators of $\mathfrak m$.

However, this definition breaks down when there are singularities in $\Spec(R)$.

\paragraph{System of Parameters.} A variation of above definition is to consider the minimal number of elements in a \textit{system of parameters}, which is a set of elements whose generated ideal contains $\mathfrak m^n$ for some $n \ge 1$.

\paragraph{Homological Definitions.} Homological definitions define dimension based on when various homology groups vanish.

In the next few sections, we will study some of these definitions more in-depth.

\subsection{Hilbert Polynomials}
The Hilbert polynomial defined in the previous section is an example of a more general construction. Let $R = \oplus R_n$ be a f.g. graded $R_0$-algebra (where $R_0$ is Noetherian) and let $M = \oplus M_n$ be a f.g. graded $R$-module. Hilbert polynomials are chiefly concerned with the growth of $M_n$ as $n \to \i$.

Define the Poincar\'{e} series of $M$ to be
\[
    P_M(t) = \sum_n \text{length}(M_n)t^n.
\]
If $x_1, \dots, x_s$ are the generators of $R$ with degrees $k_1, \dots, k_s$ respectively, we claim $P_M(t)$ can be written as
\[
    P_M(t) = \frac{f(t)}{(1 - t^{k_1}) \cdots (1-t^{k_s})}, \quad f(t) \in \bZ[t].
\]
The proof is to consider the kernel and cokernels of the map $\times x_s: M_n \to M_{n + k_s}$:
\[
    0 \to K_n \to M_n \to M_{n + k_s} \to L_{n + k_s} \to 0.
\]
As length is additive, we get
\[
    \text{length}(K_n) - \text{length}(M_n) + \text{length}(M_{n + k_s}) - \text{length}(L_{n + k_s}) = 0
\]
which implies
\[
    P_M(t) = \frac{P_L(t) - t^{k_s}P_K(t)}{1 - t^{k_s}}, \quad K = \oplus K_n, \quad L = \oplus L_n.
\]
As $x_s$ acts trivially on $K$ and $L$, we can view $K$ and $L$ as modules over the graded $R_0$-algebra generated by $x_1, \dots, x_{s-1}$ (and still have the same Poincar\'{e} series). Iteratively applying this process, we get the desired claim.

In the common case that all the $k_i = 1$, we get
\[
    P_M(t) = \frac{f(t)}{(1 - t)^s} = f(t)\left( 1 + st + \dots + \binom{n+s-1}{s-1}t^n + \dots  \right).
\]
From this, it is not hard to see that for large enough $n$, the $n$-th coefficent of $P_M(t)$ is given by a polynomial in $n$.

\subsection{Equivalence of Definitions}
We will now show that the Krull, B-U-M, Hilbert, and System-of-Parameter definitions of dimension are equivalent for Noetherian local rings. The general argument goes like
\[
    \text{Krull} \le \text{Hilbert} \le \text{Parameter} \le \text{Krull} = \text{B-U-M}.
\]
In our proof sketches, the Noetherian local assumption will be implicit.

\paragraph{Krull $=$ B-U-M.} The idea is that for a maximal increasing chain of irreducible subsets
\[
    0 = M_0 \subsetneq M_1 \subsetneq \dots \subsetneq M_n = M,
\]
going from $M_i \to M_{i - 1}$ is like taking the boundary. (More precisely, $M_{i-1}$ is in the boundary of an arbitrarily small neighborhood of any point in $M_i - M_{i-1}$.)

\paragraph{Krull $\le$ Hilbert.} First, we show that for Noetherian local ring $(R, \mathfrak m)$, we have
\[
    \text{Hilbert-dim}\left( \frac{R}{(x)} \right) < \text{Hilbert-dim}\left( R \right)
\]
when $x \in \mathfrak m$ is not a zero divisor. (Intuitively, this makes sense because quotienting makes the spectrum smaller.) We then show that quotienting by certain $x$ decreases the Krull dimension by \textit{at most} 1. Applying these two facts iteratively gives us the desired claim.

\paragraph{Hilbert $\le$ Parameter.} If the system of parameters generates the ideal $\mathfrak q$, we first show that $1 +$ the degree of the Hilbert polynomial defined by
\[
    p(k) = \text{length}\left( \frac{\mathfrak q^{k-1}}{\mathfrak q^{k}} \right)
\]
is equal to the Hilbert dimension. We then show the former quantity is $\le$ number of generators of $\mathfrak q$.

\paragraph{Parameter $\le$ Krull.} Given an irreducible subset $Z$, define the (Krull) codimension of $Z$ to be $n$ if the length of longest chain of irreducible subsets of the form
\[
    Z = Z_0 \subsetneq Z_1 \subsetneq \dots \subsetneq Z_n = \Spec(R)
\]
is $n + 1$. Then for an arbitrary set, define its codimension to be the minimum codimension of all the irreducible subsets it contains.

The idea is that if $\Spec(R)$ has Krull dimension $d$, we pick $x_1, \dots, x_d \in R$ such that the zero set of $(x_1, \dots, x_i)$ has codimension $\ge i$ for each $i$. Then we show that this implies $x_1, \dots, x_d$ form a system of parameters.

\subsection{Miscellaneous Observations}
\paragraph{Remark.} The equivalence of definitions allows us to switch between them conveniently. For instance:
\begin{itemize}
    \item Parameter defintion: easy to upper bound the dimension.
    \item Krull definition: easy to lower bound the dimension.
    \item Hilbert definition: has its own advantages (e.g. can trivially show that completions preserve dimension).
\end{itemize}

\paragraph{Krull's Principal Ideal Theorem.} PIT states that the codimension of (the zero set of) $x$ in a Noetherian ring $R$ is 1 if $x$ is not a zero divisor or a unit. The proof is to pick a minimal prime $\mathfrak p \supset (x)$, show the localization $R_{\mathfrak p}$ does not change the codimension of $(x)$ but has dimension 1, and then rule out the case when the codimension is 0.

As a consequence, quotienting a Noetherian local ring by an element that is not a zero divisor or a unit reduces the dimension by 1.

%%% Local Variables:
%%% TeX-master: "main"
%%% End: