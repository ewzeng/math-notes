\section{Week 2}

\subsection{Noetherian Modules}
Call a module Noetherian if every submodule is finitely generated. (Note: Noetherian rings are a special case.) Like Noetherian rings, there are two other equivalent definitions: ascending chains and maximal elements.

One particular feature of note is the exact sequence property: given an exact sequence of $R$-modules
\[
    0 \to A \to B \to C \to 0,
\]
then $B$ Noetherian $\iff$ $A$ and $C$ Noetherian. (Proof: straightforward.) From this, we conclude direct sums of Noetherian modules are Noetherian, and as quotients of Noetherian modules are Noetherian (like Noetherian rings), it follows that \textit{any finitely generated module over a Noetherian ring is Noetherian}.

One direct application of this result is to show that if invariant ring $P$ is a finitely generated algebra, then all syzygies are finitely generated. The 1st order comes from Hilbert's basis theorem (as we have seen). For higher order syzygies, we iteratively apply this result to show every module in the exact sequence is Noetherian, and thus the syzygies (which are submodules) are finitely generated.

\subsection{Noether's Generalization of Finite Generation of Invariants}
Recall that Hilbert's proof of finite generation of invariants required assumptions on $\text{char}(k)$ to define the Reynold's operator. Noether was able to prove a generalization without any assumption on $\text{char}(k)$. However, to do so, she had to do something different.

Borrowing notation from before, the proof is to choose a ring $S$ such that
\[
    S \subset P \subset R = k[x_1, \dots, x_n]
\]
where
\begin{enumerate}
    \item $S$ is a finitely generated $k$-algebra (and thus Noetherian by Hilbert's basis theorem).
    \item $R$ is a finitely generated $S$-module (and thus Noetherian).
\end{enumerate}
Then
\begin{itemize}
    \item 2 implies $P$ is finitely generated $S$-module.
    \item It follows that $P$ is a finitely generated $k$-algebra (consider the generators of $S$ as a $k$-algebra with the generators of $P$ as a $S$-module).
\end{itemize}
The hard part is constructing $S$, which is defined as $k$-algebra generated by the coefficents of the polynomials
\[
    f_i(x) = \prod_{g \in G} (x-gx_i), \quad 1 \leq i \leq n.
\]
From the equation $f_i(x_i) = 0$, we conclude $x_i^{|G|}$ is a linear combination (in $S$) of lower powers of $x_i$. From this, it is not hard to see that $R$ is a finitely generated $S$-module.

\subsection{Visualizing Rings}
Being able to visualize rings is a powerful technique. We go over 3 different methods.

\paragraph{Drawing a point for each element.} For instance, we can draw $\bZ[i]$ as the lattice points in $\bC$. To show $\bZ[i]$ is an Euclidean domain (i.e. has a division algorithm), it actually suffices to show that $\bC$ is covered by open disks of radius 1 centered at each lattice point, which we can easily verify. Thus, this visualization allowed us to turn an algebraic problem into a geometric one. (Similar arguments can be made with other rings like $\bZ[\sqrt{-2}]$).

Drawing a point for each element is very powerful when it works, but it doesn't work often. It is usually most applicable in algebraic number theory (or any ring whose additive group can be embedded in a vector space).

\paragraph{Drawing a point for each basis element.} This method works best when dealing with polynomials or power series over a field. In these cases, we draw a point for each monomial (e.g. we draw a grid for $k[x,y]$). (Note: monomials are not really the ``basis'' elements for power series, but they are close enough).

One application of this method is to prove the Weierstrass preparation theorem, which states the power series $k[[x,y]]$ behaves like the polynomial ring $(k[[x]])[y]$. As the latter is a UFD, this allows us to show that $k[[x,y]]$ is UFD as well.

\paragraph{Drawing a point for each prime ideal.} We first present some motivation. Let $X$ be a compact Hausdorff space and $R$ the ring of continuous functions on $X$. The key idea is that \underline{we can study $R$ by visualizing $X$}.

Given $R$, we can recover $X$ in the following manner:
\begin{itemize}
    \item $x \in X$ $\longleftrightarrow$ $I_x = \{f : f(x) = 0\}$. One can show that the $I_x$'s are precisely the maximal closed ideals of $R$.
    \item The topology on $X$ can be partially recovered in two equivalent ways. We can define a basis of open sets in $X$ to be the sets $U(f) = \{I_x : f \not \in I_x\}$. Alternatively, we can define the closed sets to be the sets $Z(I) = \{I_x : I \subset I_x\}$ where $I$ is any closed ideal.
\end{itemize}
We want to generalize this to an arbitrary ring $R$. That is, given arbitrary ring $R$, we define $\Spec_m(R)$ to be the set of maximal ideals in $R$ with a topology recovered in a similar way as above.

Given a map $f: R \to S$, we want to define a map $\Spec_m(S) \to \Spec_m(R)$ given by
\[
    f^{-1}: I \mapsto f^{-1}(I)
\]
However, we run into a problem as $f^{-1}(I)$ is not necessarily a maximal ideal. The solution is to consider prime ideals instead (which are pretty similar to maximal ideals). It is not hard to see that the preimage of prime ideals are prime. Thus instead of working with $\Spec_m(R)$, we work with $\Spec(R)$, the set of prime ideals with topology defined in a similar manner as before.

Studying $\Spec(R)$ is a powerful technique in commutative algebra. Visualizing $\Spec(R)$ is difficult (usually it has very weird topology), but we try our best to do so. For instance, $\Spec(\bZ)$ consists of the prime ideals and the ideal $(0)$, whose closure is the whole space (and thus drawn as a ``1-dimensional point'').


%%% Local Variables:
%%% TeX-master: "main"
%%% End: