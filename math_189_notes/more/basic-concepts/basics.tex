\section{Starting Principles}

The state of a quantum system $\cS$ is represented by a vector $\psi$ in the appropriate Hilbert space $\cH$. Two vectors that differ by a scalar represent the same state.
\begin{itemize}
    \item $\cH$ is called the \textbf{state space}.
    \item In classical mechanics, a state represents a \textbf{configuration} of the system. In quantum mechanics, a state instead encodes a probability distribution of configurations. More precisely, each $\psi \in \cH$ is a function on the space of configurations, and the probability of observing a set of configurations $D$ in a state $\psi$ is
    \[
        \frac{\int_D |\psi|^2}{\|\psi\|_2^2}.
    \]
    \item Very often, $\cH$ is the space of (complex) $\cL^2$ functions on the configuration space.
\end{itemize}

A \textbf{classical observable} of $\cS$ (i.e. a physical quantity that can be observed by taking a measurement of $\cS$) is a $\bC$-function $f$ on configuration space of $\cS$. To each classical observable, there exists a corresponding \textbf{quantum observable} $\hat{f}$ which is a self-adjoint operator on $\cH$.

If $\cS$ is in state represented by unit vector $\psi \in \cH$, then the expected value of measuring $f$ is
\[
    \mathbb E[f] = \langle \psi, \hat{f} \psi \rangle.
\]
This is the key relationship between $f$ and $\hat{f}$. Notice we are always measuring a classical observable. However, in a quantum state, a classical observable no longer takes a definite value; the corresponding quantum observable encodes the probabilities of what values the classical observable can take.

\subsection{Eigenvalues}

Suppose quantum observable $\hat{f}$ has an orthonomormal basis of eigenvectors $e_i$ with eigenvalues $\l_i$. Furthermore, suppose $\cS$ is in a state represented by unit vector $\psi \in \cH$, where
\[
    \psi = \sum a_i e_i.
\]
Note that
\[
    \mathbb E[f] = \sum \l_i |a_i|^2, \quad \sum |a_i|^2 = 1.
\]
This \underline{suggests but does not imply} that we only observe the $\l_i$'s when we measure $f$, with
\[
    \mathbb P(f = \l_i) = |a_i|^2.
\]
However, even though it is not implied, the above statement happens to be always true (indeed, it is the most natural). As a result, the possible values of $f$ are the eigenvalues of $\hat{f}$. Put more directly, \textbf{we only observe eigenvalues}.

To generalize this notion in the case where $\hat{f}$ does not have an orthonormal basis of eigenvectors, we will need introduce the \textbf{spectrum} and use the spectral theorem for unbounded operators. In this general setting, \textbf{we only observe the spectrum}, with probabilities given by spectral projections.

\subsection{Example}
A simple quantum system $\cS$ is a particle on the real line. In this case, we have:
\begin{itemize}
    \item Configuration space: $\bR$. Each configuration is a possible position of the particle.
    \item State space: $\cL^2(\bR)$.
    \item Common quantum observables are the position operator $X\psi(x) = x\psi(x)$ and the momentum operator $P\psi(x) = -i\hbar\frac{d\psi}{dx}$. Indeed, observe if $\psi$ is a unit vector, then
    \[
        \langle \psi, X\psi \rangle = \int_\bR x|\psi(x)|^2dx = \mathbb E[\text{position of particle}].
    \]
\end{itemize}
One may have noticed that the position and momentum operators are not defined on the whole domain $\cH = \cL^2(\bR)$. Also, one may recall that all self-adjoint operators are bounded, and the position and momentum are clearly not bounded. What's going on here?

It happens that the position and momentum operators can be defined on a dense subset of $\cH = \cL^2(\bR)$, and there is a notion of self-adjointness for such operators. In fact, we generalize the definition of linear operators to include such functions.

Also observe the position operator does not have an orthonormal basis of eigenvectors (in fact, it has none!). However, the spectrum of the position operator is $\bR$, implying that the particle may be observed anywhere on the real line $\bR$. (Something similar happens with the momentum operator).