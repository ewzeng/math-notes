\section{Finite Dimensional Representations of Lie Algebras}
It happens that studying finite dimensional irreducible representations of Lie groups is difficult, so we introduce the notion of finite dimensional Lie algebra representations.

\subsection{Definitions}
The definitions of Lie algebra representations parallel those of Lie groups. Suppose $\mathfrak g$ is a Lie algebra and let $V$ be a finite dimensional state space. Define $\gl(V)$ to be the space of operators on $V$, and note it is the associated Lie algebra of Lie group $\GL(V)$.
\begin{itemize}
    \item A (ordinary) representation of $\mathfrak g$ is a homomorphism $\mathfrak \pi: g \rightarrow \gl(V)$.
    \item A representation $\pi$ is unitary if $\pi(G) \subset \uu(v)$, the Lie algebra corresponding to $\U(V)$.
    \item A representation $\pi$ is irreducible if the only invariant subspaces under $\pi$ are $0$ and $V$.
\end{itemize}
Observe every representation of $G$ induces a representation of the corresponding Lie algebra $\mathfrak g$ with the differential map, as described previously.

\subsection{}
We are now ready for a key result: let $G$ be a connected Lie group and let $\Pi$ be a representation of $G$. Then $W \subset V$ is invariant under $\Pi$ if and only if it is invariant under the induced Lie algebra representation $\pi$. Thus:
\begin{center}
    $\Pi$ is irreducible if and only if $\pi$ is irreducible.
\end{center}

Therefore, to study finite dimensional irreducible representations of a connected Lie group $G$, we only need to study finite dimensional irreducible representations of the associated Lie algebra $\mathfrak g$. It happens that because Lie algebras are vector spaces, the latter task is easier than the former. This is one reason why we go through all the hassle of defining Lie algebra representations.

\subsection{A Look at What's to Come}
The hydrogen atom has a quantum Hamiltonian $\hat{H}$ that commutes with the rotation operators, which constitute a natural representation $\Pi$ of $\SO(3)$.

Recall that we can solve the Schr\"{o}dinger equation if we can find eigenvectors of $\hat{H}$. However, the commutativity with the rotation operators implies that the eigenspaces of $\hat{H}$ are invariant under $\Pi$, and thus are representations of $\SO(3)$. If we can classify all the irreducible representations of $\SO(3)$, this will allow us to find all eigenspaces (and thus eigenvectors) of $\hat{H}$. To classify all the irreducible representations of $\SO(3)$, we classify all irreducible representations of $\so(3)$, the associated Lie algebra.
