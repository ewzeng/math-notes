\subsection{Lie Groups and Algebras}
A matrix Lie group is a topologically closed subgroup of $GL(n; \bC)$, the set of invertible $n \times n$ matrices in $\bC$. A matrix Lie algebra is a vector subspace (over $\bR$) of $M_n(\bC)$ that is closed under the Lie bracket operation, which is defined as the commutator:
\[
    [X, Y] = XY - YX.
\]
$M_n(\bC)$ is the space of $n \times n$ matrices in $\bC$. In these notes, all Lie groups and algebras are matrix Lie groups and algebras unless otherwise specified.

\subsubsection{}
There is a natural way to associate every Lie group with a Lie algebra. Given a matrix Lie group $G$, we have
\[
    G \subset GL(n; \bC) \subset M_n(\bC) \cong \bR^{2n^2}
\]
so $G$ can be seen a real manifold inside $M_n(\bC)$. From differential topology, the tangent space of any point on a real manifold is defined as a vector subspace of the Euclidean space the manifold is embedded in. Thus, the tangent space of any point in $G$ is a (real) vector subspace of $M_n(\bC)$. If the tangent space is additionally closed under the Lie bracket operation, it becomes a natural matrix Lie algebra.

The last condition is satisfied for the tangent space $\mathfrak g$ of $G$ at the identity matrix, so $\mathfrak g$ is a Lie algebra. We define $\mathfrak g$ to be the associated Lie algebra of the Lie group $G$.

\subsubsection{}
Recall the tangent space of a point on a manifold can also be defined as the space of all the derivatives of smooth curves on the manifold passing through that point. In the case of $\mathfrak g$, it happens that we only need to consider a certain class of smooth curves on $G$ (passing through the identity) to get every element of $\mathfrak g$. More specifically, we only need to consider smooth curves of the form
\[
    t \rightarrow e^{tX}, \quad e^{tX} \in G \text{ for all } t \in \bR
\]
where the matrix exponential is defined by a power series. As
\[
    \frac{d e^{tX}}{dt} \Bigr|_{t = 0} = X.
\]
we conclude that
\[
    \mathfrak g = \{X \in M_n(\bC) : e^{tX} \in G \text{ for all } t \in \bR\}.
\]

\subsubsection{}
Recall a smooth map between manifolds induces a linear map (called the differential) between corresponding tangent spaces. In the case that the manifold is also a Lie group, the differential between the tangent spaces at the identity is also a Lie algebra homormophism. In this way, every Lie group homomorphism gives rise to a homomorphism between the corresponding Lie algebras.

\subsubsection{}
As we have seen, every matrix Lie group has a unique associated matrix Lie algebra. This fact can be generalized to abtract Lie groups and algebras. The converse is not true: a Lie algebra can have many associated Lie groups. That is, many Lie groups can generate the same Lie algebra. However, we can do better if we impose some restrictions:
\begin{itemize}
    \item Every abstract Lie algebra $\mathfrak g$ corresponds to a unique simply connected abstract Lie group $G^*$
    \item There is a group homomorphism $\phi$ from $G^*$ to any other connected abstract Lie group $G$ with the same algebra such that $\ker(\phi)$ is discrete.
\end{itemize}
In other words, in the category $\cC$ of connected abstract Lie groups with $\mathfrak g$ as their associated Lie algebra, $G^*$ is a universal object. We call $G^*$ the \textbf{universal covering group} because the discreteness of $\ker(\phi)$ implies that $G^*$ can be pictured ``wrapping" around any $G \in \cC$ some number of times. We also say the \textbf{universal cover} of $G$ is $G^*$. $\phi$ is called the \textbf{covering map}.

Two more important facts:
\begin{itemize}
    \item If $G$ is a compact matrix Lie group, then the universal $G^*$ is a matrix Lie group. This fact allows us to work again in the space of matrix Lie groups.
    \item $SU(2)$ is a universal cover of $SO(3)$
\end{itemize}
