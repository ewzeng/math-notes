\section{Finding Eigenvectors}
Now that we know the eigenspaces of $\hat{H}$ induce ``sub-representations" of the representation $\Pi$, we proceed as follows:
\begin{enumerate}
    \item We note $L^2(S^2)$ is invariant under the operators of $\Pi$, so we can study the induced sub-representation. In particular, we can break down this induced representation further into a direct sum of irreducible sub-representations. We find one specific decomposition into \textbf{spherical harmonics} and get
        \[
            L^2(S^2) = \cH_0 \oplus \cH_1 \oplus \cH_2 \oplus \dots
        \]
        We will define the spaces $\cH_l$ shortly.

    \item We show that if $V \subset L^2(\bR^3)$ induces an irreducible sub-representation of $\Pi$, then it is of the form
        \[
            \{p(x)f(|x|): p \in \cH_l\}
        \]
        for some $f$ and $\cH_l$.

    \item Recall that every sub-representation of $\Pi$ decomposes into a direct sum of irreducible sub-representations. Then from the previous discussions, we conclude every eigenvalue has an eigenvector of the form
        \[
            p(x)f(|x|)
        \]
        for some $f$ and some spherical harmonic $p$.
\end{enumerate}
The reader may have noticed we have begun to use the phrase ``induces a sub-representation of $\Pi$" instead of the simpler ``invariant subspace under the operators of $\Pi$." The two statements are equivalent, but we think the latter is the more intuitive way to think.

The rest of this section provides some more detail to the first two steps described above.

\subsection{Step 1: Spherical Harmonics}
We want to decompose $L^2(S^2)$ into a direct sum of irreducible spaces (i.e. spaces that induce irreducible sub-representations). Because we are dealing with an infinite dimensional representation, when we say ``direct sum" we mean ``direct sum followed by closure" (see the notes on Lie groups and algebras).

Let $\cP$ be the space of polynomials $\bC[x,y,z]$ restricted to $S^2$, which by Stone-Weierstrass is dense in $L^2(S^2)$. Because of this, if we can find a regular direct sum decomposition of $\cP$ into irreducible spaces, this will be a ``direct sum" of $L^2(S^2)$.

Define $\cP_l \subset \cP$ to be the space of homogenous polynomials of degree $l$. Define $\cH_l$ to be the space of spherical harmonics of degree $l$, i.e.
\[
    \cH_l = \{p \in \cP_l: \D p = 0\}
\]
To show
\[
    \cP = \bigoplus \cH_l
\]
we make the following (very elegant) argument:
\begin{enumerate}
    \item The Laplacian $\D$ is an operator $\cP_l \rightarrow \cP_{l-2}$ with kernel $\cH_l$.
    \item Multiplication by $r^2 = x^2 + y^2 + z^2$ is an operator $\cP_l \rightarrow \cP_{l+2}$. Note that $r^2p$ and $p$ represent the same function in $L^2(S^2)$ and thus are the same in $\cP$. Thus multiplying by $r^2$ is an embedding $\cP_l \rightarrow \cP_{l+2}$.
    \item We introduce an inner product on $\cP$ that makes $\D$ and $r^2$ adjoints of each other. This inner product is
        \[
            \langle f | g \rangle := \left[ f^*\left( \frac{\partial}{\partial x}, \frac{\partial}{\partial y}, \frac{\partial}{\partial z}\right) g(x,y,z) \right]_{(x,y,z) = (0, 0, 0)}
        \]
    \item The standard kernel and image relationship of adjoints will allow us to conclude that $(r^2\cP_{l-2})^\perp = \cH_l$, where orthogonality is from our newly introduced inner product. Thus we conclude
        \[
            \cP_l = \cH_1 \oplus r^2\cP_{l-2}.
        \]
    \item Now we are ready for the punchline. Let $p \in \cP$. We know $p \in \cP_l$ for some $l$, so $p = p_0 + q_0$ where $p_0 \in \cH_l$ and $q_0 \in r^2\cP_{l-2}$. Note $q_0$ is a function in $\cP_{l-2}$. If $q_0$ is nonzero, we repeat this process to $q_0$ and eventually get
        \[
            p = p_0 + p_1 + \dots + p_k
        \]
        where $p_i \in \cH_{l-2i}$.
\end{enumerate}
To show the $\cH_l$ are irreducible subspaces, we make the following argument:
\begin{enumerate}
    \item As $\D$ is rotationally symmetric, it commutes with all the rotation operators. Thus the kernel (e.g. 0-eigenspace) is an invariant subspace.
    \item Each $\cP_l$ is an invariant subspace. Then by taking intersections with $\ker(\D)$, we conclude that each $\cH_l$ is an invariant subspace.
    \item We use our theorem on the classification of irreducible representations of $\so(3)$ to prove each $\cH_l$ is irreducible (see notes on $\SO(3)$, $\so(3)$). More precisely, we show that the $\pi(F_i)$ produce the ``chain of eigenvectors". We can verify this by recalling the relationship between $\pi(F_i)$ and the componentwise angular momentum operators (see the optional section on them).
\end{enumerate}

\subsection{Step 2}
To prove that every irreducible invariant subspace $V \subset L^2(\bR^3)$ is of the form
\[
    \{p(x)f(|x|): p \in \cH_l\}
\]
the basic idea is to project $V$ to $L^2(\bR^3)$ and make some arguments using Schur's lemma. The key reasons why this works come from step 1:
\begin{itemize}
    \item Each $\cH_l$ is irreducible.
    \item The direct sum the $\cH_l$'s is $L^2(S^2)$.
\end{itemize}
