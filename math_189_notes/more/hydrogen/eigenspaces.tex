\section{Finding Eigenvectors}
Now that we know the eigenspaces of $\hat{H}$ induce ``sub-representations" of the representation $\Pi$, we proceed as follows:
\begin{enumerate}
    \item We note $L^2(S^2)$ is invariant under the operators of $\Pi$, so we can study the induced sub-representation. In particular, we can break down this induced representation further into a direct sum of irreducible sub-representations. From this, we will conclude:
        \[
            L^2(S^2) = \cH_0 \oplus \cH_1 \oplus \cH_2 \oplus \dots
        \]
        where each $\cH_l$ induces an irreducible sub-representation of $\Pi$. The spaces $\cH_l$ (which we will define shortly) make up the \textbf{spherical harmonics}.

    \item We show that if $V \subset L^2(\bR^3)$ induces an irreducible sub-representation of $\Pi$, then it is of the form
        \[
            \{p(x)f(|x|): p \in \cH_l\}
        \]
        for some $f$ and $\cH_l$.

    \item Recall that every sub-representation of $\Pi$ decomposes into a direct sum of irreducible sub-representations. Then from the previous discussions, we conclude every eigenvalue has an eigenvector of the form
        \[
            p(x)f(|x|)
        \]
        for some $f$ and some spherical harmonic $p$.
\end{enumerate}
The reader may have noticed we have begun to use the phrase ``induces a sub-representation of $\Pi$" instead of the simpler ``invariant subspace under the operators of $\Pi$." The two statements are equivalent, but we think the latter is the more intuitive way to think.

The rest of this section provides some more detail to the steps described above.

\subsection{Step 1: Spherical Harmonics}
We want to decompose $L^2(S^2)$ into a direct sum of irreducible spaces (under the operators of $\Pi$). Because we are dealing with an infinite dimensional representation, when we say ``direct sum" we mean ``direct sum followed by closure" (see my notes on Lie groups and algebras).

Let $\cP$ be the space of polynomials $\bC[x,y,z]$ restricted to $S^2$, which by Stone-Weierstrass is dense in $L^2(S^2)$. Because of this, if we can find a regular direct sum decomposition of $\cP$ into irreducible spaces, this will be a ``direct sum" of $L^2(S^2)$.

Define $\cP_l \subset \cP$ to be the space of homogenous polynomials of degree $l$.
