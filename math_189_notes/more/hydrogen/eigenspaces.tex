\section{Finding Eigenvectors}
Now that we know the eigenspaces of $\hat{H}$ induce ``sub-representations" of the representation $\Pi$, we proceed as follows:
\begin{enumerate}
    \item We note $L^2(S^2)$ is invariant under the operators of $\Pi$, so we can study the induced sub-representation. In particular, we can break down this induced representation further into a direct sum of irreducible sub-representations. From this, we will conclude:
        \[
            L^2(S^2) = V_0 \oplus V_1 \oplus V_2 \oplus \dots
        \]
        where each $V_i$ induces an irreducible sub-representation of $\Pi$. The spaces $V_i$ (which we will define shortly) are called the spaces of \textbf{spherical harmonics}.

    \item We show that if $V \subset L^2(\bR^3)$ induces an irreducible sub-representation of $\Pi$, then it is of the form
        \[
            \{p(x)f(|x|): p \in V_i\}
        \]
        for some $f$ and $V_i$.

    \item Recall that every sub-representation of $\Pi$ decomposes into a direct sum of irreducible sub-representations. Then from the previous discussions, we conclude every eigenvalue has an eigenvector of the form
        \[
            p(x)f(|x|)
        \]
        for some $f$ and some spherical harmonic $p$.
\end{enumerate}
The reader may have noticed we have begun to use the term ``induces a sub-representation of $\Pi$" instead of the simpler ``invariant subspace under the operators of $\Pi$." The two statements are equivalent, but we the latter is the more intuitive way to think about the decomposition of representations.

The rest of this section provides some more detail to the steps described above.
