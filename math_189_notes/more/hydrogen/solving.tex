\section{Solving the Schr\"{o}dinger Equation}
Now that we know every eigenvalue (i.e. energy level) of $\hat{H}$ has an eigenvector of the form
\[
    p(x)f(|x|), \quad p \in \cH_l
\]
we can plug this eigenvector back into the equation
\[
    \hat{H}\psi = -\frac{\hbar^2}{2m}\D\psi + V(|x|)\psi = E\psi
\]
to solve for the eigenvalues $E$. After some calculation, we get
\[
    - \frac{\hbar^2}{2m}\left[ \frac{d^2f}{dr^2} + \frac{2(l+1)}{r} \frac{df}{dr} \right] + V(r)f(r) = Ef(r), \quad r = |x|.
\]
\subsection{The Hydrogen Atom}
Unforunately, this is as far as we can get studying a general 3D system with radial potential. To go further, we need to know something about $V$. Thus, we go back to the hydrogen atom, which has a radial potential
\[
    V(r) = - \frac{Q^2}{r}.
\]
Hence the equation we want to solve is:
\[
    - \frac{\hbar^2}{2m}\left[ \frac{d^2f}{dr^2} + \frac{2(l+1)}{r} \frac{df}{dr} \right] - \frac{Q^2}{r}f(r) = Ef(r).
\]
It happens that the only eigenvalues we care about are the negative eigenvalues becuase they represent the ``bound states" of an electron in the hydrogen atom. The method of solving for eigenvalues $E < 0$ is as follows:
\begin{enumerate}
    \item Notice that for large $r$, the terms with $1/r$ vanish, and we get
        \[
            - \frac{\hbar}{2m} \frac{d^2f}{dr^2} \approx Ef
        \]
        As $f$ must be $L^2$, it cannot be a growing exponential, so it must behave like a decaying exponential. Thus we guess
        \[
            f = g(r) \exp\left\{ - \frac{\sqrt{2mE}}{\hbar}r\right\}
        \]
        where $g(r)$ grows at most polynomially. (This is where we use the fact $E < 0$.)

    \item In fact, we guess $g(r)$ is a polynomial. Plugging our solution into the equation, we get a recurrence relation of the coefficents of $g(r)$. We find that the highest exponent $n$ of $g(r)$ must satisfy
        \[
            - \frac{\sqrt{2mE}}{\hbar} n = \frac{mQ}{\hbar^2}
        \]
        which implies
        \[
            E = - \frac{mQ^2}{2\hbar^2n^2}
        \]
\end{enumerate}
Although this derivation can be made more rigorous, it captures the general idea and completely characterizes all the eigenvalues $E < 0$.

