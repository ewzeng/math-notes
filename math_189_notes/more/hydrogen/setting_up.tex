\section{Setting Up}

\subsection{Generalizing}
The quantum Hamiltonian of the wave equation for an electron in a hydrogen atom is:
\[
    \hat{H} = - \frac{\hbar^2}{2m}\Delta - \frac{Q^2}{|x|}
\]
where $m$ is the mass of the election, $x$ is the (3D) position of the election with respect to a fixed nucleus, and $Q$ is the charge of the electon.

We want to solve this equation. However, it happens that the method of solution in this system is virtually the same as in a general three dimensional quantum system with radial potential. Thus in this set of notes, we choose to generalize and solve the latter system.

The quantum Hamiltonian of the wave equation in a three-dimensional quantum system with radial potential is:
\[
    \hat{H} = - \frac{\hbar^2}{2m}\Delta + V
\]
where $V$ is any radial function in $\bR^3$.

\subsection{}
As always, to solve the wave equation, we first find the eigenvectors of $\hat{H}$. The radial potential of $\hat{H}$ intuitively suggests to us that the eigenvectors (which physically represent a stationary state) are also radially symmetric. That is, eigenvectors look like:
\[
    \psi(x) = p(x)f(|x|).
\]
where $p(x) \in L^2(S^2)$.

\subsection{}
There is another perspective of viewing radial symmetry via commutativity with the angular momentum operators. More specifically, ...
