\section{Setting Up}

\subsection{Generalizing}
The quantum Hamiltonian for an electron in a hydrogen atom is:
\[
    \hat{H} = - \frac{\hbar^2}{2m}\Delta - \frac{Q^2}{|x|}
\]
where $m$ is the mass of the election, $x$ is the (3D) position of the election with respect to a fixed nucleus, and $Q$ is the charge of the electon.

We want to solve the Schr\"{o}dinger equation for this Hamiltonian. However, it happens that the method of solution in this system is virtually the same as in a general three dimensional system with radial potential. Thus in this set of notes, we choose to generalize and solve the latter system.

The quantum Hamiltonian of a particle in a three-dimensional system with radial potential is:
\[
    \hat{H} = - \frac{\hbar^2}{2m}\Delta + V
\]
where $V$ is any radial function in $\bR^3$. We want to solve the Schr\"{o}dinger equation for this Hamiltonian.

\subsection{}
As always, to solve the Schr\"{o}dinger equation, we first find the eigenvectors of $\hat{H}$. This will be our next goal.

\subsection{Exploiting Symmetry}
The rotational symmetry of $\hat{H}$ implies that if we rotate a vector $\psi \in L^2(\bR^3)$, apply $\hat{H}$, and then rotate the resulting vector back, the final vector should be the same vector obtained by applying $\hat{H}$ to $\psi$.

More formally, consider the natural action of the rotation group $\SO(3)$ on $L^2(\bR^3)$ given by:
\[
    \psi(x) \mapsto \psi(R^{-1}x), \quad R \in \SO(3).
\]
This gives rise to a infinite dimensional unitary representation of $\SO(3)$:
\[
    (\Pi(R)\psi)(x) = \psi(R^{-1}x).
\]
(As mentioned in a previous set of notes, infinite dimensional representations require extra technical conditions to check, which we will not do here). Our discussion in the beginning can be formally written as:
\[
    \Pi(R)^{-1}\hat{H}\Pi(R)\psi = \hat{H}\psi.
\]
This implies $\hat{H}\Pi(R) = \Pi(R)\hat{H}$. That is, $\hat{H}$ commutes with all the rotation operators. Linear algebra tells us this implies the eigenspaces of $\hat{H}$ are invariant under $\Pi(R)$ for all $R \in \SO(3)$. In other words, the eigenspaces $\hat{H}$ induce representations of $\SO(3)$.

\subsection{(Optional) Angular Momentum Operators}
Let $\pi$ be the Lie algebra representation $\so(3) \rightarrow \gl(L^2(\bR^3))$ induced by $\Pi$. Let $F_1, F_2, F_3$ be the usual basis of $\so(3)$ (see the set of notes of $\SO(3), \so(3)$). Then it happens that the
\[
    i\hbar\pi(F_j)
\]
are the componentwise angular momentum operators. More specifically,
\[
    \begin{split}
        i\hbar\pi(F_1) &= X_2P_3 - X_3P_2\\
        i\hbar\pi(F_2) &= X_3P_1 - X_1P_3\\
        i\hbar\pi(F_3) &= X_1P_2 - X_2P_1\\
    \end{split}
\]
where $X_i, P_i$ are the position and momentum operators in each component. Considering the relationship between Lie groups and Lie algebras, this implies that the angular momentum operators are the ``infinitesimal" generators of the rotation group operators.

\subsubsection{Aside}
Attentive readers may have noticed that $\GL(L^2(\bR^3)$ is not a Lie group because it is infinite-dimensional, and by the same reason, nor is $\gl(L^2(\bR^3))$ a Lie algebra. How can we then define the Lie algebra representation $\pi$ induced by $\Pi$? The definition we have been using is done in the finite dimensional case, where $\GL$ is a Lie group and $\gl$ is a Lie algebra, so we were able to consider Lie algebra homomorphisms induced by Lie group homomorphisms.

The answer lies in a slight workaround. Let $\mathfrak g$ be the Lie algebra corresponding to $G$. Recall
\[
    \mathfrak g = \{X \in M_n(\bC) : e^{tX} \in G \text{ for all } t \in \bR\}.
\]
If $\Pi'$ is a Lie group homomorphism from $G$, the induced Lie algebra homormophism $\pi'$ can alternatively be defined as the map that satisfies
\[
    \Pi'(e^{tX}) = e^{t\pi'(X)}.
\]
This is an equivalent definition to the one we have been using. However, this definition can be used to define induced homomorphisms when the target is no longer a Lie group. Thus we use it to define $\pi$:
\[
    \Pi(e^{tX}) = e^{t\pi(X)}.
\]
This definition also allows us to more rigorously say that the angular momentum operators are the ``infinitesimal" generators of the rotation group operators (previously we were intuitively assuming $\GL(L^2(\bR^3))$ was a Lie group and $\gl(L^2(\bR^3))$ was the corresponding Lie algebra).
