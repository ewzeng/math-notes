\section{Overview}
\subsection{Motivation}
$\SO(3)$ is an important Lie group for two reasons:
\begin{enumerate}
    \item The hydrogen atom has a quantum Hamiltonian $\hat{H}$ that commutes with the angular momentum operators, which constitute a natural representation $\Pi$ of $\SO(3)$.

    Recall that we can solve the wave equation if we can find eigenvectors of $\hat{H}$. However, the commutativity with the angular momentum operators implies that the eigenspaces of $\hat{H}$ are invariant under $\Pi$, and thus are representations of $\SO(3)$. Therefore, if we can classify all representations of $\SO(3)$, this will allow us to find all eigenvectors of $\hat{H}$.

    \item  When we study spin, we will be taking the tensor product with a vector space that carries a finite dimensional projective unitary representation of $\SO(3)$.
\end{enumerate}
Therefore, our goal in this set of notes will be to classify all ordinary and projective unitary representations of $\SO(3)$.

\subsection{Roadmap}
Because $\SO(3)$ is compact, to classify all (ordinary) representations of $\SO(3)$, it suffices to classify all the irreducible representations of $\SO(3)$. Furthermore, because $\SO(3)$ is connected, every irreducible representation of $\SO(3)$ induces a irreducible representation of $\so(3)$, so we only need to study the irreducible representations of $\so(3)$.

By applying one of the de-projectivization lemmas, we see that every finite dimensional projective unitary representation of $\SO(3)$ induces a finite dimensional representation of $\so(3)$. Using the same argument as Lie groups, every finite dimensional representation of $\so(3)$ is a direct sum of irreducible representations of $\so(3)$.

Therefore, the key to everything is to classify the irreducible representations of $\so(3)$. 
