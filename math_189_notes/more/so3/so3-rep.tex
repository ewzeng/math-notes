\section{Irreducible Representations of so(3)}

\subsection{}
Exponentiating $\SO(3)$, it is not hard to show
\[
    \so(3) = \{X \in M_n(\bR): X^* = - X\}
\]
Thus we have a natural basis for $\so(3)$:
\[
    F_1 =
    \begin{pmatrix}
        0&0&0\\
        0&0&-1\\
        0&1&0
    \end{pmatrix}
    \quad
    F_2 = 
    \begin{pmatrix}
        0&0&1\\
        0&0&0\\
        -1&0&0
    \end{pmatrix}
    \quad
    F_3 =
    \begin{pmatrix}
        0&-1&0\\
        1&0&0\\
        0&0&0
    \end{pmatrix}
\]
We are now ready for our main result.

\subsection{Classification of Irreducible Representations}
\begin{thm}
    Let $\pi: \so(3) \rightarrow \gl(V)$ be a finite dimensional irreducible representation of $\so(3)$. Define the operators:
    \[
        L^+ = i\pi(F_1) - \pi(F_2)
    \]
    \[
        L^- = i\pi(F_1) + \pi(F_2)
    \]
    \[
        L_3 = i\pi(F_3)
    \]
    Then there exists a basis of $v_0, \dots, v_{2l}$ of eigenvectors of $L_3$ such that:
    \[
        \begin{split}
            L_3v_j &= (l - j )v_j\\
           L^- v_j &= 
        \begin{cases}
            v_{j+1} & j < 2l\\
            0   & j = 2l
        \end{cases}\\
           L^+ v_j &= 
        \begin{cases}
            j(2l + 1 - j)v_{j-1} & j < 2l\\
            0   & j = 0
        \end{cases}
        \end{split}
    \]
\end{thm}
Intuitively, the $v_i$'s form a ``chain" of eigenvectors and the operators $L^+$ and $L^-$ ``raise" and ``lower" eigenvectors along the chain, respectively.

\subsection{Interpretation}
The theorem states that the structure of any finite dimensional irreducible representation of $\so(3)$ is completely determined by the number $l$ (which may not be an integer if the dimension of $V$ is odd). We call $l$ the \textbf{spin} of the representation $\pi$. 

If $\pi: \so(3) \rightarrow \gl(V)$ is a finite dimensional irreducible representation with spin $l$, we often denote $V = V_l$.

