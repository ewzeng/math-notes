\documentclass{beamer}
%%%%%%%%%%%%%%%%%%%%%%%%%%%%%%%%%%%%%%%%%%%%%%%%%%%%%%%
\usepackage{amsfonts}
\usepackage{amsmath,amssymb,amsthm,graphicx,mathrsfs,url}
\usepackage{multimedia}
\AtBeginDocument{%
  \SetMathAlphabet{\mathrm}{normal}{\encodingdefault}{cmss}{\mddefault}{n}
  \SetMathAlphabet{\mathrm}{bold}{\encodingdefault}{cmss}{\bfdefault}{n}}

\AtBeginSection{\frame{\sectionpage}}

\theoremstyle{plain}
\newtheorem{thm}{Theorem}
\newtheorem{prb}{Problem}
\newtheorem{lem}[thm]{Lemma}
\newtheorem{prop}[thm]{Proposition}
\newtheorem{cor}[thm]{Corollary}
\newtheorem{oppm}[thm]{Open Problem}

\theoremstyle{definition}
\newtheorem{defn}{Definition}[section]
\newtheorem{conj}{Conjecture}[section]
\newtheorem{exmp}{Example}[section]
\newtheorem{coex}{Counterexample}[section]

%Macros for Greek Letters
\renewcommand{\a}{\alpha}
\renewcommand{\b}{\beta}
\renewcommand{\d}{\delta}
\newcommand{\D}{\Delta}
\newcommand{\e}{\varepsilon}
\newcommand{\g}{\gamma}
\newcommand{\G}{\Gamma}
\renewcommand{\l}{\lambda}
\renewcommand{\L}{\Lambda}
\newcommand{\s}{\sigma}
\renewcommand{\th}{\theta}
\renewcommand{\o}{\omega}
\renewcommand{\O}{\Omega}
\renewcommand{\S}{\Sigma}
\renewcommand{\t}{\tau}
\newcommand{\var}{\varphi}
\newcommand{\z}{\zeta}

%Macros for math cal letters
\newcommand{\cA}{{\mathcal A}}
\newcommand{\cB}{{\mathcal B}}
\newcommand{\cC}{{\mathcal C}}
\newcommand{\cD}{{\mathcal D}}
\newcommand{\cE}{{\mathcal E}}
\newcommand{\cF}{{\mathcal F}}
\newcommand{\cH}{{\mathcal H}}
\newcommand{\cI}{{\mathcal I}}
\newcommand{\cK}{{\mathcal K}}
\newcommand{\cL}{{\mathcal L}}
\newcommand{\cM}{{\mathcal M}}
\newcommand{\cN}{{\mathcal N}}
\newcommand{\cO}{{\mathcal O}}
\newcommand{\cP}{{\mathcal P}}
\newcommand{\cS}{{\mathcal S}}
\newcommand{\cT}{{\mathcal T}}
\newcommand{\cU}{{\mathcal U}}
\newcommand{\cV}{{\mathcal V}}
\newcommand{\cW}{{\mathcal W}}
\newcommand{\cY}{{\mathcal Y}}

%Macros for blackboard bold letters
\newcommand{\bZ}{{\mathbb Z}}
\newcommand{\bR}{{\mathbb R}}
\newcommand{\bC}{{\mathbb C}}
\newcommand{\bT}{{\mathbb T}}
\newcommand{\bN}{{\mathbb N}}
\newcommand{\bQ}{{\mathbb Q}}
\newcommand{\bF}{{\mathbb F}}

%Other macros
\renewcommand{\i}{\infty}


\beamertemplatenavigationsymbolsempty
\usetheme{Madrid}
\title[Spectral Theorem]{Spectral Theorem for Unbounded Self-Adjoint Operators}
\author[E. Zeng]{Edward Zeng}
\institute[UC Berkeley]{University of California, Berkeley \protect\\ DRP}
\date{\today}

\begin{document}

\maketitle

\begin{frame}{Classical vs Quantum Mechanics}
    In classical mechanics: study a system by studying observables (e.g. position, momentum).

    \bigskip
    In quantum mechanics:
    \begin{itemize}
        % Don't say ``usually" a Hilbert space. It is a Hilbert space!
        \item State space $\cH$ is a Hilbert space
        % Self-adjoint = complex version of symmetric
        \item Observables correspond to self-adjoint operators on $\cH$
        \item Instead of studying observables, study corresponding operators
    \end{itemize}
\end{frame}

\begin{frame}{Position Operator}
    For a particle in $\bR$, usually:
    \begin{itemize}
        % Mention that L^2 means integrable
        \item $\cH = L^2(\bR)$, each state a ``probability distribution" of the position of particle
        \item Position operator $X: \psi(x) \mapsto x\psi(x)$
    \end{itemize}

    \pause
    \bigskip
    For many reasons, want to study eigenvalues of $X$:
    \begin{itemize}
        \item Eigenvectors represent possible positions of particle
        \item Eigenvalues encode probability of such positions
    \end{itemize}
    So want some sort of a spectral theorem!
    % Mention spectral theorem for matrices
\end{frame}

\begin{frame}{Two Technical Problems}
    Two technical problems:
    \begin{itemize}
        % Given an example?
        \item $X$ is not defined on all of $\cH = \cL^2(\bR)$! $X\psi$ might not be integrable.
            \begin{itemize}
                % Mention position operator is just multiplication by $x$
                \item Ex: $ \frac{1}{x^2} \mapsto \frac{1}{x}$
            \end{itemize}
        \item $X$ has no eigenvectors! Proof:
            \begin{itemize}
                \item $\l \psi(x) = x\psi(x)$
                \item $\psi(x)$ is an eigenvector only if $\psi(x) = 0$ everywhere but a point
            \end{itemize}
    \end{itemize}

    \pause
    \bigskip
    The solutions:
    \begin{itemize}
        \item Develop a theory for ``operators" defined on a dense subset of domain
        \item Don't study eigenvalues, study the spectrum (``eigenvalues" of ``approximate eigenvectors")
    \end{itemize}
\end{frame}

\section*{What does the spectral theorem say?}

\begin{frame}{Integration}
    Let's talk integration!
    \[
        \int_E f(\l) d\mu
    \]

    \bigskip
    Regular measure-theory integration:
    \begin{itemize}
        \item $f: \bR \rightarrow \bR$
        \item $\mu: \text{Borel subsets of } \bR \rightarrow \bR$
        \item Integral is some real value
    \end{itemize}

    \pause
    \bigskip
    \textbf{Operator-valued} integration:
    \begin{itemize}
        \item $f: \bR \rightarrow \bR$
        \item $\mu: \text{Borel subsets of } \bR \rightarrow \text{projections}$
        \item Integral is some \textit{operator}
    \end{itemize}

    $\mu$ is called a ``projection-valued" measure.
\end{frame}


\begin{frame}{Spectrum vs Eigenvalue}
    Given operator $A$:
    \begin{itemize}
        \item $\l$ is an eigenvalue if $Ax - \l x$ is not \textit{injective}
        \item $\l$ is part of the \textbf{spectrum} if $Ax - \l x$ is not \textit{invertible}
    \end{itemize}
    Spectrum denoted $\s(A)$. Eigenvalues $\subset \s(A)$.

    \pause
    \bigskip
    If $A$ self-adjoint:
    \begin{itemize}
        \item $\s(A) \subset \bR$
        \item $\l \in \s(A)$ if exist sequence $\{\psi_n\}$ with $\|\psi_n\| = 1$ and
            \[
                \lim_{n \rightarrow \i} \|A\psi_n - \l \psi_n\| = 0
            \]
        \item Where the notion ``approximate eigenvector" arises
    \end{itemize}
\end{frame}

\begin{frame}{Spectrum of $X$}
    Given any $\l \in \bR$:
    \begin{itemize}
        \item Suppose $\psi_n$ has a tall bump near $x = \l$ and zero elsewhere
        \item Note $X\psi_n = x\psi_n(x) \approx \l \psi_n(x)$
        \item Can show $\l$ in spectrum
    \end{itemize}

    So $\s(X) = \bR$. Approximate eigenvector = Dirac delta at $\l$
\end{frame}

\begin{frame}{Statement}
    \begin{theorem}[Spectral Theorem for Unbounded Self-Adjoint Operators]
        If $A$ is self-adjoint, then there exists a unique projection-valued measure $\mu^A$ such that
        \[
            \int_{\s(A)} \l d\mu^A = A.
        \]
    \end{theorem}
    \medskip
    Intuition for $\mu^A$: given $E \subset \bR$
    \begin{itemize}
        \item $v_i$ = ``approximate eigenvectors" with ``eigenvalues" in $E$
        \item $P$ = projection that projects onto $\text{span}\{v_i\}$
        \item $\mu^A(E) = P$
    \end{itemize}
\end{frame}

\begin{frame}{Proof Sketch}
    Proof sketch:
    \begin{enumerate}
        \item Prove for bounded self-adjoint operators
            \begin{itemize}
                \item Define continuous functional calculus
                \item Generalize to the measurable functions
            \end{itemize}
        \item Extend to bounded normal case
        \item Extend to unbounded self-adjoint case
            \begin{itemize}
                \item Cayley transform
            \end{itemize}
    \end{enumerate}
\end{frame}

\begin{frame}{Acknowledgements}
    \begin{center}
        THANKS YONAH FOR BEING A WONDERFUL MENTOR!
    \end{center}
\end{frame}

\end{document}
