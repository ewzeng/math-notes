\documentclass[12pt, letterpaper]{article}
\usepackage[utf8]{inputenc}
\usepackage{amsmath,amssymb,amsthm}
\usepackage{xcolor} %For colored text
\usepackage{tcolorbox}
\usepackage{bbm}
\tcbuselibrary{breakable,skins} %tcolorbox libraries

%Title and margin formatting
\usepackage[left=2cm,top=3cm,bottom=3cm,right=2cm]{geometry} %Customize margins.
\usepackage{fancyhdr}
\setlength{\headheight}{15pt} %To avoid compiler warnings (12pt too small)
\fancyhead[L]{Edward Zeng}
\fancyhead[C]{Mathematical Methods in Mechanics}
\fancyhead[R]{\today}

%Inkscape
\usepackage{import}
\usepackage{pdfpages}
\usepackage{transparent}

\newcommand{\incfig}[2][1]{%
    \def\svgwidth{#1\columnwidth}
    \import{./figures/}{#2.pdf_tex}
}

%Environments
\theoremstyle{plain}
\newtheorem{thm}{Theorem}[section]
\newtheorem{cor}[thm]{Corollary}
\newtheorem{lem}[thm]{Lemma}
\newtheorem{exercise}{Exercise}

\theoremstyle{definition}
\newtheorem{dfn}{Definition}[section]

\theoremstyle{remark}
\newtheorem*{remark}{Remark}

\newenvironment{details}[1]
{ \leavevmode \\ \noindent \textit{#1}.}
{}

\tcolorboxenvironment{thm}{
    enhanced jigsaw,colframe=cyan,interior hidden,
    breakable,before skip=10pt,after skip=10pt}
\tcolorboxenvironment{cor}{
    enhanced jigsaw,colframe=cyan,interior hidden,
    breakable,before skip=10pt,after skip=10pt}
\tcolorboxenvironment{lem}{
    enhanced jigsaw,colframe=cyan,interior hidden,
    breakable,before skip=10pt,after skip=10pt}
\tcolorboxenvironment{dfn}{
    enhanced jigsaw,colframe=lightgray,interior hidden,
    breakable,before skip=10pt,after skip=10pt}
\tcolorboxenvironment{proof}{
    blanker, breakable,left=5mm,
    before skip=10pt, after skip=10pt,
    borderline west={1mm}{0pt}{red}}
\tcolorboxenvironment{details}{
    blanker, breakable,left=5mm,
    before skip=10pt, after skip=10pt,
    borderline west={1mm}{0pt}{green}}
\tcolorboxenvironment{remark}{
    blanker, breakable,left=5mm,
    before skip=10pt, after skip=10pt,
    borderline west={1mm}{0pt}{orange}}

%Macros for Greek Letters
\renewcommand{\a}{\alpha}
\renewcommand{\b}{\beta}
\renewcommand{\d}{\delta}
\newcommand{\D}{\Delta}
\newcommand{\e}{\varepsilon}
\newcommand{\g}{\gamma}
\newcommand{\G}{\Gamma}
\renewcommand{\l}{\lambda}
\renewcommand{\L}{\Lambda}
\newcommand{\s}{\sigma}
\renewcommand{\th}{\theta}
\renewcommand{\o}{\omega}
\renewcommand{\O}{\Omega}
\renewcommand{\S}{\Sigma}
\renewcommand{\t}{\tau}
\newcommand{\var}{\varphi}
\newcommand{\z}{\zeta}

%Macros for math cal letters
\newcommand{\cA}{{\mathcal A}}
\newcommand{\cB}{{\mathcal B}}
\newcommand{\cC}{{\mathcal C}}
\newcommand{\cD}{{\mathcal D}}
\newcommand{\cE}{{\mathcal E}}
\newcommand{\cF}{{\mathcal F}}
\newcommand{\cH}{{\mathcal H}}
\newcommand{\cI}{{\mathcal I}}
\newcommand{\cK}{{\mathcal K}}
\newcommand{\cL}{{\mathcal L}}
\newcommand{\cM}{{\mathcal M}}
\newcommand{\cN}{{\mathcal N}}
\newcommand{\cO}{{\mathcal O}}
\newcommand{\cP}{{\mathcal P}}
\newcommand{\cS}{{\mathcal S}}
\newcommand{\cT}{{\mathcal T}}
\newcommand{\cU}{{\mathcal U}}
\newcommand{\cV}{{\mathcal V}}
\newcommand{\cW}{{\mathcal W}}
\newcommand{\cY}{{\mathcal Y}}

%Macros for blackboard bold letters
\newcommand{\bZ}{{\mathbb Z}}
\newcommand{\bR}{{\mathbb R}}
\newcommand{\bC}{{\mathbb C}}
\newcommand{\bT}{{\mathbb T}}
\newcommand{\bN}{{\mathbb N}}
\newcommand{\bQ}{{\mathbb Q}}
\newcommand{\bF}{{\mathbb F}}

%Other macros
\renewcommand{\i}{\infty}
\DeclareMathOperator{\ran}{ran}
\DeclareMathOperator{\sinc}{sinc}
\DeclareMathOperator{\spann}{span}
\newcommand{\hsep}{\noindent\rule{\linewidth}{0.4pt}}
