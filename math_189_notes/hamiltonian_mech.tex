\section{Hamiltonian Mechanics}
The Hamilton equations provide a way to describe many different systems. The components are:
\begin{itemize}
    \item $q(t)$: The state, usually a vector of the position of all particles.
    \item $p(t)$: How the state is changing, usually a vector of the momentum of all the particles.
    \item $H(p, q)$: The hamiltonian, which is a value that remains constant as the system evolves, usually total energy. $H$ is given as a function of $p$ and $q$.
\end{itemize}
The Hamilton equations are:
\[
    \dot{p} = - \frac{\partial H}{\partial q}, \quad \dot{q} = \frac{\partial H}{\partial p}.
\]
Intuitively, these equations tell us that $p$ and $q$ evolve in a way that keep $H$ constant.

\hsep

\noindent
\textbf{Example.} Consider the system of a particle in a conservative force field $F$. Note if $F = -\nabla V$, then $V$ represents potential energy. We let $q$ denote the particle's position, $p$ the particle's momentum, and define:
\[
    H(p, q) = \text{kinetic} \ + \ \text{potential} = \frac{p^2}{2m} + V(q).
\]
Then the Hamilton equation tell us
\[
    \dot{q} = \frac{p}{m}, \quad \dot{p} = - V'(q).
\]
The first equation relates position and momentum, and the second tells us that any change in momentum will be reflected in the potential energy of the particle such that the total energy stays constant (e.g. as the particle gains potential energy, it slows down).

\hsep

Observe that given initial conditions and a suitable hamiltonian, we can use the equations to solve for $p(t)$ and $q(t)$. In other words, the Hamilton equations fully describe the system, a powerful property.

We call the space of possible $(p, q)$ the \textbf{phase space} and we define \textbf{phase flow} to be the vector field
\[
    (\dot{p}, \dot{q}) = \left( - \frac{\partial H}{\partial q}, \frac{\partial H}{\partial p}\right)
\]
which represents how the system will evolve from a given point in the phase space. Using Gauss's divergence theorem, we can show that the phase flow is volume preserving (i.e. divergence is zero everywhere). This is an example of a general property of Hamiltonian systems: phase flow preserves the sympletic. For now, think of the sympletic as some volume measurement using differential forms.
