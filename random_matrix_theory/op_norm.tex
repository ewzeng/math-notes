\section*{Operator Norm of Random Matrices}

There are two different ways to control the operator norm of random matrices.

\paragraph{Epsilon-net argument.} An $\e$-net is a maximal set of points on the unit sphere $S$ such that each point is separated by a distance at least $\e$. The $\e$-net argument works as follows: let $\S$ be a $1/2$-net (other $\e$ choices work too). Observe the triangle inequality implies
\[
    \begin{split}
        \|M\|_{op} \ge \l & \implies \text{for some $y \in S$ we have } |My| > \l            \\
                          & \implies \text{for some $y \in \S$ we have } |My| > \frac{\l}{2} \\
    \end{split}
\]
Then by the union bound, we have
\[
    \bP(\|M\|_{op} \ge \l) \le \sum_{y \in \S} \bP\left(|My| > \frac{\l}{2}\right).
\]
Note we have transformed a difficult value to control (operator norm) into controlling a finite number of simpler events. This is the essence of the $\e$-net argument: reduce a complex event into simpler events, and the use metric arguments to reduce into a finite number of simpler events.

\paragraph{Moment method.} This method works only for Hermitian random matrices. If an $n \times n$ Hermitian matrix $M$ has eigenvalues $\l_i$, then
\[
    \|M\|_{op} = \max |\l_i|.
\]
Then it is not difficult to show
\[
    \|M\|^k_{op} \leq \Tr(M^k) = \sum \l_i^k \leq n\|M\|^k_{op}
\]
when $k$ is even. Thus, to control the operator norm, it suffices to bound $\bE \Tr(M^k)$ and apply Markov. This bounding be done with careful combinatorial arguments.

The moment method yields a slighter weaker upper bound than $\e$-net with an extra $\log n$ factor (lower bound fine). Removing the $\log n$ factor is rather technical.

The advantage of the moment method is that not only does it show $\|M\| \sim O(\sqrt{n})$, it produces the exact bound:
\[
    \limsup_{n \to \i} \frac{\|M_n\|}{\sqrt{n}} \le 2 \ \text{a.s.} \quad (\text{i.e. the Bai-Yin Theorem}).
\]

%%% Local Variables:
%%% TeX-master: "main"
%%% End: