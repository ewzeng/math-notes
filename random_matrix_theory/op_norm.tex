\section*{Operator Norm of Random Matrices}

There are different ways to control the operator norm of random matrices.

\paragraph{Epsilon-net argument.} An $\e$-net is a maximal set of points on the unit sphere $S$ such that each point is separated by a distance at least $\e$. The $\e$-net argument works as follows: let $\S$ be a $1/2$-net (other $\e$ choices work too). Observe the triangle inequality implies
\[
    \begin{split}
        \|M\|_{op} \ge \l & \implies \text{for some $y \in S$ we have } |My| > \l            \\
                          & \implies \text{for some $y \in \S$ we have } |My| > \frac{\l}{2} \\
    \end{split}
\]
Then by the union bound, we have
\[
    \bP(\|M\|_{op} \ge \l) \le \sum_{y \in \S} \bP\left(|My| > \frac{\l}{2}\right).
\]
Note we have transformed a difficult value to control (operator norm) into controlling a finite number of simpler events. This is the essence of the $\e$-net argument: reduce a complex event into simpler events, and the use metric arguments to reduce into a finite number of simpler events.

%%% Local Variables:
%%% TeX-master: "main"
%%% End: