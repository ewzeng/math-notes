\section{A Gas of Electrons}
Note we can rewrite the joint distribution of eigenvalues as
\[
    \begin{split}
        P(\l_1, \dots, \l_n) &= Z^{-1} \prod_{i < j} |\l_i-\l_j|^2e^{-n\sum V(\l_i)}\\
        &= Z^{-1} \exp\left\{- \left( \sum_{i \neq j} \log |\l_i-\l_j|^{-1} + n \sum V(\l_i) \right) \right\}\\
    \end{split}
\]
In statistical mechanics, this happens to be the equation representing the distribution of a gas of electrons in a plane where the electrons are confined to a 1-D wire!
\begin{itemize}
    \item $Z$ happens to be the partition function of this electron gas!
    \item The position of the electrons are the eigenvalues (the log difference in the equation represents the logarithmic Couloumb potential of a 2D gas). Thus we have a ``repulsion of eigenvalues.''
    \item $V$ represents an external confining potential that prevents the eigenvalues escaping to infinity.
\end{itemize}

