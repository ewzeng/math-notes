\section*{Central Limit Theorem}

CLT states that for iid (real) random variables $X_1, \dots, X_n$ of mean $\mu$ and variance $\s^2$, the distribution of the random variable
\[
    Z_n = \frac{S_n - n\mu}{\sqrt{n}\s}
\]
approaches the unit Gaussian as $n \to \i$. (Note we divide by $\sqrt{n}$ because in the previous section we noted $S_n$ is concentrated in an interval of size $O(\sqrt{n})$). Here we give different proofs of CLT to showcase different techniques.

\paragraph{Fourier Method.} For a (real-valued) random variable $X$, define the characteristic function
\[
    F_X(t) = \bE e^{itX} = \int e^{itX} d\mu = \int_\bR e^{itX} f(X) dX
\]
where $f(X)$ is the pdf of $X$. Note:
\begin{itemize}
    \item $F_X(t)$ is the Fourier transform of $f(X)$. By Fourier transform properties, we thus expect
    \[
        \text{$X$ and $Y$ have same pdf} \iff F_X(t) = F_Y(t).
    \]
    \item Thus to prove CLT, it suffices to show
    \[
        F_{Z_n}(t) \to F_{N(0,1)} = e^{-t^2/2}
    \]
    as $n \to \i$. We can compute $F_{Z_n}(t)$ by computing $F_{X_i}(t)$'s (say, with power series), and putting everything together with the identity $F_{X+Y}(t) = F_{X}(t)F_{Y}(t)$.
\end{itemize}

\paragraph{Moment Method.} The basic idea of the moment method is to note that under nice conditions (e.g. Gaussian falloff for the $Y_i$'s), we have
\[
    Y_i \to Y \iff \bE Y_i^k \to \bE Y^k \text{ for all } k \in \bN.
\]
The proof comes from noting that the $\bE Y_i^k$'s encode the same information as characteristic function $F_{Y_i}$. (In fact, from this perspective, the moment method is exactly the same as the Fourier method. However, later we will see that the moment method is easier to generalize.)

The proof of CLT then works as follows:
\begin{enumerate}
    \item Suppose the $X_i$'s are bounded (can generalize to unbounded by truncation). By concentration inequalities previously deduced, the $Z_n$'s have Gaussian falloffs.
    \item Thus it suffices to show
    \[
        \bE Z_n^k \to \bE N(0,1)^k \text{ for all } k \in \bN.
    \]
    $\bE Z_n^k$ can be computed by expanding and applying counting arguments.
\end{enumerate}

%%% Local Variables:
%%% TeX-master: "main"
%%% End: