\section{Localization}
Given a commutative ring $A$, we can construct the field of fractions:
\[
    \left\{ \frac{a}{b} : a, b \in A, b \neq 0\right\}
\]
with the equivalence relation
\[
    \frac{a}{b} = \frac{c}{d} \iff \text{ there exists nonzero $s'' \in A$ such that } s''(ad - bc) = 0.
\]
The $s''$ is needed to make the equivalence relation transitive. If $A$ is an integral domain, we can drop the $s''$ and achieve the usual cross-multiply equivalence relation.

However, there many are ``intermediate" constructions that don't construct all the fractions, but only fractions with certain denominators. Let $S \subset A - \{0\}$ be a set that contains the identity and is closed under multiplication. We can consider $S$ as the set of denominators, and define the ring
\[
    S^{-1}A = \left\{ \frac{a}{b} : a \in A, b \in S \right\}
\]
with the equivalence relation
\[
    \frac{a}{b} = \frac{c}{d} \iff \text{ there exists nonzero $s'' \in S$ such that } s''(ad - bc) = 0.
\]
This construction is called a localization. The reason is because if $\mathfrak p \subset A$ is a prime ideal and $S = A - \mathfrak p$, the resulting construction $S^{-1}A$ is a local ring, i.e. it has a unique maximal ideal.

Note if $A$ is an integral domain then it can be naturally embedded in any localization $S^{-1}A$. We require $A$ to be an integral domain because else the equivalence relation may identify
\[
    \frac{a}{1} = \frac{b}{1}, \quad a,b \in A
\]
even if $a \neq b$.
