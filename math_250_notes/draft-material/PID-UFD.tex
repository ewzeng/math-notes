\section{PIDs, UFDs}
Call a ring principal if every ideal is principal. Principal integral domains (PIDs) often arise from rings that are equipped with a division algorithm. Indeed, we can show that if a ring has a division algorithm, then every ideal is divisble by the ``smallest" element of the ideal (however size is defined in the division algorithm), and thus generated by that element.

It should come as no surprise then that that PIDs are UFDs, i.e. every non-zero non-unit can be factored uniquely into a product of irreducible elements (up to permutation and units). The proof is roughly the following:
\begin{itemize}
    \item Show irreducible elements are prime (i.e. generate prime ideals)
    \item Prove existence of factorization in similar way as existence of factorization over $\bZ$:
        \begin{itemize}
            \item Let $\{a_i\}$ denote the set of unfactorable elements.
            \item Pick a maximal ideal $(a) \in \{(a_i)\}$. (That is, pick a ``minimal" element in $\{a_i\}$. Can do this by applying Noetherian property.)
            \item Show if $(a)$ is not irreducible, we can factor $a$ into smaller elements, i.e. create a larger unfactorable ideals. Contradiction.
        \end{itemize}
\end{itemize}
