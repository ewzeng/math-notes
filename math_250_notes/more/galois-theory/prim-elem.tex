\section{Primitive Elements}

Call an extension $E/F$ \textbf{simple} if $E = F(\a)$ for some $\a$, where $\a$ is called a \textbf{primitive element}. The primitive element theorem states that all separable extensions are simple, and as a consequence, our earlier analysis of separable extensions was complete.

To prove the primitive element theorem, we proceed as follows:
\begin{enumerate}
    \item We show that if $E/F$ is separable, then $K/F$ is separable for any intermediate field $F \subset K \subset E$.
    \item We show if $F(\b, \g)/F$ is separable, the there exists $\a$ such that $F(\b, \g) = F(\a)$.
    \item Step 1 allows us to iteratively apply Step 2 to eventually find an $\a$ such that $E = F(\a)$.
\end{enumerate}

We elaborate more on the first two steps.

\subsection{Step 1: Separability of Intermediate Fields}

\subsection{Step 2: Reduction}

