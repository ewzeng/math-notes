\section{Basics}

A ring is a set $R$ with two operations $+, \times$ such that:
\begin{itemize}
    \item $(R, +)$ is an abelian group
    \item $\times$ is associative and distributes over $+$
    \item A multiplicative identity exists (some definitions omit this)
\end{itemize}

Here are some important definitions and concepts we expect the reader to be familiar with:
\begin{itemize}
    \item A unit is an element with a multiplicative inverse.
    \item An ideal is an additive subgroup of $R$ closed under left and right multiplication by $R$.
    \item The kernel of ring homomorphisms are precisely the ideals.
    \item An ideal is principle is it is generated by one element.
    \item $R$ is a commutative ring if $\times$ is commutative.
    \item An integral domain is a commutative ring with no zero divisors (except 0).
    \item If $R$ is commutative: quotient by prime ideal $\implies$ integral domain, quotient by maximal ideal $\implies$ field.
    \item The previous point implies maximal $\implies$ prime. The converse is false.
\end{itemize}

Two standard examples of rings are $\bZ$ and $k[x]$ (polynomials over field $k$).