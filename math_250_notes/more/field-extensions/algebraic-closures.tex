\section{Algebraic Closures}
An \textbf{algebraic closure} of $F$ is a ``minimal" algebraically closed field $L$ that $F$ can be embedded in. More precisely, suppose $F$ can be embedded in some algebraically closed field $E$. Then the subfield $L \subset E$ is an algebraic closure of $F$ if it is algebraically closed and only contains elements that are algebraic over $F$.

\subsection{Existence}
Suppose $F$ is embedded in some algebraically closed field $E$, as above. To prove the existence of an algebraic closure $L \subset E$, we simply define
\[
    L = \{\text{ all elements in $E$ algebraic over $F$ }\}
\]
and show $L$ is an algebraically closed field.

At first glance, this doesn't seem very easy. Suppose $\a$ and $\b$ are roots of two polynomials in $F[x]$. How do we show $\a + \b$ is a root of some polynomial in $F[x]$? Constructing a polynomial with $\a + \b$ as a root is very difficult. However, we can bypass the constructive proof with a very elegant argument:
\begin{enumerate}
    \item As $\a$ and $\b$ are algebraic, $F(\a, \b)$ is finite dimensional over $F$.
    \item Note $\a + \b \in F(\a, \b)$. As a result, $F(\a + \b) \subset F(\a, \b)$ is finite dimensional over $F$, and thus $\a + \b$ is algebraic.
\end{enumerate}
The same argument suffices for $\a\b$ and $\a/\b$. The fact that $L$ is a field follows. We apply a similar argument to show $L$ is algebraically closed:
\begin{enumerate}
    \item Let $a_0 + a_1x + \dots + a_nx^n \in L[x]$, and let $\a \in E$ be a root. Note $F[a_0, \dots, a_n, \a]$ is a finite dimensional extension of $F[a_0, \dots, a_n]$, which is a finite dimensional extension of $F$.
    \item We conclude $F[\a]$ is finite dimensional over $F$, and thus $\a$ is algebraic, i.e. $\a \in L$.
\end{enumerate}

\subsection{Uniqueness}
We prove the following theorem:
\begin{center}
    If $K$ and $L$ are algebraic closures of $F$, then $K$ and $L$ are isomorphic.
\end{center}
Unforunately, the isomorphism is not unique, so instead of referring to ``the" algebraic closure, we still say ``an" algebraic closure.

The proof is to construct a map $\s: K \rightarrow L$ that is identity on $F$. Then after showing the $\s(K)$ is an algebraic closure of $F$, the observation
\[
    F \subset \s(K) \subset L
\]
will allow us to conclude $\s(K) = L$, and thus $\s$ is an isomorphism. Alternatively, we could make an argument similar to the one presented in the discussion on fixed base fields to deduce that $\s$ is an isomorphism.

To construct $\s$, we start with the natural inclusion $F \rightarrow L$, and keep extending it until we get a map $K \rightarrow L$. More precisely:
\begin{enumerate}
    \item Start with $\s_F : F \rightarrow L$.
    \item Given $\s_M: M \rightarrow L$, we can extend to $\s_{M(\a)}: M(\a) \rightarrow L$ for some algebraic $\a \not \in M$ by defining
        \[
            \a \mapsto \text{ a root of the minimal polynomial $p$ of $\a$.}
        \]
        This is indeed a homomorphism because $M(\a)$ is isomorphic to $M[x]/(p)$, so if a map $M[x] \rightarrow L$ vanishes on $(p)$, then it induces a map $M[x]/(p) = M(\a) \rightarrow L$. The map $\s_{M(\a)}$ is induced from this way.
    \item Repeat until we get a map $\s: K \rightarrow L$. Because every element in $K$ is algebraic, this process will ``eventually" end. We can put things in more formal terms using Zorn's lemma.
\end{enumerate}
