\section{PIDs, UFDs}
\begin{itemize}
    \item A principal ideal domain (PID) is an integral domain that is principal (i.e. every ideal is principal).
    \item A irreducible element is an unfactorable element.
    \item A unique factorization domain (PID) is an integral domain such that every element is uniquely factorable into irreducible elements.
\end{itemize}

PIDs often arise from rings that are equipped with a division algorithm. Indeed, we can show that if a ring has a division algorithm, then every ideal is divisble by the ``smallest" element of the ideal (however size is defined in the division algorithm), and thus generated by that element.

It should come as no surprise then that PIDs are UFDs. The proof is done in two steps:
\begin{enumerate}
    \item Prove existence of factorization in similar way as existence of factorization over $\bZ$:
        \begin{itemize}
            \item Let $\{a_i\}$ denote the set of unfactorable elements.
            \item Pick a maximal ideal $(a) \in \{(a_i)\}$. (That is, pick a ``minimal" element in $\{a_i\}$. Can do this by applying the Noetherian property of PIDs.)
            \item Show if $(a)$ is not irreducible, we can factor $a$ into smaller elements, i.e. create larger unfactorable ideals. Contradiction.
        \end{itemize}
    \item Show irreducible elements are prime (i.e. generate prime ideals). This allows us to show that the factorization is unique.
\end{enumerate}

Indeed, it is not difficult to show for PIDs that prime $\iff$ irreducible $\iff$ maximal. 
