\section{Dedekind Rings}
Given a commutative ring $A$, we can ask if the set of non-zero ideals form a group. That is, given ideals $I$ and $J$, we want to be able to find an ideal $K$ such that
\[
    IK = J.
\]
This would be a powerful property. However, it would almost never be true. For instance, in $\bZ$, there is no ideal $K$ such that
\[
    (2\bZ)K = 3\bZ.
\]
However, if we could assign meaning to a ``fractional" ideal
\[
    \frac{3}{2}\bZ
\]
then things could be different. This is the motivation behind Dedekind rings.

Suppose $A$ can be embedded in its field of fractions $K$ (i.e. $A$ is an integral domain). Define a fractional ideal $\mathfrak a$ of $A$ to be a non-zero additive subgroup of $K$ such that
\[
    A\mathfrak a = \mathfrak a
\]
and such that there exists nonzero $c \in A$ with
\[
    c\mathfrak a \subset A, \quad \text{(the ``bounded denominator" condition)}
\]
Note regular ideals are also fractional ideals. We say that $A$ is a Dedekind ring if the fractional ideals form a group.
