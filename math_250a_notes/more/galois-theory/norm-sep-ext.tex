\section{Normal and Separable Extensions}

Let $E$ be an algebraic extension of a base field $F$. We are interested in the homomorphisms $\s: E \to \bar{E}$ that fix $F$. Pictorially:
\[
    \begin{tikzcd}
        E \arrow[r, "\sigma"]  & \bar{E} \\
        F \arrow[u] \arrow[ru] &        
    \end{tikzcd}
\]
We say:
\begin{itemize}
    \item $E/F$ is a \textbf{normal extension} if the image of every $\s$ is in $E$. That is, every $\s$ is an automorphism of $E$.
    \item $E/F$ is a \textbf{separable extension} if $[E : F] = n$ is finite and and there are precisely $n$ distinct $\s$'s.
\end{itemize}

To get an intuition for these two definitions, we consider the scenario $E = F(\a)$ with $p$ being the minimal polynomial of $\a$. In this case, every $\s: E \to \bar{E}$ is uniquely determined by the value of $\s(\a)$, which must be a root of $p$. This implies:
\begin{itemize}
    \item If $E/F$ is normal, then every root of $p$ is in $E$. Thus, $E$ is the splitting field of $p$.
    \item If $E/F$ is separable, then every root of $p$ is distinct. This is true for most polynomials and thus suggests that most finite extensions that one encounters are separable.
\end{itemize}

It happens that the assumption $E = F(\a)$ well illuminates the structure of normal and separable extensions. In fact, it is not hard to show that an extension is normal if and only if it is a splitting field of a family of polynomials. And in the next section we will see that every separable extension is of the form $E = F(\a)$.