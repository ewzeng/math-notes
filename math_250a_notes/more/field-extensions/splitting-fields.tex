\section{Splitting Fields}
Let $F$ be a field contained in a field $E$. We say a polynomial $f \in F[x]$ \textbf{splits} in $E$ if all the all the roots of $f$ are in $E$. That is, $f$ can be factored into linear factors in $E$.

We say $E$ is an \textbf{splitting field} of $f$ if it is a minimal field extension of $F$ such that $f$ splits. Similar to algebraic closures, we have a uniqueness theorem:
\begin{center}
    If $K$ and $L$ are splitting fields of $f \in F[x]$, then $K$ and $L$ are isomorphic.
\end{center}
Let $\bar{\cdot}$ denote algebraic closure and let $\a_1, \dots, a_n \in \bar{F}$ be the roots of $f$. One proof is to pick an isomorphism $\bar{K} \to \bar{F}$ that fixes $F$ and show that this induces an isomorphism between $K$ and $F(\a_1, \dots, \a_n)$. The same argument can be repeated for $L$.

The definitions of splitting and splitting fields easily generalize to a family of polynomials. The uniqueness result will still hold.
