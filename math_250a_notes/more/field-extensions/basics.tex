\section{Basics}
\subsection{}
Suppose $E$ is a field extension of a field $F$, i.e. $E$ is a field with $F \subset E$. Then $E$ can be seen as a vector space over $F$. To emphasize this fact, we often write $E$ as $E/F$. We define
\[
    [E : F] = \text{ dimension of } E/F
\]
and call this the \textbf{degree} of the field extension $E$ over $F$. If we have three fields $k \subset F \subset E$, then we have the following very useful and intuitive relation:
\[
    [E : k] = [E : F] \cdot [F : k].
\]

\subsection{}
Suppose again that $E$ is a field extension of $F$. If $\a \in E$, we denote $F(\a)$ the smallest field in $E$ that contains both $F$ and $\a$. We often say $F(\a)$ is ``the field $F$ extended by $\a$." We distinguish $F(\a)$ from $F[\a]$, the set of elements in $E$ that can be written as a polynomial in $\a$ with coefficents in $F$.
