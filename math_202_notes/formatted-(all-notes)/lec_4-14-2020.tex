\section{Lecture 4-14-2020}
We now return to Hilbert spaces operators. The following is an interesting representation trick for positive operators.
\begin{thm}[Positive Operator Representation]
    \label{positive operator representation}
    If $T \in \cB(\cH)$ is a positive operator, then there exists positive $S \in \cB(\cH)$ such that $T = S^2$. We often denote $S = \sqrt{T}$.
\end{thm}
\begin{remark}
    (1) We define $T$ to be positive if $ \langle Tv, v \rangle \geq 0$ for all $v \in \cH$. (2) Our previous use of $\sqrt{ }$ is a specific instance of this definition. In particular, we can now define $|T|$ for any $T \in \cB(\cH)$, not just compact $T$. (3) We will prove this theorem later.
\end{remark}
Now we are ready to introduce the concept of trace.
\begin{dfn}[Trace]
    Let $T \in \cB(\cH)$ be a positive operator. Then define the trace
    \[
        \tau(T) = \sum \langle Te_i, e_i\rangle
    \]
    where $\{e_i\}$ is any orthonormal basis.
\end{dfn}
\begin{remark}
    The difficulty of defining trace in general are technicalities surrounding the convergence of the sum. Therefore we first define trace for the positive operators and then extend to a larger class of operators.
\end{remark}
\begin{details}{Some technical details}
To show $\tau(T)$ is irrespective of (orthonormal) basis comes from the positive operator representation trick followed by Parseval's identity.
\end{details}

We now extend the idea of trace.

\begin{dfn}
    Let
    \begin{itemize}
        \item $\mathfrak{m}_\t \subset \cB(\cH)$ be the $\bC$-span of the positive operators with finite trace.
        \item $\mathfrak{n}_\t \subset \cB(\cH)$ be the set of elements $S$ s.t. $S^*S \in \mathfrak{m}_\t$.
    \end{itemize}
\end{dfn}

Trace can be naturally defined on $\mathfrak{m}_\t$ (note it may be complex but not infinite). We would like to study $\mathfrak{m}_\t$ and $\mathfrak{n}_\t$. To do so, we go to a more general setting.

\begin{dfn}
    Let $A$ be a C*-algebra. A weight is a function $\o: A^+ \rightarrow \bR^+ \cup \i$ such that $\o(a+b) = \o(a) + \o(b)$ and $\o(ra) = r\o(a)$ when $r \geq 0$.
\end{dfn}

\begin{remark}
    In this generalization, we replace $\cB(\cH)$ with $A$ and $\t$ with $\o$. $\mathfrak m_\o$ and $\mathfrak n_\o$ are defined analogously.
\end{remark}
