\section{Lecture 3-12-2020}
Regularity properties of measures are important because they allow us to approximate arbitrary measurable sets with ``nicer" sets. Thus we can often extend properties on nicer sets to a more general class.

Here are some implications about regularity:
\begin{itemize}
    \item $\mu$ Borel measure, outer regular, inner regular for open sets $\implies$ inner regular for $\s$-finite Borel sets. (Borel measure = defined on Borel sets and finite on compact sets.)
    \item $\mu$ Borel measure, open sets $\s$-compact $\implies$ $\mu$ is regular (i.e. both inner and outer regular).
\end{itemize}
\begin{details}{Intuition}
    The proofs of these implications are a bit technical (but not hard). But intuitively, the statements all say something about approximation by ``nice" sets, so there is no surprise the statements are related.
\end{details}
In light of R-M, this means that positive Radon measures correspond to ``well-behaved" Borel measures.

From R-M, we can deduce R-M-like statements (sometimes also called R-M) for the duals of $C_\i(X)$, and when $X$ is compact, $C(X)$.
