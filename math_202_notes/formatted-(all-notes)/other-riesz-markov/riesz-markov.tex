\documentclass[12pt, letterpaper]{article}
\usepackage[utf8]{inputenc}
\usepackage{amsmath,amssymb}
\usepackage{xcolor} %For colored text

%Title and margin formatting
\usepackage[left=2cm,top=3cm,bottom=3cm,right=2cm]{geometry} %Customize margins.
\usepackage{fancyhdr}
\setlength{\headheight}{15pt} %To avoid compiler warnings (12pt too small)
\fancyhead[L]{Edward Zeng, SID: 3034036984}
\fancyhead[C]{Math 202B Notes}
\fancyhead[R]{\today}

%Inkscape
\usepackage{import}
\usepackage{pdfpages}
\usepackage{transparent}

\newcommand{\incfig}[2][1]{%
    \def\svgwidth{#1\columnwidth}
    \import{./figures/}{#2.pdf_tex}
}

%Macros for Greek Letters
\renewcommand{\a}{\alpha}
\renewcommand{\b}{\beta}
\renewcommand{\d}{\delta}
\newcommand{\D}{\Delta}
\newcommand{\e}{\varepsilon}
\newcommand{\g}{\gamma}
\newcommand{\G}{\Gamma}
\renewcommand{\l}{\lambda}
\renewcommand{\L}{\Lambda}
\newcommand{\s}{\sigma}
\renewcommand{\th}{\theta}
\renewcommand{\o}{\omega}
\renewcommand{\O}{\Omega}
\renewcommand{\S}{\Sigma}
\renewcommand{\t}{\tau}
\newcommand{\var}{\varphi}
\newcommand{\z}{\zeta}

%Macros for math cal letters
\newcommand{\cA}{{\mathcal A}}
\newcommand{\cB}{{\mathcal B}}
\newcommand{\cC}{{\mathcal C}}
\newcommand{\cD}{{\mathcal D}}
\newcommand{\cE}{{\mathcal E}}
\newcommand{\cF}{{\mathcal F}}
\newcommand{\cH}{{\mathcal H}}
\newcommand{\cI}{{\mathcal I}}
\newcommand{\cK}{{\mathcal K}}
\newcommand{\cL}{{\mathcal L}}
\newcommand{\cM}{{\mathcal M}}
\newcommand{\cN}{{\mathcal N}}
\newcommand{\cO}{{\mathcal O}}
\newcommand{\cP}{{\mathcal P}}
\newcommand{\cS}{{\mathcal S}}
\newcommand{\cT}{{\mathcal T}}
\newcommand{\cU}{{\mathcal U}}
\newcommand{\cV}{{\mathcal V}}
\newcommand{\cW}{{\mathcal W}}
\newcommand{\cY}{{\mathcal Y}}

%Macros for blackboard bold letters
\newcommand{\bZ}{{\mathbb Z}}
\newcommand{\bR}{{\mathbb R}}
\newcommand{\bC}{{\mathbb C}}
\newcommand{\bT}{{\mathbb T}}
\newcommand{\bN}{{\mathbb N}}
\newcommand{\bQ}{{\mathbb Q}}
\newcommand{\bF}{{\mathbb F}}

%Other macros
\renewcommand{\i}{\infty}

\begin{document}
\pagestyle{fancy}
\section*{Reisz-Markov}
Here, we wish to provide a intuitive presentation of Reisz-Markov and a motivation of its proof.

Suppose $\phi$ is a positive linear functional on $C_c(\bR)$, i.e. a positive Radon measure. It is not far-fetched to hope for a representation like
\[
    \phi(f) = \int f d \mu_\phi.
\]
After all, integration plays a large part in the dual spaces of $L^q$. So we ask ourselves: if such a representation existed, what would $\mu_\phi$ look like?

Let $f$ be the following function:
\begin{figure}[ht]
    \centering
    \incfig[0.5]{A-Simple-Example}
    \caption{A Simple Example}
    \label{fig:A-Simple-Example}
\end{figure}

By linearity, we can decompose $\phi(f) \approx 1 \cdot \phi(g_{(1,2)}) + 2 \cdot \phi(g_{(2,3)})$ where $g_E$ is the continuous analog to the charateristic function $\chi_E$, i.e.
\begin{figure}[ht]
    \centering
    \incfig[0.5]{Decomposition}
    \caption{Decomposition}
    \label{fig:Decomposition}
\end{figure}

This suggests that we should set $\mu_\phi(E) = \phi(g_E)$. If you think about it (intuitively), this indeed turns $\phi$ into an integral. Hence, we start our construction of $\mu_\phi$ by defining
\[
    \mu_\phi(U) = \text{sup}\{\phi(f): 0 \leq f \leq \chi_U, \ f \in C_c(X), \ \text{supp}\{f\} \subset U\}
\]
for open sets $U$. (Note as $\phi$ is positive, this supremum is really just approximating $\chi_U$ by continuous functions.)

\end{document}
