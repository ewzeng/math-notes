\documentclass[12pt, letterpaper]{article}
\usepackage[utf8]{inputenc}
\usepackage{amsmath,amssymb, amsthm}
\usepackage{xcolor} %For colored text
\newtheorem*{remark}{Remark}
\newtheorem{exercise}{Exercise}

%Title and margin formatting
\usepackage[left=2cm,top=3cm,bottom=3cm,right=2cm]{geometry} %Customize margins.
\usepackage{fancyhdr}
\setlength{\headheight}{15pt} %To avoid compiler warnings (12pt too small)
\fancyhead[L]{Edward Zeng, SID: 3034036984}
\fancyhead[C]{Math 202B Notes}
\fancyhead[R]{\today}

%Inkscape
\usepackage{import}
\usepackage{pdfpages}
\usepackage{transparent}

\newcommand{\incfig}[2][1]{%
    \def\svgwidth{#1\columnwidth}
    \import{./figures/}{#2.pdf_tex}
}

%Macros for Greek Letters
\renewcommand{\a}{\alpha}
\renewcommand{\b}{\beta}
\renewcommand{\d}{\delta}
\newcommand{\D}{\Delta}
\newcommand{\e}{\varepsilon}
\newcommand{\g}{\gamma}
\newcommand{\G}{\Gamma}
\renewcommand{\l}{\lambda}
\renewcommand{\L}{\Lambda}
\newcommand{\s}{\sigma}
\renewcommand{\th}{\theta}
\renewcommand{\o}{\omega}
\renewcommand{\O}{\Omega}
\renewcommand{\S}{\Sigma}
\renewcommand{\t}{\tau}
\newcommand{\var}{\varphi}
\newcommand{\z}{\zeta}

%Macros for math cal letters
\newcommand{\cA}{{\mathcal A}}
\newcommand{\cB}{{\mathcal B}}
\newcommand{\cC}{{\mathcal C}}
\newcommand{\cD}{{\mathcal D}}
\newcommand{\cE}{{\mathcal E}}
\newcommand{\cF}{{\mathcal F}}
\newcommand{\cH}{{\mathcal H}}
\newcommand{\cI}{{\mathcal I}}
\newcommand{\cK}{{\mathcal K}}
\newcommand{\cL}{{\mathcal L}}
\newcommand{\cM}{{\mathcal M}}
\newcommand{\cN}{{\mathcal N}}
\newcommand{\cO}{{\mathcal O}}
\newcommand{\cP}{{\mathcal P}}
\newcommand{\cS}{{\mathcal S}}
\newcommand{\cT}{{\mathcal T}}
\newcommand{\cU}{{\mathcal U}}
\newcommand{\cV}{{\mathcal V}}
\newcommand{\cW}{{\mathcal W}}
\newcommand{\cY}{{\mathcal Y}}

%Macros for blackboard bold letters
\newcommand{\bZ}{{\mathbb Z}}
\newcommand{\bR}{{\mathbb R}}
\newcommand{\bC}{{\mathbb C}}
\newcommand{\bT}{{\mathbb T}}
\newcommand{\bN}{{\mathbb N}}
\newcommand{\bQ}{{\mathbb Q}}
\newcommand{\bF}{{\mathbb F}}

%Other macros
\renewcommand{\i}{\infty}

\begin{document}
\pagestyle{fancy}

\section*{Inductive Limit Topology}
\subsection*{Notation}
Let $X$ be LCH. Let $C_c(X)$ be the subset of $C(X)$ with compact support. For $S \subset X$, let $C_\i(S)$ be the subset of $C(S)$ that ``vanishes at infinity," i.e. $\e$-bounded outside a compact set.

\subsection*{Definition of ILT}
The ILT on $C_c(X)$ is the strongest topology such that the natural inclusion
\[
    C_\i(U) \rightarrow C_c(X)
\]
is continuous for all open $U$ with compact closure. (Here, we equip $C_\i(U)$ with the $\|\cdot\|_\i$ norm.)

\begin{remark}
    In general, there is no natural inclusion from $C(S) \rightarrow C_c(X)$ for any $S \subset X$. Nor is there a natural inclusion $C_\i(U) \rightarrow C_c(X)$ if $U$ is not open or does not have compact closure.
\end{remark}

\subsection*{Motivation}
Why are we concerned with the ILT? Note that $C_c(X)$ with $\|\cdot\|_\i$ norm makes all the maps $C_\i(U) \rightarrow C_c(X)$ continuous, so the $\|\cdot\|_\i$ topology is weaker than the ILT.

This makes the ILT topology easier to work with because the ILT allows for more continuous linear functionals than $\|\cdot\|_\i$. (Think about it, a stonger topology means more continuous maps can be defined on it.) For instance, every positive Radon measure is continuous for the ILT, but this is not the case for $\|\cdot\|_\i$. 

\begin{exercise}
    For $X = \bR$, what is an example of a open set in the ILT that is not an open set for $\|\cdot\|_\i$? Hint: consider the positive Radon measure given by
    \[
        \phi(f) = \int f d \lambda
    \]
    where $\l$ is the usual measure on $\bR$. Note $\phi$ is discontinuous for $\|\cdot\|_\i$, and consider $\phi^{-1}(0,1)$.
\end{exercise}

\subsection*{Criterion for Continuity}
Suppose $T$ is a linear functional defined on $C_c(X)$. Then $T$ is continuous for the ILT if for all $U$ open with compact closure, the map
\[
    T|_{C_\i(U)}: C_\i(U) \rightarrow \bR
\]
is continuous. ($C_\i(U)$ is equipped with the $\| \cdot\|_\i$ norm.) Think about why.

\end{document}
