\section{Lecture 4-2-2020}
\begin{thm}
    If $T \in \cB(\cH)$ is a self-adjoint compact operator, then $\|T\|$ and/or $-\|T\|$ are eigenvalues.
\end{thm}
\begin{details}{Proof gist}
    Construct a sequence $\{\z_n\}$ such that $\|\z_n\| = 1$ and $T(\z_n) - \|T\|\z_n \rightarrow 0$ or $T(\z_n) + \|T\|\z_n \rightarrow 0$ (This is called an ``approximate eigenvector"). Use compactness to select the limit $\z$ of a subsequence of $\z_n$.
\end{details}
\begin{details}{More detail}
    For self-adjoint operators, note $ \sup\{|\langle T(\z), \z\rangle|: \|\z\| \leq 1\} = \|T\|$. This is how we construct the sequence.
\end{details}
\begin{thm}
    Let $T \in \cB(\cH)$ be a self-adjoint compact operator. Then
    \[
        \cH = \overline{\bigoplus \cH_\l}, \quad \l \ \text{eigenvalue}
    \]
    where $\cH_\l$ is the $\l$-eigenspace.
\end{thm}
\begin{details}{Proof gist}
    Let
    \[
        \cK = \overline{\bigoplus \cH_\l}, \quad \l \ \text{eigenvalue}.
    \]
    Note $T$ is invariant over $\cK^\perp$, so $T_{\cK^\perp} = T_*$ makes sense. As $T_*$ has eigenvalues $\|T_*\|$ or $-\|T_*\|$, we conclude that $\cK^{\perp} = \{0\}$.
\end{details}
\begin{remark}
    This is one example of studying operator by studying the invariant subspaces. This is a powerful technique.
\end{remark}
From the theroem above, we can construct a orthonormal basis of eigenvectors (i.e. the spectral thereom).
