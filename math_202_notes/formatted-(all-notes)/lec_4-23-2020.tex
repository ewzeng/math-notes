\section{Lecture 4-23-2020}
We begin by introducing a notation for rank 1 operators.
\begin{dfn}
    For $\xi, \eta \in \cH$, let $D_{\xi \eta}(\zeta) = \langle \zeta, \eta) \xi$
\end{dfn}
By Riesz-Representation, all rank 1 operators can be represented as so.
\begin{thm}
    $\cB(\cH) = (\cB^1(\cH))'$, where $M \mapsto \t(M \cdot)$.
\end{thm}
\begin{details}{Surjectivity}
    To show all continuous functionals are of the form $\t(M \cdot)$, suppose $\phi \in (\cB^1(\cH))'$. We constuct $M$ such that $\phi$ and $\t(M \cdot)$ agree on all rank 1 operators, and thus $\phi = \t(M \cdot)$. To do this, we apply the Riesz-Representation of bounded sesquilinear forms to find an $M$ such that
    \[
        \langle M\xi, \nu \rangle = \phi(D_{\xi \eta}).
    \]
\end{details}
\begin{thm}
    $\cB^1(\cH) = (\cB_c(\cH))'$, where $T \mapsto \t(\cdot T)$.
\end{thm}
\begin{details}{Surjectivity}
    To show all continuous functionals are of the form $\t(\cdot T)$, suppose $\psi \in (\cB_c(\cH))'$. Maneuver into the setting $\psi \in (\cB^2(\cH))'$ and as $\cB^2(\cH)$ is a Hilbert space, apply Riesz-Representation to get
    \[
        \psi(S) = \t(TS) = \t(ST)
    \]
    on $\cB^2(\cH)$. (Play around to show the commutativity.) Then apply finite projection trick and polar decomposition to show $T \in \cB^1(\cH)$. As $\cB^2(\cH)$ is dense in $\cB^1(\cH)$, this result extends to $\cB^1(\cH)$.
\end{details}
\begin{remark}
    If we see trace as an integral, then we obtain some familiar expressions. $\t(M\cdot)$ becomes
    \[
        \int M \cdot d\mu
    \]
    and $\t(\cdot T)$ becomes
    \[
        \int \cdot T d\mu.
    \]
    Moreover, $\t(T^*S)$ becomes
    \[
        \int T^*S d\mu.
    \]
    This suggests that the motivation for some expressions involving trace (or weights in general) come from seeing trace (and weights) as some sort of integral.
\end{remark}
