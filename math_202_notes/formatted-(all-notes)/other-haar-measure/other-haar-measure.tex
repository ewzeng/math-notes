% Usage: This preamble assumes the macro \topic has been defined
% Thus, you should use the preamble like so:
%
% \def\topic{topic-name}
% % Usage: This preamble assumes the macro \topic has been defined
% Thus, you should use the preamble like so:
%
% \def\topic{topic-name}
% % Usage: This preamble assumes the macro \topic has been defined
% Thus, you should use the preamble like so:
%
% \def\topic{topic-name}
% \input{preamble.tex}
% \begin{document}
% ...
% \end{document}

% ==========================================================================
% The design of this document is inspired by Berkeley EECS
% The exam class creates cooler looking documents than the article class
\documentclass[letterpaper, 12pt]{exam} 
\usepackage[utf8]{inputenc}
\usepackage{amsmath,amssymb,amsthm}

% Format header and footer
% chead[T1]{T2} puts T1 on the first-page header and T2 on the other headers
% textsf yields san serif font
\pagestyle{headandfoot}
\extraheadheight[1.3in]{0in}
\extrafootheight{-.2in}
\rhead[\Large \textsf{Notes}\\
       \vspace{10pt}
]{}
\chead[\hrule height 3pt
       \vspace{10pt}
       \Large \textsf{Math 250A}\\
       \vspace{5pt}
       \Large \textsf{\topic}\\
       \vspace{10pt}
       \hrule height 3pt
]{}
\footer{}{\thepage}{}

% Change font for section and subsection headings to san serif
% The font sizes are kept default
\usepackage{titlesec}
\titleformat*{\section}{\Large\sffamily}
\titleformat*{\subsection}{\large\sffamily}
\titleformat*{\subsubsection}{\normalsize\sffamily}

% Format paragraphs by setting indents to zero and modifying the space between
% paragraphs
\usepackage{parskip}

% The LaTex exam class has the following bug:
% It states that margins should be 1-in on all sides for letterpaper.
% However, clearly the left margin is greater than the right margin.
% This problem does not persist in other types of paper (try a4paper)
% The hack fix is to include this package. I have no idea why.
\usepackage{hyperref}

% Theorem Environments
% Note: everything is un-numbered
\theoremstyle{definition} % Hack to make theorem environments not italicized
\newtheorem*{thm}{Theorem}
\newtheorem*{cor}{Corollary}
\newtheorem*{lem}{Lemma}
\newtheorem*{exercise}{Exercise}
\newtheorem*{dfn}{Definition}

% Be able to draw commutative diagrams
\usepackage{tikz-cd}

% Macros for Greek Letters
\renewcommand{\a}{\alpha}
\renewcommand{\b}{\beta}
\renewcommand{\d}{\delta}
\newcommand{\D}{\Delta}
\newcommand{\e}{\varepsilon}
\newcommand{\g}{\gamma}
\newcommand{\G}{\Gamma}
\renewcommand{\l}{\lambda}
\renewcommand{\L}{\Lambda}
\newcommand{\s}{\sigma}
\renewcommand{\th}{\theta}
\renewcommand{\o}{\omega}
\renewcommand{\O}{\Omega}
\renewcommand{\t}{\tau}
\newcommand{\var}{\varphi}
\newcommand{\z}{\zeta}

% Macros for math cal letters
\newcommand{\cA}{{\mathcal A}}
\newcommand{\cB}{{\mathcal B}}
\newcommand{\cC}{{\mathcal C}}
\newcommand{\cD}{{\mathcal D}}
\newcommand{\cE}{{\mathcal E}}
\newcommand{\cF}{{\mathcal F}}
\newcommand{\cH}{{\mathcal H}}
\newcommand{\cI}{{\mathcal I}}
\newcommand{\cK}{{\mathcal K}}
\newcommand{\cL}{{\mathcal L}}
\newcommand{\cM}{{\mathcal M}}
\newcommand{\cN}{{\mathcal N}}
\newcommand{\cO}{{\mathcal O}}
\newcommand{\cP}{{\mathcal P}}
\newcommand{\cS}{{\mathcal S}}
\newcommand{\cT}{{\mathcal T}}
\newcommand{\cU}{{\mathcal U}}
\newcommand{\cV}{{\mathcal V}}
\newcommand{\cW}{{\mathcal W}}
\newcommand{\cY}{{\mathcal Y}}

% Macros for blackboard bold letters
\newcommand{\bZ}{{\mathbb Z}}
\newcommand{\bR}{{\mathbb R}}
\newcommand{\bC}{{\mathbb C}}
\newcommand{\bT}{{\mathbb T}}
\newcommand{\bN}{{\mathbb N}}
\newcommand{\bQ}{{\mathbb Q}}
\newcommand{\bF}{{\mathbb F}}

% Other macros
\renewcommand{\i}{\infty}
\DeclareMathOperator{\ran}{ran}
\DeclareMathOperator{\sinc}{sinc}
\DeclareMathOperator{\spann}{span}
\DeclareMathOperator{\Bil}{Bil}
\DeclareMathOperator{\Hom}{Hom}
\DeclareMathOperator{\Endo}{End}

% \begin{document}
% ...
% \end{document}

% ==========================================================================
% The design of this document is inspired by Berkeley EECS
% The exam class creates cooler looking documents than the article class
\documentclass[letterpaper, 12pt]{exam} 
\usepackage[utf8]{inputenc}
\usepackage{amsmath,amssymb,amsthm}

% Format header and footer
% chead[T1]{T2} puts T1 on the first-page header and T2 on the other headers
% textsf yields san serif font
\pagestyle{headandfoot}
\extraheadheight[1.3in]{0in}
\extrafootheight{-.2in}
\rhead[\Large \textsf{Notes}\\
       \vspace{10pt}
]{}
\chead[\hrule height 3pt
       \vspace{10pt}
       \Large \textsf{Math 250A}\\
       \vspace{5pt}
       \Large \textsf{\topic}\\
       \vspace{10pt}
       \hrule height 3pt
]{}
\footer{}{\thepage}{}

% Change font for section and subsection headings to san serif
% The font sizes are kept default
\usepackage{titlesec}
\titleformat*{\section}{\Large\sffamily}
\titleformat*{\subsection}{\large\sffamily}
\titleformat*{\subsubsection}{\normalsize\sffamily}

% Format paragraphs by setting indents to zero and modifying the space between
% paragraphs
\usepackage{parskip}

% The LaTex exam class has the following bug:
% It states that margins should be 1-in on all sides for letterpaper.
% However, clearly the left margin is greater than the right margin.
% This problem does not persist in other types of paper (try a4paper)
% The hack fix is to include this package. I have no idea why.
\usepackage{hyperref}

% Theorem Environments
% Note: everything is un-numbered
\theoremstyle{definition} % Hack to make theorem environments not italicized
\newtheorem*{thm}{Theorem}
\newtheorem*{cor}{Corollary}
\newtheorem*{lem}{Lemma}
\newtheorem*{exercise}{Exercise}
\newtheorem*{dfn}{Definition}

% Be able to draw commutative diagrams
\usepackage{tikz-cd}

% Macros for Greek Letters
\renewcommand{\a}{\alpha}
\renewcommand{\b}{\beta}
\renewcommand{\d}{\delta}
\newcommand{\D}{\Delta}
\newcommand{\e}{\varepsilon}
\newcommand{\g}{\gamma}
\newcommand{\G}{\Gamma}
\renewcommand{\l}{\lambda}
\renewcommand{\L}{\Lambda}
\newcommand{\s}{\sigma}
\renewcommand{\th}{\theta}
\renewcommand{\o}{\omega}
\renewcommand{\O}{\Omega}
\renewcommand{\t}{\tau}
\newcommand{\var}{\varphi}
\newcommand{\z}{\zeta}

% Macros for math cal letters
\newcommand{\cA}{{\mathcal A}}
\newcommand{\cB}{{\mathcal B}}
\newcommand{\cC}{{\mathcal C}}
\newcommand{\cD}{{\mathcal D}}
\newcommand{\cE}{{\mathcal E}}
\newcommand{\cF}{{\mathcal F}}
\newcommand{\cH}{{\mathcal H}}
\newcommand{\cI}{{\mathcal I}}
\newcommand{\cK}{{\mathcal K}}
\newcommand{\cL}{{\mathcal L}}
\newcommand{\cM}{{\mathcal M}}
\newcommand{\cN}{{\mathcal N}}
\newcommand{\cO}{{\mathcal O}}
\newcommand{\cP}{{\mathcal P}}
\newcommand{\cS}{{\mathcal S}}
\newcommand{\cT}{{\mathcal T}}
\newcommand{\cU}{{\mathcal U}}
\newcommand{\cV}{{\mathcal V}}
\newcommand{\cW}{{\mathcal W}}
\newcommand{\cY}{{\mathcal Y}}

% Macros for blackboard bold letters
\newcommand{\bZ}{{\mathbb Z}}
\newcommand{\bR}{{\mathbb R}}
\newcommand{\bC}{{\mathbb C}}
\newcommand{\bT}{{\mathbb T}}
\newcommand{\bN}{{\mathbb N}}
\newcommand{\bQ}{{\mathbb Q}}
\newcommand{\bF}{{\mathbb F}}

% Other macros
\renewcommand{\i}{\infty}
\DeclareMathOperator{\ran}{ran}
\DeclareMathOperator{\sinc}{sinc}
\DeclareMathOperator{\spann}{span}
\DeclareMathOperator{\Bil}{Bil}
\DeclareMathOperator{\Hom}{Hom}
\DeclareMathOperator{\Endo}{End}

% \begin{document}
% ...
% \end{document}

% ==========================================================================
% The design of this document is inspired by Berkeley EECS
% The exam class creates cooler looking documents than the article class
\documentclass[letterpaper, 12pt]{exam} 
\usepackage[utf8]{inputenc}
\usepackage{amsmath,amssymb,amsthm}

% Format header and footer
% chead[T1]{T2} puts T1 on the first-page header and T2 on the other headers
% textsf yields san serif font
\pagestyle{headandfoot}
\extraheadheight[1.3in]{0in}
\extrafootheight{-.2in}
\rhead[\Large \textsf{Notes}\\
       \vspace{10pt}
]{}
\chead[\hrule height 3pt
       \vspace{10pt}
       \Large \textsf{Math 250A}\\
       \vspace{5pt}
       \Large \textsf{\topic}\\
       \vspace{10pt}
       \hrule height 3pt
]{}
\footer{}{\thepage}{}

% Change font for section and subsection headings to san serif
% The font sizes are kept default
\usepackage{titlesec}
\titleformat*{\section}{\Large\sffamily}
\titleformat*{\subsection}{\large\sffamily}
\titleformat*{\subsubsection}{\normalsize\sffamily}

% Format paragraphs by setting indents to zero and modifying the space between
% paragraphs
\usepackage{parskip}

% The LaTex exam class has the following bug:
% It states that margins should be 1-in on all sides for letterpaper.
% However, clearly the left margin is greater than the right margin.
% This problem does not persist in other types of paper (try a4paper)
% The hack fix is to include this package. I have no idea why.
\usepackage{hyperref}

% Theorem Environments
% Note: everything is un-numbered
\theoremstyle{definition} % Hack to make theorem environments not italicized
\newtheorem*{thm}{Theorem}
\newtheorem*{cor}{Corollary}
\newtheorem*{lem}{Lemma}
\newtheorem*{exercise}{Exercise}
\newtheorem*{dfn}{Definition}

% Be able to draw commutative diagrams
\usepackage{tikz-cd}

% Macros for Greek Letters
\renewcommand{\a}{\alpha}
\renewcommand{\b}{\beta}
\renewcommand{\d}{\delta}
\newcommand{\D}{\Delta}
\newcommand{\e}{\varepsilon}
\newcommand{\g}{\gamma}
\newcommand{\G}{\Gamma}
\renewcommand{\l}{\lambda}
\renewcommand{\L}{\Lambda}
\newcommand{\s}{\sigma}
\renewcommand{\th}{\theta}
\renewcommand{\o}{\omega}
\renewcommand{\O}{\Omega}
\renewcommand{\t}{\tau}
\newcommand{\var}{\varphi}
\newcommand{\z}{\zeta}

% Macros for math cal letters
\newcommand{\cA}{{\mathcal A}}
\newcommand{\cB}{{\mathcal B}}
\newcommand{\cC}{{\mathcal C}}
\newcommand{\cD}{{\mathcal D}}
\newcommand{\cE}{{\mathcal E}}
\newcommand{\cF}{{\mathcal F}}
\newcommand{\cH}{{\mathcal H}}
\newcommand{\cI}{{\mathcal I}}
\newcommand{\cK}{{\mathcal K}}
\newcommand{\cL}{{\mathcal L}}
\newcommand{\cM}{{\mathcal M}}
\newcommand{\cN}{{\mathcal N}}
\newcommand{\cO}{{\mathcal O}}
\newcommand{\cP}{{\mathcal P}}
\newcommand{\cS}{{\mathcal S}}
\newcommand{\cT}{{\mathcal T}}
\newcommand{\cU}{{\mathcal U}}
\newcommand{\cV}{{\mathcal V}}
\newcommand{\cW}{{\mathcal W}}
\newcommand{\cY}{{\mathcal Y}}

% Macros for blackboard bold letters
\newcommand{\bZ}{{\mathbb Z}}
\newcommand{\bR}{{\mathbb R}}
\newcommand{\bC}{{\mathbb C}}
\newcommand{\bT}{{\mathbb T}}
\newcommand{\bN}{{\mathbb N}}
\newcommand{\bQ}{{\mathbb Q}}
\newcommand{\bF}{{\mathbb F}}

% Other macros
\renewcommand{\i}{\infty}
\DeclareMathOperator{\ran}{ran}
\DeclareMathOperator{\sinc}{sinc}
\DeclareMathOperator{\spann}{span}
\DeclareMathOperator{\Bil}{Bil}
\DeclareMathOperator{\Hom}{Hom}
\DeclareMathOperator{\Endo}{End}

\begin{document}
\pagestyle{fancy}
\section{Haar Measure}
\begin{dfn}{Haar Measure}
A Haar measure is a left-multiplication invariant Borel measure defined on a locally compact topological group $G$. 
\end{dfn}
We describe how to compute Haar measures for $G$ naturally homeomorphic to open subsets $U \subset \bR^n$. The key is to guess that the Haar measure is a Radon measure absolutely continuous to the Lebesgue measure. In other words,
\[
    \phi(f) = \int_G f(x)w(x)dx.
\]
where $dx$ is the inherited Lebesgue measure on $G$. Invariance under left multiplication means
\[
    \phi(f) = \int_G f(x)w(x)dx = \int_G f(yx)w(x)dx = \phi(f(y\cdot \_)).
\]
Let $T$ be the change of variables $x \rightarrow yx$. When $G$ is identified with $U$, we can compute $S(x) = \text{det}(dT_x)$. Then our equation becomes
\[
    \int_G f(x)w(x)dx = \int_G f(x)w(y^{-1}x)|S(x)|dx.
\]
We can apply the change of variables formula because we are working in the Lebesgue measure. Then it remains to guess a $w$ such that
\[
    w(x) = w(y^{-1}x)|S(x)|.
\]
\begin{remark}
    Haar proved Haar measures are unique up to multiplication by a constant.
\end{remark}
\begin{remark}
    One can also derive right-invariant Haar measures by composing the left-invariant Haar measure with the group inverse.
\end{remark}
\end{document}
