\section{Lecture 4-16-2020}
We would like to study $\mathfrak m_\t$, $\mathfrak n_\t$. First, we give a basic overview of some properties of $\mathfrak m_\t$ and $\mathfrak n_\t$.
\begin{itemize}
    \item $\mathfrak m_\t$ and $\mathfrak n_\t$ are vector spaces. The first is by definition, the second follows from the parallelogram inequality for *-algebras:
        \[
            (S+T)^*(S+T) + (S-T)^*(S-T) = 2S^*S + 2T^*T.
        \]
    \item $\mathfrak n_\t$ is an left ideal. Key trick is to note $S^*TS \leq \|T\|S^*S$, which implies
        \begin{equation}
            \label{left ideal equation}
            \t\big((TS)^*(TS)\big) \leq \|T^*T\|\t(S^*S) = \|T\|^2\t(S^*S)
        \end{equation}
        as $\t$ is a positive operator. This equation will appear again in the next lecture.
    \item $\mathfrak m_\t$ is a *-subalgebra $\implies$ $\mathfrak n_\t$ is a *-subalgebra $\implies$ $\mathfrak n_\t$ is a two-sided ideal.
    \item By positive operator representation, $\mathfrak n_\t$ is a two-sided ideal $\implies$ $\mathfrak m_\t$ is is two-sided ideal.
\end{itemize}

Now we link trace and compactness.
\begin{lem}
    Let $T \in \cB(\cH)$. If $\t(|T|^n) < \i$ for some $n \in \bZ^+$, then $T$ is compact.
\end{lem}
\begin{details}{Key Trick}
    Observe compactness is preserved in the following links: $T \leftrightarrow |T| \leftrightarrow \sqrt{|T|^n}$. Use $\t(|T|^n) < \i$ to construct a finite rank projection $P$ such that
    \[
        \|\sqrt{|T|^n}-\sqrt{|T|^n}P\| < \e.
    \]
\end{details}
From the lemma, we deduce that $\mathfrak m_\t, \mathfrak n_\t  \subset \cB_c(\cH)$. Thus we can focus our attention to only compact operators.
\begin{thm}
    We have:
    \begin{itemize}
        \item $\mathfrak m_\t = \{T \in \cB_c(\cH): \t(|T|) < \i \} = \cB^1(\cH)$
        \item $\mathfrak n_\t = \{T \in \cB_c(\cH): \t(|T|^2) < \i \} = \cB^2(\cH)$
    \end{itemize}
\end{thm}
\begin{details}{Key Trick}
    For $m_\t$, use polar decomposition, the fact that $\mathfrak m_\t$ is an ideal, and the spectral theorem.
\end{details}
\begin{dfn}
    The operators in $\cB^2(\cH)$ are called Hilbert-Schmidt operators.
\end{dfn}
