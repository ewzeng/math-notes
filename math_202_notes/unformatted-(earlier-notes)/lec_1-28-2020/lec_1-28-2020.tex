\documentclass[12pt, letterpaper]{article}
\usepackage[utf8]{inputenc}
\usepackage{amsmath,amssymb}
\usepackage{xcolor} %For colored text
\usepackage{parskip}

%Title and margin formatting
\usepackage[left=2cm,top=3cm,bottom=3cm,right=2cm]{geometry} %Customize margins.
\usepackage{titling} %Customize the position of the title.
\setlength{\droptitle}{-50pt} %Raise the position of the title.

%Inkscape
\usepackage{import}
\usepackage{pdfpages}
\usepackage{transparent}

\newcommand{\incfig}[2][1]{%
    \def\svgwidth{#1\columnwidth}
    \import{./figures/}{#2.pdf_tex}
}

%Macros for Greek Letters
\renewcommand{\a}{\alpha}
\renewcommand{\b}{\beta}
\renewcommand{\d}{\delta}
\newcommand{\D}{\Delta}
\newcommand{\e}{\varepsilon}
\newcommand{\g}{\gamma}
\newcommand{\G}{\Gamma}
\renewcommand{\l}{\lambda}
\renewcommand{\L}{\Lambda}
\newcommand{\s}{\sigma}
\renewcommand{\th}{\theta}
\renewcommand{\o}{\omega}
\renewcommand{\O}{\Omega}
\renewcommand{\S}{\Sigma}
\renewcommand{\t}{\tau}
\newcommand{\var}{\varphi}
\newcommand{\z}{\zeta}

%Macros for math cal letters
\newcommand{\cA}{{\mathcal A}}
\newcommand{\cB}{{\mathcal B}}
\newcommand{\cC}{{\mathcal C}}
\newcommand{\cD}{{\mathcal D}}
\newcommand{\cE}{{\mathcal E}}
\newcommand{\cF}{{\mathcal F}}
\newcommand{\cH}{{\mathcal H}}
\newcommand{\cI}{{\mathcal I}}
\newcommand{\cK}{{\mathcal K}}
\newcommand{\cL}{{\mathcal L}}
\newcommand{\cM}{{\mathcal M}}
\newcommand{\cN}{{\mathcal N}}
\newcommand{\cO}{{\mathcal O}}
\newcommand{\cP}{{\mathcal P}}
\newcommand{\cS}{{\mathcal S}}
\newcommand{\cT}{{\mathcal T}}
\newcommand{\cU}{{\mathcal U}}
\newcommand{\cV}{{\mathcal V}}
\newcommand{\cW}{{\mathcal W}}
\newcommand{\cY}{{\mathcal Y}}

%Macros for blackboard bold letters
\newcommand{\bZ}{{\mathbb Z}}
\newcommand{\bR}{{\mathbb R}}
\newcommand{\bC}{{\mathbb C}}
\newcommand{\bT}{{\mathbb T}}
\newcommand{\bN}{{\mathbb N}}
\newcommand{\bQ}{{\mathbb Q}}
\newcommand{\bF}{{\mathbb F}}

%Other macros
\renewcommand{\i}{\infty}

\begin{document}

For vector space $V$, the following are equivalant:
\begin{itemize}
    \item The topology on $V$ is the initial topology from a collection of semi-norms on $V$.

    \item The topology on $V$ is translation invariant and has a subbase of convex sets.
\end{itemize}
When the above scernaios occur, we say the topology on $V$ is \textbf{locally convex}.

\noindent\rule{\textwidth}{1pt}

\textbf{Alaoglu's theorem.} If $V$ is a normed vector space, then the closed unit ball $B$ in $V'$ is compact for the weak-* topology.

\textbf{Point of confusion.} Two different topologies are used in the statement of the theorem. The closed unit ball is defined by the norm-topology, but the actual topology on $V'$ is weak-*.

\textbf{Proof gist.} Use Tynchoff's theorem to construct a compact space and establish a homeomorphism between that compact space and $B$ with the weak-* topology.

\textbf{More detail.} Define $D_v = \{t \in \bR: |t| < ||v|| \}$. Then consider the map
\[
    J: B \rightarrow \prod^{\i}_{v \in V}D_v = P
\]
given by the (informal) expression
\[
    J(\phi) \mapsto \prod^{\i}_{v \in V}\phi(v).
\]
By comparing the subbases of $B$ and $J(B)$, we conclude they are homeomorphic. To show $J(B)$ is closed (hence compact), we note each element of $P$ determines a function $f$. If $f$ is in the closure of $J(B)$, then by approximating $f$ by elements of $J(B)$, we show $f$ is linear and thus an element of $J(B)$. 

\end{document}
