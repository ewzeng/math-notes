\documentclass[12pt, letterpaper]{article}
\usepackage[utf8]{inputenc}
\usepackage{amsmath,amssymb}
\usepackage{xcolor} %For colored text
\usepackage{parskip}

%Title and margin formatting
\usepackage[left=2cm,top=3cm,bottom=3cm,right=2cm]{geometry} %Customize margins.
\usepackage{fancyhdr}
\setlength{\headheight}{15pt} %To avoid compiler warnings (12pt too small)
\fancyhead[L]{Edward Zeng, SID: 3034036984}
\fancyhead[C]{Math 202B Notes}
\fancyhead[R]{\today}

%Inkscape
\usepackage{import}
\usepackage{pdfpages}
\usepackage{transparent}

\newcommand{\incfig}[2][1]{%
    \def\svgwidth{#1\columnwidth}
    \import{./figures/}{#2.pdf_tex}
}

%Macros for Greek Letters
\renewcommand{\a}{\alpha}
\renewcommand{\b}{\beta}
\renewcommand{\d}{\delta}
\newcommand{\D}{\Delta}
\newcommand{\e}{\varepsilon}
\newcommand{\g}{\gamma}
\newcommand{\G}{\Gamma}
\renewcommand{\l}{\lambda}
\renewcommand{\L}{\Lambda}
\newcommand{\s}{\sigma}
\renewcommand{\th}{\theta}
\renewcommand{\o}{\omega}
\renewcommand{\O}{\Omega}
\renewcommand{\S}{\Sigma}
\renewcommand{\t}{\tau}
\newcommand{\var}{\varphi}
\newcommand{\z}{\zeta}

%Macros for math cal letters
\newcommand{\cA}{{\mathcal A}}
\newcommand{\cB}{{\mathcal B}}
\newcommand{\cC}{{\mathcal C}}
\newcommand{\cD}{{\mathcal D}}
\newcommand{\cE}{{\mathcal E}}
\newcommand{\cF}{{\mathcal F}}
\newcommand{\cH}{{\mathcal H}}
\newcommand{\cI}{{\mathcal I}}
\newcommand{\cK}{{\mathcal K}}
\newcommand{\cL}{{\mathcal L}}
\newcommand{\cM}{{\mathcal M}}
\newcommand{\cN}{{\mathcal N}}
\newcommand{\cO}{{\mathcal O}}
\newcommand{\cP}{{\mathcal P}}
\newcommand{\cS}{{\mathcal S}}
\newcommand{\cT}{{\mathcal T}}
\newcommand{\cU}{{\mathcal U}}
\newcommand{\cV}{{\mathcal V}}
\newcommand{\cW}{{\mathcal W}}
\newcommand{\cY}{{\mathcal Y}}

%Macros for blackboard bold letters
\newcommand{\bZ}{{\mathbb Z}}
\newcommand{\bR}{{\mathbb R}}
\newcommand{\bC}{{\mathbb C}}
\newcommand{\bT}{{\mathbb T}}
\newcommand{\bN}{{\mathbb N}}
\newcommand{\bQ}{{\mathbb Q}}
\newcommand{\bF}{{\mathbb F}}

%Other macros
\renewcommand{\i}{\infty}

\begin{document}
\pagestyle{fancy}

\textbf{Theorem.} The space of real measures $\cS$ forms a vector lattice, with the positive elements of the lattice being the finite positive measures.

\textbf{Proof gist.} For $\mu \in \cS$, we show that the total variation measure is $|\mu|$ (here, $| \cdot |$ is interpreted in the lattice sense). As we have seen before, this implies that $\cS$ is a lattice.

\textbf{More detail.} Let $\|\mu\|$ denote total variation measure. We show $\|\mu\|$ is countably additive by the usual argument (prove the inequality both ways, consider disjoint unions, etc). Then we show $\|\mu\|$ is finite by the following technicalitiy:
\begin{itemize}
    \item Real (and Banach) measures need to be absolutely convergent by definition, i.e. for disjoint sets $E_n$,
        \[
            \sum \mu(E_n) = \sum \mu(E_j)
        \]
        where $E_j$ represents a different ordering.

    \item If $\|\mu\|(E) = \i$ (i.e. $E$ is unbounded) but $\mu(E) < \i$, we construct a sequence of disjoint sets whose ``partial sums" are not absolutely convergent.
\end{itemize}
Finally we show $\|\mu\| = |\mu|$ by using the fact
\[
    \|\mu(E)\| = \sup\{ \mu(E_1) - \mu(E_2): E_1 \sqcup E_2 = E\}.
\]

\textbf{A final detail.} The construction of the sequence of disjoint sets mentioned above is as follows:
\begin{itemize}
    \item split $E$ into two disjoing unbounded sets $E_1, F_1$ where $|\mu(F_1)| > 1$.
    \item repeat for $E = E_1$
\end{itemize}
Our result is $F_1, F_2, F_3 \dots$.

\textbf{Remark.} It is not difficult to see that total variation is also a norm, and this norm plays well with the lattice. In other words, $\cS$ is a normed vector lattice.

\end{document}
