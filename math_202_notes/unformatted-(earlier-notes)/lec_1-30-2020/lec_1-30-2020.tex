\documentclass[12pt, letterpaper]{article}
\usepackage[utf8]{inputenc}
\usepackage{amsmath,amssymb}
\usepackage{xcolor} %For colored text
\usepackage{parskip}

%Title and margin formatting
\usepackage[left=2cm,top=3cm,bottom=3cm,right=2cm]{geometry} %Customize margins.
\usepackage{titling} %Customize the position of the title.
\setlength{\droptitle}{-50pt} %Raise the position of the title.

%Inkscape
\usepackage{import}
\usepackage{pdfpages}
\usepackage{transparent}

\newcommand{\incfig}[2][1]{%
    \def\svgwidth{#1\columnwidth}
    \import{./figures/}{#2.pdf_tex}
}

%Macros for Greek Letters
\renewcommand{\a}{\alpha}
\renewcommand{\b}{\beta}
\renewcommand{\d}{\delta}
\newcommand{\D}{\Delta}
\newcommand{\e}{\varepsilon}
\newcommand{\g}{\gamma}
\newcommand{\G}{\Gamma}
\renewcommand{\l}{\lambda}
\renewcommand{\L}{\Lambda}
\newcommand{\s}{\sigma}
\renewcommand{\th}{\theta}
\renewcommand{\o}{\omega}
\renewcommand{\O}{\Omega}
\renewcommand{\S}{\Sigma}
\renewcommand{\t}{\tau}
\newcommand{\var}{\varphi}
\newcommand{\z}{\zeta}

%Macros for math cal letters
\newcommand{\cA}{{\mathcal A}}
\newcommand{\cB}{{\mathcal B}}
\newcommand{\cC}{{\mathcal C}}
\newcommand{\cD}{{\mathcal D}}
\newcommand{\cE}{{\mathcal E}}
\newcommand{\cF}{{\mathcal F}}
\newcommand{\cH}{{\mathcal H}}
\newcommand{\cI}{{\mathcal I}}
\newcommand{\cK}{{\mathcal K}}
\newcommand{\cL}{{\mathcal L}}
\newcommand{\cM}{{\mathcal M}}
\newcommand{\cN}{{\mathcal N}}
\newcommand{\cO}{{\mathcal O}}
\newcommand{\cP}{{\mathcal P}}
\newcommand{\cS}{{\mathcal S}}
\newcommand{\cT}{{\mathcal T}}
\newcommand{\cU}{{\mathcal U}}
\newcommand{\cV}{{\mathcal V}}
\newcommand{\cW}{{\mathcal W}}
\newcommand{\cY}{{\mathcal Y}}

%Macros for blackboard bold letters
\newcommand{\bZ}{{\mathbb Z}}
\newcommand{\bR}{{\mathbb R}}
\newcommand{\bC}{{\mathbb C}}
\newcommand{\bT}{{\mathbb T}}
\newcommand{\bN}{{\mathbb N}}
\newcommand{\bQ}{{\mathbb Q}}
\newcommand{\bF}{{\mathbb F}}

%Other macros
\renewcommand{\i}{\infty}

\begin{document}

\textbf{Hahn-Banach Separation Theorem.} Let $V$ be a real normed vector space. Suppose $O$ and $C$ are disjoint convex subsets, with $O$ open. Then there exists a linear functional $\phi$ that separates $O$ and $C$, i.e.
\[
    \phi(O) < t \leq \phi(C)
\]
where $t$ is a constant.

\textbf{Proof gist.} Use the Hahn-Banach theorem to construct a functional $\phi$ such that
\[
    \phi(O-C) < 0.
\]

\textbf{In more detail.} $O-C$ is an open convex set not containing the origin, and let $v_0 \in O-C$. Pictorially, we have
\begin{figure}[ht]
    \centering
    \incfig[0.5]{A-convex-open-set}
    \caption{A convex open set}
    \label{fig:A-convex-open-set}
\end{figure}

If we define $\phi$ on $Rv_0$ with $\phi(-v_0) = 1$ and make the obvious extension (in this finite dimensional case), then clearly $\phi(O-C) < 0$. The general case is a more technical version of this.

\textbf{Extra detail.} To pick the appropriate gauge to make the extension in the general case, let $U = O - C - v_0$ and define
\[
    m_u(v) = \inf{\{s \in \bR^+: \frac{v}{s} \in U\}}.
\]
Then proceed as before, but use the gauge $m_u$ to perform the extension.

\newpage

\textbf{Hahn-Banach Extension for spaces over $\bC$.} Let $V$ be a complex normed vector space, $W$ a subspace, and $p$ a semi-norm. If $\phi$ is a continuous linear functional on $W$ dominated by $p$, then there exists a continuous linear extension $\tilde{\phi}$ on $V$ dominated by $p$.

\textbf{Proof gist.} Use the real Hahn-Banach theorem to construct $\tilde{\phi}$.

\textbf{More detail.} Let $\psi = \Re{\phi}$ on $W$, and pretend $V$ is a real vector space (i.e. $iv$ is not a multiple of $v$). Then use the real version of Hahn-Banach to extend $\psi$ to $\tilde{\psi}$ on $V$. Now define
\[
    \tilde{\phi}(v) = \tilde{\psi(v)} - i\tilde{\psi(iv)}
\]
and show $\tilde{\phi}$ satisfies the properties we want.
\end{document}
