\documentclass[12pt, letterpaper]{article}
\usepackage[utf8]{inputenc}
\usepackage{amsmath,amssymb}
\usepackage{xcolor} %For colored text
\usepackage{parskip}

%Title and margin formatting
\usepackage[left=2cm,top=3cm,bottom=3cm,right=2cm]{geometry} %Customize margins.
\usepackage{fancyhdr}
\setlength{\headheight}{15pt} %To avoid compiler warnings (12pt too small)
\fancyhead[L]{Edward Zeng, SID: 3034036984}
\fancyhead[C]{Math 202B Notes}
\fancyhead[R]{\today}

%Inkscape
\usepackage{import}
\usepackage{pdfpages}
\usepackage{transparent}

\newcommand{\incfig}[2][1]{%
    \def\svgwidth{#1\columnwidth}
    \import{./figures/}{#2.pdf_tex}
}

%Macros for Greek Letters
\renewcommand{\a}{\alpha}
\renewcommand{\b}{\beta}
\renewcommand{\d}{\delta}
\newcommand{\D}{\Delta}
\newcommand{\e}{\varepsilon}
\newcommand{\g}{\gamma}
\newcommand{\G}{\Gamma}
\renewcommand{\l}{\lambda}
\renewcommand{\L}{\Lambda}
\newcommand{\s}{\sigma}
\renewcommand{\th}{\theta}
\renewcommand{\o}{\omega}
\renewcommand{\O}{\Omega}
\renewcommand{\S}{\Sigma}
\renewcommand{\t}{\tau}
\newcommand{\var}{\varphi}
\newcommand{\z}{\zeta}

%Macros for math cal letters
\newcommand{\cA}{{\mathcal A}}
\newcommand{\cB}{{\mathcal B}}
\newcommand{\cC}{{\mathcal C}}
\newcommand{\cD}{{\mathcal D}}
\newcommand{\cE}{{\mathcal E}}
\newcommand{\cF}{{\mathcal F}}
\newcommand{\cH}{{\mathcal H}}
\newcommand{\cI}{{\mathcal I}}
\newcommand{\cK}{{\mathcal K}}
\newcommand{\cL}{{\mathcal L}}
\newcommand{\cM}{{\mathcal M}}
\newcommand{\cN}{{\mathcal N}}
\newcommand{\cO}{{\mathcal O}}
\newcommand{\cP}{{\mathcal P}}
\newcommand{\cS}{{\mathcal S}}
\newcommand{\cT}{{\mathcal T}}
\newcommand{\cU}{{\mathcal U}}
\newcommand{\cV}{{\mathcal V}}
\newcommand{\cW}{{\mathcal W}}
\newcommand{\cY}{{\mathcal Y}}

%Macros for blackboard bold letters
\newcommand{\bZ}{{\mathbb Z}}
\newcommand{\bR}{{\mathbb R}}
\newcommand{\bC}{{\mathbb C}}
\newcommand{\bT}{{\mathbb T}}
\newcommand{\bN}{{\mathbb N}}
\newcommand{\bQ}{{\mathbb Q}}
\newcommand{\bF}{{\mathbb F}}

%Other macros
\renewcommand{\i}{\infty}

\begin{document}
\pagestyle{fancy}
\textbf{Gauge.} Let $V$ be a vector space. A gauge is a function $p: V \rightarrow \bR$ that is ``half-linear", i.e.
\begin{itemize}
\item
For $r > 0$, we have $p(rv) = rp(v)$.
\item
$p(u + v) \leq p(u) + p(v)$.
\end{itemize}

\noindent\rule{\textwidth}{1pt}
\textbf{Main lemma for Hahn-Banach.} A linear functional defined on a subspace $W$ of $V$ subordinate to gauge $p$ can be extended to a subordinate linear functional defined on $W \oplus \text{span}(v_0)$.

\textbf{Proof gist.} Show the existence of $\alpha$ such that
\[
\tilde{\phi}(w + rv_0) = \phi(w) + r \alpha
\]
is subordinate to $p$. Key trick is a separation of variables.

\noindent\rule{\textwidth}{1pt}
\textbf{Hahn-Banach.} A linear functional defined on a subspace $W$ of $V$ subordinate to gauge $p$ can be extended to a subordinate linear functional defined on $V$.

\textbf{Proof gist.} The main lemma extends linear functionals one dimension at a time. Apply Zorn's lemma to it (by considering the family of pairs of vector subspaces and subordinate linear functionals defined on them).

\textbf{Remark.} To show the existence of continous linear functionals on normed vector spaces, let the gauge $p$ be the norm. Then any linear functional subordinate to $p$ is continuous (indeed, Lipschitz).

\end{document}
