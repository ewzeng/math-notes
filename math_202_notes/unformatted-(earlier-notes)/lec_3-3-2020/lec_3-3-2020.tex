\documentclass[12pt, letterpaper]{article}
\usepackage[utf8]{inputenc}
\usepackage{amsmath,amssymb}
\usepackage{xcolor} %For colored text
\usepackage{parskip}

%Title and margin formatting
\usepackage[left=2cm,top=3cm,bottom=3cm,right=2cm]{geometry} %Customize margins.
\usepackage{fancyhdr}
\setlength{\headheight}{15pt} %To avoid compiler warnings (12pt too small)
\fancyhead[L]{Edward Zeng, SID: 3034036984}
\fancyhead[C]{Math 202B Notes}
\fancyhead[R]{\today}

%Inkscape
\usepackage{import}
\usepackage{pdfpages}
\usepackage{transparent}

\newcommand{\incfig}[2][1]{%
    \def\svgwidth{#1\columnwidth}
    \import{./figures/}{#2.pdf_tex}
}

%Macros for Greek Letters
\renewcommand{\a}{\alpha}
\renewcommand{\b}{\beta}
\renewcommand{\d}{\delta}
\newcommand{\D}{\Delta}
\newcommand{\e}{\varepsilon}
\newcommand{\g}{\gamma}
\newcommand{\G}{\Gamma}
\renewcommand{\l}{\lambda}
\renewcommand{\L}{\Lambda}
\newcommand{\s}{\sigma}
\renewcommand{\th}{\theta}
\renewcommand{\o}{\omega}
\renewcommand{\O}{\Omega}
\renewcommand{\S}{\Sigma}
\renewcommand{\t}{\tau}
\newcommand{\var}{\varphi}
\newcommand{\z}{\zeta}

%Macros for math cal letters
\newcommand{\cA}{{\mathcal A}}
\newcommand{\cB}{{\mathcal B}}
\newcommand{\cC}{{\mathcal C}}
\newcommand{\cD}{{\mathcal D}}
\newcommand{\cE}{{\mathcal E}}
\newcommand{\cF}{{\mathcal F}}
\newcommand{\cH}{{\mathcal H}}
\newcommand{\cI}{{\mathcal I}}
\newcommand{\cK}{{\mathcal K}}
\newcommand{\cL}{{\mathcal L}}
\newcommand{\cM}{{\mathcal M}}
\newcommand{\cN}{{\mathcal N}}
\newcommand{\cO}{{\mathcal O}}
\newcommand{\cP}{{\mathcal P}}
\newcommand{\cS}{{\mathcal S}}
\newcommand{\cT}{{\mathcal T}}
\newcommand{\cU}{{\mathcal U}}
\newcommand{\cV}{{\mathcal V}}
\newcommand{\cW}{{\mathcal W}}
\newcommand{\cY}{{\mathcal Y}}

%Macros for blackboard bold letters
\newcommand{\bZ}{{\mathbb Z}}
\newcommand{\bR}{{\mathbb R}}
\newcommand{\bC}{{\mathbb C}}
\newcommand{\bT}{{\mathbb T}}
\newcommand{\bN}{{\mathbb N}}
\newcommand{\bQ}{{\mathbb Q}}
\newcommand{\bF}{{\mathbb F}}

%Other macros
\renewcommand{\i}{\infty}

\begin{document}
\pagestyle{fancy}

\textbf{Some observations.} Working in a lattice space allows us to use a few tricks:
\begin{itemize}
    \item We can decompose $x = x^+ - x^-$, so we need only study positive elements. We have already seen this many times.
    \item Given $x$, we can create elements like $x \vee 1$, $x \wedge 1$, etc. More generally, we can ``combine" multiple elements. This is a powerful property for construction (c.f. Stone Weierstrass).
\end{itemize}
The following toy example will show the power of the second trick. Then we will use the second trick to prove partitions of unity.

\textbf{Toy example (a generalization of Hahn Decomp).} Suppose $\mu, \nu$ are real measures such that $\mu \wedge \nu = 0$. Then $\mu$ and $\nu$ are mutually singular.

\textbf{Proof gist.} If not, construct a strictly positive measure smaller than both.

\textbf{More detail.} Afer applying Lebesgue Decomposition and then R-N, we get
\[
    \int_E h d\mu = \nu_{ac}(E).
\]
If $h \neq 0$ a.e., consider the strictly positive measure induced by $h \wedge 1$.

\noindent\rule{\textwidth}{1pt}

\textbf{Partitions of Unity for LCH Spaces.} If $X$ is LCH, then for any compact $C \subset X$ and open cover $\theta_i$ of $C$, there exists positive functions $f_i \in C_C(X)$ such that $\sum f_i = 1$ on $C$ and supp$(f_i) \subset \theta_i$.

\textbf{Remark.} Support is defined as the closure of the carrier.

\textbf{Proof gist.} Use LCH Urysohn's lemma (and friend) to construct a function
\[
    g = \sum g_i \geq 1.
\]
Then consider
\[
    \frac{g}{g \vee 1}.
\]

\textbf{Construction for first equation.} We first show there exists closed sets $B_i \subset \theta_i$ such that $C \subset \cup B_i$. (Need LCH Urysohn's friend). Then use LCH Urysohn's lemma to define $g_i(x) = 1$ for $x \in B_i$ and supp$(g_i) \subset \theta_i$.
\begin{itemize}
    \item \textit{LCH Urysohn's friend} (used to prove LCH Urysohn). $X$ LCH, subset $C$ compact, $C \subset U$ open. Then there exists open $V$, $\bar{V}$ compact, such that $C \subset V \subset \bar{V} \subset U$.
    \item \textit{LCH Urysohn's lemma}. If subset $C$ is compact and $C \subset \theta$ is open, then there exists a continuous $f: X \rightarrow [0,1]$ with $f(x) = 1$ for $x \in C$ and $f(x) = 0$ for $x \not \in \theta$.
\end{itemize}
Sorry, I come up with weird names.

\end{document}
