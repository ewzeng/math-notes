\documentclass[12pt, letterpaper]{article}
\usepackage[utf8]{inputenc}
\usepackage{amsmath,amssymb}
\usepackage{xcolor} %For colored text
\usepackage{parskip}

%Title and margin formatting
\usepackage[left=2cm,top=3cm,bottom=3cm,right=2cm]{geometry} %Customize margins.
\usepackage{fancyhdr}
\setlength{\headheight}{15pt} %To avoid compiler warnings (12pt too small)
\fancyhead[L]{Edward Zeng, SID: 3034036984}
\fancyhead[C]{Math 202B Notes}
\fancyhead[R]{\today}

%Inkscape
\usepackage{import}
\usepackage{pdfpages}
\usepackage{transparent}

\newcommand{\incfig}[2][1]{%
    \def\svgwidth{#1\columnwidth}
    \import{./figures/}{#2.pdf_tex}
}

%Macros for Greek Letters
\renewcommand{\a}{\alpha}
\renewcommand{\b}{\beta}
\renewcommand{\d}{\delta}
\newcommand{\D}{\Delta}
\newcommand{\e}{\varepsilon}
\newcommand{\g}{\gamma}
\newcommand{\G}{\Gamma}
\renewcommand{\l}{\lambda}
\renewcommand{\L}{\Lambda}
\newcommand{\s}{\sigma}
\renewcommand{\th}{\theta}
\renewcommand{\o}{\omega}
\renewcommand{\O}{\Omega}
\renewcommand{\S}{\Sigma}
\renewcommand{\t}{\tau}
\newcommand{\var}{\varphi}
\newcommand{\z}{\zeta}

%Macros for math cal letters
\newcommand{\cA}{{\mathcal A}}
\newcommand{\cB}{{\mathcal B}}
\newcommand{\cC}{{\mathcal C}}
\newcommand{\cD}{{\mathcal D}}
\newcommand{\cE}{{\mathcal E}}
\newcommand{\cF}{{\mathcal F}}
\newcommand{\cH}{{\mathcal H}}
\newcommand{\cI}{{\mathcal I}}
\newcommand{\cK}{{\mathcal K}}
\newcommand{\cL}{{\mathcal L}}
\newcommand{\cM}{{\mathcal M}}
\newcommand{\cN}{{\mathcal N}}
\newcommand{\cO}{{\mathcal O}}
\newcommand{\cP}{{\mathcal P}}
\newcommand{\cS}{{\mathcal S}}
\newcommand{\cT}{{\mathcal T}}
\newcommand{\cU}{{\mathcal U}}
\newcommand{\cV}{{\mathcal V}}
\newcommand{\cW}{{\mathcal W}}
\newcommand{\cY}{{\mathcal Y}}

%Macros for blackboard bold letters
\newcommand{\bZ}{{\mathbb Z}}
\newcommand{\bR}{{\mathbb R}}
\newcommand{\bC}{{\mathbb C}}
\newcommand{\bT}{{\mathbb T}}
\newcommand{\bN}{{\mathbb N}}
\newcommand{\bQ}{{\mathbb Q}}
\newcommand{\bF}{{\mathbb F}}

%Other macros
\renewcommand{\i}{\infty}

\begin{document}
\pagestyle{fancy}
In the next few lectures, we wish to study the dual space of $C_c(X)$, where $X$ is LCH. We call a positive linear functional on $C_c(X)$ a positive Radon measure (PRM). It is our end goal to characterize all PRMs as integrals (Riesz-Markov), so we begin by constructing a measure for every PRM. An overview of the construction is as follows: PRM $ \rightarrow $ content $ \rightarrow $ outer measure $ \rightarrow $ measure.

\noindent\rule{\textwidth}{1pt}

If $\phi$ is a PRM, define $\mu_\phi$ on open sets by
\[
    \mu_\phi(U) = \sup \{ \phi(f): 0 \leq f \leq \chi_U, \ \text{supp}\{f\} \subset U\}.
\]
Then $\mu_\phi$ is a \textbf{content}, i.e. it satisfies
\begin{itemize}
    \item If $U$ is open with $\bar{U}$ compact, then $\mu_\phi(U)$ is finite. This is because if $U$ is open with $\bar{U}$ compact, then $C_\i(U) \subset C_c(X)$.
    \item $\mu_\phi$ is monotone. Easy.
    \item $\mu_\phi$ is countably subadditive. Use partition of unity to break up a large function to one defined on many small domains.
    \item $\mu_\phi$ is finitely additive. Not hard.
\end{itemize}
A content can be thought about ``finitely additive measure." We extend content $\mu_\phi$ to an outer measure by defining
\[
    \mu_\phi^*(A) = \inf\{\mu_\phi(U): A \subset U\}.
\]
Countable subadditivity follows from definition and countable subadditivity of $\nu$. We then use Caratheodory's theorem to filter the outer measure into a measure, which we also denote as $\mu_\phi$ (note $\mu_\phi$ denotes both the content and the measure).

\end{document}
