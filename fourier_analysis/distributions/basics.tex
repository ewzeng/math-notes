This is a short note with material drawn from \textit{A Guide to Distribution Theory and Fourier Transforms} by Strichartz. It's a wonderful book.

\sepline

\section{Motivation}
Objects like the Dirac delta occur frequently in PDEs. Our goal is to rigorously define these ``generalized functions'' and define nice ways to manipulate them (e.g. taking derivatives or Fourier transforms). Among other things, this allows us to introduce these ``generalized functions'' as solutions to PDEs, giving us the ability to describe a wider class of physical scenarios.

\section{Generalized Functions}
Suppose $f(x)$ represents the temperature at position $x$. Because the bulb of a thermometer has nonzero size, if we try to measure the heat at $x = c$, we are really measuring the weighted average
\[
    \langle f, \var \rangle = \int f(x)\var(x)dx
\]
where $\var(x)$ is concentrated near $c$. Therefore, as long as the ``measurements'' $\langle f, \var \rangle$ behave nicely, we can drop the hypothesis that $f$ is a classical function and consider a larger class of objects. This gives us the freedom describe more physical situations (e.g. Dirac delta distribution of heat).

Our excursion suggests the following:
\begin{enumerate}
    \item We first define a space $\cT$ of test functions (the measuring tools).
    \item The space of generalized functions is then $\cT'$.
\end{enumerate}

Different choices of $\cT$ lead to different families of generalized functions. We consider two in particular.

\section{Distributions}
For open set $\O \subset \bR^n$, define $\cD(\O)$ to be the space of $C^\i$ functions with compact support that stays away from $\partial \O$. For the test functions $\cD(\O)$, the corresponding generalized functions $\cD'(\O)$ are called distributions. For $f \in \cD'(\O)$, we write
\[
    f(\var) = \langle f, \var \rangle.
\]

\section{Calculus of Distributions}
We now define the derivative of a distribution.

First observe $\cD(\O) \subset \cD'(\O)$ as any locally-integrable function is also a distribution.
Then using the trick that
\[
    \psi(x) =
    \begin{cases}
        e^{-1/x^2}                                                      & x > 0 \\
        0                                                               & x \le 0
    \end{cases}
\]
is $C^\i$ and generalizing to higher dimensions, one can construct test functions piecewise. This flexibility allows one to show $\cD(\O)$ is dense in $\cD'(\O)$ in the sense that for any $f \in \cD'(\O)$, there exists a sequence of $\var_n \in \cD(\O)$ such that
\[
    \var_n \to f
\]
in distribution. Therefore, it is natural to define $\frac{\partial}{\partial x}f$ as
\[
        \left\langle \frac{\partial}{\partial x}f, \var \right\rangle := \lim_{n \to \i} \left\langle \frac{\partial}{\partial x}\var_n, \var \right\rangle, \quad \forall \var \in \cD(\O).
\]
Note integration of parts implies
\[
    \begin{split}
        \lim_{n \to \i} \left\langle \frac{\partial}{\partial x}\var_n, \var \right\rangle & = \lim_{n \to \i} -\left\langle \var_n, \frac{\partial}{\partial x}\var \right\rangle\\
                                                                                           & = -\left\langle f, \frac{\partial}{\partial x}\var \right\rangle, \quad \forall \var \in \cD(\O)
    \end{split}
\]
so we have
\[
        \left\langle \frac{\partial}{\partial x}f, \var \right\rangle = -\left\langle f, \frac{\partial}{\partial x}\var \right\rangle, \quad \forall \var \in \cD(\O).
\]

This limiting definition works more generally. In fact, suppose we have the adjoint identity
\[
    \left\langle T\psi, \var \right\rangle = \left\langle \psi, S\var \right\rangle, \quad \forall \psi, \var \in \cD(\O)
\]
where $T, S: \cD(\O) \to \cD(\O)$ and $T$ is the operator we are interested in. Then the limiting definition implies
\begin{equation}
    \label{second-def}
    \left\langle Tf, \var \right\rangle = \left\langle f, S\var \right\rangle, \quad \forall \var \in \cD(\O).
\end{equation}
We can take (\ref{second-def}) as an alternative definition of $Tf$. The advantage of this definition is that it does not require the density condition that the limiting definition requires.

\section{Tempered Distributions}
Define the Schwartz space $\cS(\bR^n)$ to be the space of $C^\i$ functions that are rapidly decreasing, i.e. for any $\var \in \cS(\bR^n)$,
\[
    \var(x)p(x)
\]
is bounded for any polynomial $p(x)$. The corresponding space of distributions $\cS'(\bR^n)$ are called the tempered distributions. This is because classical functions in $\cS'(\bR^n)$ cannot blow up as fast as some classical functions in $\cD'(\bR^n)$.

\section{Fourier Transform of Tempered Distributions}
Our goal is to define the FT of elements in $\cS'(\bR^n)$ by using definition (\ref{second-def}). Thus, we need to show:
\begin{enumerate}
    \item FT maps $\cS(\bR^n) \to \cS(\bR^n)$.
    \item FT has an adjoint operator that also maps $\cS(\bR^n) \to \cS(\bR^n)$.
\end{enumerate}
The first item follows from the observation that:
\begin{itemize}
    \item Rapidly decreasing $\to$ Fourier transform smooth.
    \item Smooth $\to$ Fourier transform rapidly decreasing.
\end{itemize}
This is why we use $\cS(\bR^n)$. Note we cannot use $\cD(\bR^n)$ because the only element $\in \cD(\bR^n)$ whose FT is in $\cD(\bR^n)$ is 0.

Now observe we have the adjoint identity
\[
    \left\langle \hat{\psi}, \var \right\rangle =  \left\langle \psi, \hat{\var} \right\rangle, \quad \forall \psi, \var \in \cS(\bR^n)
\]
Hence we define the FT of $f \in \cS'(\bR^n)$ to be
\[
    \left\langle \hat{f}, \var \right\rangle := \left\langle f, \hat{\var} \right\rangle, \quad \forall \var \in \cS(\bR^n).
\]

%%% Local Variables:
%%% TeX-master: "main"
%%% End:
