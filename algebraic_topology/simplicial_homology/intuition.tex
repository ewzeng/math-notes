\section{Intuition}
The definition of homology groups may seem a little abstract, so before we begin, we provide some intuition behind homology groups. It may also be useful to read this section after reading the definitions of homology groups.

To study a topological object, we can consider the set of loops on the object, identifying two loops to be equivalent if we can deform one loop into another. This is a very natural mathematical structure to study, and very similar to the definition of the fundamental group (however, the fundamental group considers a family of loops based at a particular point).

We can reformulate the equivalence relation between loops from a different point of view. Suppose we have two loops, labeled $A$ and $B$, as shown:
\begin{figure}[H]
    \includegraphics[width=5cm]{figures/intuition-1}
    \centering
\end{figure}
The loops $A$ and $B$ are equivalent (i.e. can be deformed into one another) if and only if the region $C$ shown below has no ``holes":
\begin{figure}[H]
    \includegraphics[width=5cm]{figures/intuition-2}
    \centering
\end{figure}
As $C$ has no holes, one posisble oriented boundary of $C$, denoted $\partial C$, is:
\begin{figure}[H]
    \includegraphics[width=5cm]{figures/intuition-3}
    \centering
\end{figure}
Note if we ``add" loop $B$ and loop $\partial C$, the oppositely oriented segments cancel out, and we get loop $A$:
\begin{figure}[H]
    \includegraphics[width=5cm]{figures/intuition-4}
    \centering
\end{figure}
To summarize: loops $A$ and $B$ are equivalent if they ``differ" by the oriented boundary of some 2D region (which in this case is $C$). This observation turns out to be true with more complicated examples (although the region might no longer be a single connected component).

This reformulation of the equivalence relation turns out to be easy to formalize on simplicial complexes and easy to extend to higher dimensions. In fact, we will see:
\begin{itemize}
    \item $q$-cycle: higher dimension analog of loops
    \item $q$-boundary: higher dimension analog of (oriented) boundaries
    \item $q$-th homology group: the group of $q$-cycles, identifying two cycles to be equal if they ``differ" by a $q$-boundary
\end{itemize}
