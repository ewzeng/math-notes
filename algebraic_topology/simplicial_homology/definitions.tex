\section{Definitions}

\subsection{Orientation}

Define the orientation of a simplex $A$ as a ordering of the vertices of $A$. We say two orientations are equivalent if they differ by a even permutation of vertices.

Suppose the orientation of $A$ is $(v_1, v_2, \dots, v_q)$. Let $B$ be the face obtained by removing $v_i$. Define the induced orientation on $B$ to be:
\begin{itemize}
    \item $(v_1, \dots, \hat{v_i}, \dots, v_q)$ if $i$ is even.
    \item Else the opposite orientation.
\end{itemize}

\subsection{Chains, Cycles, and Boundaries}

For a simplicial complex $K$, define $C_q(K)$ to be the free abelian group generated by the oriented $q$-simplicies of $K$, with the relation $\s + \t = 0$ if $\s$ and $\t$ are opposite orientations of the same simplex. An element of $C_q(K)$ is called a \textbf{$q$-chain}.

Suppose $A$ is a $q$-simplex with orientation $(v_1, \dots, v_q)$. Define the \textbf{boundary map} $\partial$ to be:
\[
    \partial A = \sum_{i = 0}^q(-1)^q (v_0, \dots, \hat{v_i}, \dots, v_q)
\]
$\partial$ can be extended linearly to map from $C_q(K) \to C_{q-1}(K)$.

Define
\[
    Z_q(K) = \ker\left(\partial: C_q(K) \to C_{q-1}(K)\right),
\]
\[
    B_q(K) = \Ima\left(\partial: C_{q+1}(K) \to C_q(K)\right).
\]
Elements of $Z_q(K)$ are called \textbf{$q$-cycles} and elements of $B_q(K)$ are called \textbf{$q$-boundaries}. We define the \textbf{$q$-th homology group} to be
\[
    H_q(K) = Z_q(K) / B_q(K).
\]
(For an intuition behind this definition, see the previous section.) Of course, for this quotient to make sense, we need $B_q(K) \subset Z_q(K)$. This is equivalent to the statement ``the boundary of the boundary is zero'' which is not too difficult to verify.

