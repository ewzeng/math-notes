\section{Pointwise Convergence of Fourier Series}
When does a Fourier series converge pointwise to the original function? To answer this question, we need the following tool:
\begin{lem}[Riemann-Lebesgue]
    Let $f$ be a piecewise continuous function defined on $[-\pi,\pi]$. Then
    \[
        \int_{-\pi}^\pi f(x)e^{ikx}dx \rightarrow 0
    \]
    as $k \rightarrow \i$ or $k \rightarrow -\i$.
\end{lem}
\begin{details}{Proof gist}
    Prove for $f \in C^1$ using integration by parts. Then extend to continuous functions via uniform approximation by $C^1$ functions.
\end{details}
\begin{remark}
    The lemma still holds when we replace $e^{ikx}$ with $\sin(kx)$ or $\cos(kx)$.
\end{remark}
Now we are ready to state the result on pointwise convergence.
\begin{thm}
    \label{ptconv}
    The Fourier series of $f$ converges pointwise to itself if $f$ is $2\pi$-periodic and $C^1(\bR)$.
\end{thm}
\begin{details}{Proof gist}
    Use Riemann-Lebesgue to prove
    \[
        \Big| f(x) - \frac{1}{2\pi}\Big(\sum_{n=-k}^k \int_{-\pi}^\pi f(t)e^{-int}dt\Big)
        e^{inx}\Big| \rightarrow 0
    \]
    as $k \rightarrow \i$.
\end{details}
\begin{details}{Key trick}
    To evaluate the above expression, we introduce the Dirichlet kernel
    \[
        D_k(x) = \frac{1}{2\pi} \sum^{k}_{n=-k}e^{inx} = \frac{\sin(k+ \frac{1}{2})x}{\sin \frac{1}{2}x}
    \]
    and rewrite the expression as
    \[
        \Big|\int_{-\pi}^\pi \big(f(x)-f(t)\big) D_k(x-t)dt\Big|.
    \]
\end{details}
\begin{remark}
    We require $f$ to be $2\pi$-periodic and $f \in C^1(\bR)$ instead of just $f \in C^1([-\pi,\pi])$ to avoid some technical issues on the boundary.
\end{remark}
\begin{remark}
    Because of our formulation of Riemann-Lebesgue, Theorem \ref{ptconv} can be extended to continuous $2\pi$-periodic functions that are piecewise $C^1$ with no extra effort.
\end{remark}

