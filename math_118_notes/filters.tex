\section{Filters}
A filter takes an input signal and outputs an output signal. More formally, a filter is a map between function spaces, often $L^1$, $L^2$, or piecewise continuous functions. We are purposely vague on the precise domain and codomain, as they may differ depending on context.
\begin{dfn}
    A filter $L$ is
    \begin{itemize}
        \item[(a)] time-invariant if $(L[f])(x-a) = (L[f(\_-a)])(x)$.
        \item[(b)] causal if $f(x) = 0$ for $x < 0$ implies $L[f](x) = 0$ for $x < 0$.
    \end{itemize}
\end{dfn}
In practice, we only care about linear time-invariant causal filters. Here, we attempt an informal characterization of such filters.
\begin{exercise}
    Consider the convolutional filter $L: L^2 \rightarrow L^2$ defined as $L[f] = f * g$ for some fixed $g \in L^2$. Show $L$ is linear and time-invariant.
\end{exercise}
\begin{thm}
    Under some mild assumptions, all time-invariant linear filters are convolution filters.
\end{thm}
\begin{remark}
    We add ``mild assumptions" because we present an informal argument. We will not delve into the conditions that will make this argument formal.
\end{remark}
The motivation for this proof is to note that for any convolutional filter $L$, we have $g = L[\d]$, where $\d$ is the Dirac delta function.
\begin{details}{Proof gist}
    Show $L[f] = f * L[\d]$.
\end{details}
\begin{proof}
    \[
        \begin{split}
            (f * L[\d])(t) &= \int_{-\i}^\i f(x)L[\d](t-x)dx\\
                           &= \int_{-\i}^\i f(x)L[\d_x](t)dx\\
                           &= L[\int_{-\i}^{\i} f(x)\d_xdx](t)\\
                           &= L[f](t)
        \end{split}
    \]
    Here, the second equality comes from time invariance, and the third equality comes from linearity (intuitively, the integral can be seen as a sum).
\end{proof}
\begin{exercise}
    Show that a convolutional filter $L$ is causal iff $g(x) = 0$ a.e. when $x < 0$.
\end{exercise}

