\section{The Discrete Fourier Transform}
We now develop Fourier series in a different direction. Computers often compute integrals as Riemann sums, so the coefficents of Fourier series are often approximated as
\[
    c_k \approx a_k := \frac{1}{n} \sum^{n-1}_{j=0} f( \frac{2\pi j}{n}) e^{-ik \frac{2\pi j}{n}}.
\]
Thus given these $n$ sample points, we can approximately calculate all Fourier coefficents. This is the discrete Fourier transform. We express our finding more compactly in the following definition.
\begin{dfn}
    DFT is the following matrix multiplication:
    \[
        \begin{pmatrix}
            a_1 \\ a_2 \\ \vdots \\ a_n
        \end{pmatrix}
        =
        \frac{1}{n}
        \overline{F_n}
        \begin{pmatrix}
            f( 0) \\
            f( \frac{2\pi}{n}) \\
            \vdots\\
            f( \frac{2\pi(n-1)}{n})
        \end{pmatrix},
        \quad
        F_n =
        \begin{pmatrix}
            1   &   1   &   1   &   \dots   &   1\\
            1   &   w   &   w^2 &   \dots   &   w^{n-1}\\
            1   &   w^2 &   w^4 &   \dots   &   w^{2(n-1)}\\
            \vdots&\vdots&\vdots& \ddots    & \vdots\\
            \hdots&\hdots&\hdots&\hdots     & w^{(n-1)^2}
        \end{pmatrix},
        \quad
        w = e^{ \frac{2\pi i}{n}}.
    \]
\end{dfn}
\begin{remark}
    As the $a_i$'s are cyclic, we need not calculate more than $a_1, \dots, a_n$.
\end{remark}
One key property of DFT is that it can be computed in $\cO(n\log{n})$ time (using a divide-and-conquer technique called the FFT). This provides another reason why DFT is often preferable in practical situations.
\begin{remark}
    There are many FT analogs in DFT, but the proofs are much easier in the discrete case (as many technicalities disappear).
\end{remark}
