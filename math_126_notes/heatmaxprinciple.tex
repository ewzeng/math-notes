\subsection{Maximum Principle}
The maximum principle can be used to show uniqueness of the heat equation solution and continual dependence on data. Just consider the difference of two solutions.
\begin{thm}[Maximum Principle]
    The maximum temperature is always achieved on the boundary or initial conditions.
\end{thm}
\begin{remark}
    This makes physical sense. Heat spreads out.
\end{remark}
\begin{details}{Key trick}
    Consider the pertubation $u + \e x^2$ to remove a possible degenerate maximum.
\end{details}



