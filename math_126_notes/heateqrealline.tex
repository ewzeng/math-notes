\subsection{Heat Equation on $\bR$}
Because the heat equation is easiest to study without boundary conditions, we first study the heat equation on $\bR$.
\subsubsection{Fundamental Solution}
An important solution to the heat equation on $\bR$ for $t > 0$ is given by the Gaussian:
\[
    \Phi(x,t) = \frac{1}{\sqrt{4\pi kt}}e^{ \frac{-x^2}{4kt}}.
\]
$\Phi$ is called the kernel kernel/fundamental solution and is often derived via an Ansatz.
\begin{remark}
    $\Phi$ is undefined for $t = 0$. It is only a solution for $t > 0$ but it approaches the Dirac delta as $t \rightarrow 0$.
\end{remark}
\subsubsection{Initial Conditions}
Now suppose we are given the initial conditions $u(x,0) = g(x)$. First observe that $\Phi(x-y,t)g(y)$ is a solution to the heat equation. By linearity, this suggests that
\[
    \Phi * g = \int_\bR \Phi(x-y,t)g(y) dy
\]
is a solution to the heat equation.

Note $\Phi * g$ is not defined at $t = 0$. But because $\Phi$ limits to the Dirac delta as $t \rightarrow 0$, thus $\Phi * g \rightarrow g$ as $t \rightarrow 0$. It is in this sense that we have solved the IVP.

\subsubsection{Nonhomogeneous Equation}
The nonhomogeneous heat equation is
\[
    u_t = ku_{xx} + f(x,t).
\]
\begin{details}{Physical interpretation}
    Heat is being generated at a \textbf{rate} of $f(x,t)$ (in addition to regular heat flow).
\end{details}
This is a standard application of Duhamel's principle. The key idea behind Duhamel is that heat generation at a rate $r$ over the time period $[s,s+\e]$ can be approximated by adding
\[
    v(x,t,s) =
    \begin{cases}
        0   &   t < s\\
        \tilde{u}(x,t-s) & t > s,
    \end{cases}
\]
where $\tilde{u}(x,t)$ is the solution to the homogeneous heat equation with the initial condition $\tilde{u}(x,0) = \e r$. 

Thus, by linearity, the solution to nonhomogeneous heat equation with the initial condition $u(x,0) = 0$ can be well approximated by adding together the solutions of many homogeneous heat equations. We take the limit of this approximation, turn a Riemann sum into an integral, and get
\[
    u(x,t) = \int_0^t \int_\bR \Phi(x - y, t-s)f(y,s)dyds.
\]
