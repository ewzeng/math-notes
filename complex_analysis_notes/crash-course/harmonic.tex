\section{Harmonic Functions}
A function $f: \bR^2 \to \bR$ is harmonic if
\[
    \Delta f = \nabla^2 f = 0.
\]
Recall that a harmonic function represents an equilibrium state, e.g. heat equilibrium.

From the C-R equations, one can show that the real and imaginary parts of a holomorphic function are harmonic. Conversely, one can show that every harmonic function is locally the real part of a holomorphic function. This is why complex analysis is often used to study harmonic functions.

The Dirichlet problem concerns the construction of harmonic functions for given boundary values. For instance, the Dirichlet problem on $\bD \subset \bC$ is to find a harmonic function $g$ defined on $\bD$ such that
\[
    \lim_{r \to 1} g(re^{i\theta}) = f(\theta)
\]
for a given continuous $f$. We will solve the Dirichlet problem on the unit disk by considering the Fourier problem.

\section{Fourier Problem}

Given $f: [0, 2\pi) \to \bC$, we can decompose $f$ into its Fourier series
\[
    f(\theta) = \sum_{n = -\i}^\i \hat{f}(n)e^{in\theta}
\]
where the equality is convergence of the infinite sum in the $\cL^2$ Hilbert space. However, sometimes we want something stronger. More specifically, how can we recover $f$ \textit{pointwise} from its Fourier coefficents?

The naive solution is to define
\[
    (s_Nf)(\theta) = \sum_{n = -N}^N \hat{f}(n)e^{in\theta}
\]
and let $N \to \i$. However, this converges pointwise to $f$ only if $f$ is $C^1$ (when seen as a function on $S^1$). A better solution is to define for $0 \le r < 1$
\[
    (D_rf)(\theta) = \sum_{n = -\i}^\i r^{|n|}\hat{f}(n)e^{in\theta}
\]
and let $r \uparrow 1$. As multiplication in the Fourier domain is convolution, we rewrite $D_r$ as a convolution and get
\[
    D_r(f)(\theta) = \int_{-\pi}^\pi \frac{1 -r^2}{1 - 2r\cos(\theta-t) + r^2}f(t)dt = \int_{-\pi}^\pi P_r(\theta-t)f(t)dt
\]
where $P_r$ is called the \textbf{Poisson kernel}. Then a divide-and-conquer computation of the integral
\[
    |D_r(f)(\theta) - f(t)| \le \int P_r(\theta - t)|f(\theta)-f(t)|dt
\]
yields that $D_r(f) \to f$ uniformly as $r \uparrow 1$ when $f$ is continuous. (Another perspective: $P_r(\xi)$ converges to the Dirac delta as $r \uparrow 1$.)

\section{Dirichlet Problem on the Unit Disk}
Remarkably, it happens that we can solve the Dirichlet problem on the unit disk by setting
\[
    g(re^{i\theta}) = D_r(f)(\theta).
\]

%%% Local Variables:
%%% TeX-master: "main"
%%% End: