\section{Holomorphic Functions}
Complex analysis is the study of functions $f: \bC \to \bC$.

Call $f$ differentiable (or \textbf{holomorphic}) on open set $U \subset \bC$ if for all $x \in U$, there exists $f'(x)$ s.t.
\[
    f(x + h) = f(x) + f'(x)h + \mathfrak r(h), \quad \lim_{h \to 0} \frac{\mathfrak r(h)}{h} = 0.
\]
In other words, seen as a function $\bR^2 \to \bR^2$, the Jacobian of $f$ represents multiplication by a complex number (i.e. scale + rotate). That is,
\[
    \text{Jacobian}(f) =
    \begin{pmatrix}
        a & -b \\
        b  & a \\
    \end{pmatrix},
    \quad f'(x) = a + bi.
\]
This is equivalently expressed by the Cauchy-Riemann equations.

Conversely, if the Jacobian exists and has this form, $f$ is holomorphic. Thus satisfying C-R equations + additional mild conditions (for Jacobian existence) implies complex differentiability.

\section{Power Series}
Define a formal power series (around 0) to be a series of the form
\[
    \sum_{n = 0}^\i a_nz^n.
\]
A formal power series represents a function $f(z)$ inside the radius of convergence, which one can show via the comparsion test is
\[
    \frac{1}{r} = \limsup_{n \to \i} |a_n|^{\frac{1}{n}}.
\]
Inside the radius of convergence, formal power series can be manipulated formally (i.e. like polynomials). That is, adding, multiplying, inverting, differentiating, integrating, and composing power series formally will correspond to adding, multiplying... the corresponding functions they represent. The reason is because inside the radius of convergence formal power series converge absolutely and uniformly, which allows us to exchange limits (and thus makes power series act as polynomials).

We can also consider formal power series centered at $z_0$, not necessarily 0:
\[
    \sum_{n = 0}^\i a_n(z-z_0)^n.
\]
The theory is the same.

Call a function $\textbf{analytic}$ if at every point $x$ in domain, the function can be represented as a power series (centered at $x$) in a neighborhood of $x$. In other words, the Taylor expansion around a point converges locally to the function itself.

\section{Integrals}
Suppose $f: U \to \bC$ is continuous, and $\g \subset U$ is a smooth path. We define the complex path integral to be
\begin{empheq}[box=\mybluebox]{equation*}
    \int_\g fdz = \int_a^b f(\g(t))\g'(t)dt
\end{empheq}
where $\g(t): [a, b] \to \bC$ is a parameterization of $\g$. Via change of variables, we can show that this value of the integral is independent of parameterization. Furthermore, we can extend to piecewise smooth curves by summing over each smooth segment.

\begin{thm}[Goursat]
    Let $T$ be a triangle and let $f$ be a function holomorphic on $R$. Then
    \[
        \int_{\partial T} f = 0.
    \]
\end{thm}
\begin{proof}
    Subdivision trick.
\end{proof}

By approximating any path with polygonal paths, one can show via Goursat that for any closed path $P$, if $f$ is holomorphic on $P$ and in the interior of $P$, then the integral of $f$ over $P$ is 0. In fact, one can show that any two homotopic paths (in the domain of $f$) have the same integral.

This allows us to find the \textbf{primitive} $F$ (i.e. complex antiderivative) of any holomorphic function $f$ defined on a path-connected domain $U$. We pick $z_0 \in U$, set $F(z_0) = 0$, and define
\[
    F(z_0) = \int_\g f(z)dz
\]
where $\g$ is any path connecting $z_0$ and $z$.

The fact that any two homotopic paths have the same integral also allows us to prove the following remarkable fact using a shrinking circles argument:

\begin{thm}[Cauchy's Integral Formula]
    Let $D$ be a closed disc of positive radius, and let $f$ be holomorphic on $D$. Then for every $z_0 \in D$, we have
    \begin{empheq}[box=\mybluebox]{equation*}
        f(z_0) = \frac{1}{2\pi i}\int_{\partial D} \frac{f(z)}{z - z_0}dz.
    \end{empheq}
\end{thm}
In other words, the value of $f$ at $z_0$ is some weird weighted average of the values of $f$ on a circle surrounding $z_0$. This is an amazing fact (and perhaps reflects the fact that the real and imaginary components of $f$ satisfy Laplace's equation, which describes the equilibrium of heat).

By using a geometric series trick, one can use Cauchy's integral formula to show that holomorphic functions are analytic, and thus infinitely differentiable. Moreover, using Cauchy's integral formula, one can show that if an sequence of holomorphic functions $f_1, f_2, \dots$ converges uniformly on each compact $K$, then the limit will be holomorphic.

\section{Laurent Series and Singularities}
Define a Laurent series (around 0) is a series of the form
\[
    \sum_{n = -\i}^\i a_nz^n.
\]
\begin{thm}
    Suppose $f(z)$ is holomorphic for $r < |z| < R$ (i.e. an annulus). Then $f$ has a (unique) Laurent expansion which conveges uniformly and absolutely on $r + \e \le |z| \le R - \e$ for all $\e > 0$.
\end{thm}
Like power series, we can define Laurent series to be centered at $z_0$, not necessarily 0.

Suppose $f$ is holomorphic on $B(z_0,r) - z_0$, the open ball with the center removed. Then we say $f$ has a \textbf{isolated singularity} at $z_0$. 
\begin{itemize}
    \item Call this singularity \textbf{removable} if the Laurent expansion around $z_0$ has no negative exponents (i.e. can define a value at $f(z_0)$ to make $f$ holomorphic).
    \item Call this singularity a \textbf{pole} of order $n$ if the largest negative exponent of the Laurent expansion is $-n$.
    \item Call this singularity an \textbf{essential} singularity if it is a pole of $-\i$. By Casorati-Weierstrass (proof: elegant, short, but tricky), the image of any neighborhood of an essential singularity is dense in the complex numbers (so things really blow up).
\end{itemize}

\section{Calculus of Residues}
Given a Laurent expansion of $f$ centered at $z_0$, define $\Res(f, z_0)$ to be the coefficent of $(z - z_0)^{-1}$. Observe that for a circle $C$ centered at the origin we have
\[
    \int_{C} z^n dz =
    \begin{cases}
        0      & n \neq -1 \\
        2\pi i & n = -1
    \end{cases}.
\]
Thus one can conclude that for a circle centered at $z_0$, we have
\[
    \int_C f(z)dz = 2\pi i\Res(f,z_0).
\]
Using homotopy and the shrinking paths trick, we can generalize our result to the following:
\begin{thm}[Residue Formula]
    Let $f$ be holomorphic on the interior of a closed curve $\g$ except at points $z_0, \dots, z_n$. Then
    \begin{empheq}[box=\mybluebox]{equation*}
        \int_{\g} f(z)dz = 2\pi i\sum_j \Res(f, z_j).
    \end{empheq}
\end{thm}

One important application of the residue formula (and complex analysis in general) is the computation of definite real integrals, often of the form
\[
    \int_{-\i}^\i g(x)dx.
\]
One way to compute this (and similar integrals) is to extend $g$ to part of the complex plane, compute the integral over a nice closed path that gets bigger and bigger, show the integral over certain paths limit to 0, and note that the remaining part of the path is the desired integral (or something close).

%%% Local Variables:
%%% TeX-master: "main"
%%% End: