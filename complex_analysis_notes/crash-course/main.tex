\documentclass[letterpaper]{article} 
\usepackage[utf8]{inputenc}
\usepackage{amsmath,amssymb,amsthm}
\usepackage{hyperref}

% TOC
\usepackage{tocloft}
\renewcommand{\cftsecleader}{\cftdotfill{\cftdotsep}}

% Cool looking font.
\usepackage{newpxmath}
\usepackage{newpxtext}

% Format paragraphs by setting indents to zero and modifying the space
% between paragraphs.
\usepackage{parskip}

% Create nice looking boxes for equations. Rule of thumb: don't use
% for more than 10% of the equations. Usage:
% \begin{empheq}[box=\mybluebox]{equation*}
%     [Equation here]
% \end{empheq}
\usepackage{empheq, xcolor}
\definecolor{myblue}{rgb}{.8, .8. 1}
\newcommand*\mybluebox[1]{%
    \colorbox{myblue}{\hspace{1em}#1\hspace{1em}}}

% Theorem Environments.
% Note: everything is un-numbered.
\theoremstyle{definition} % Hack to make theorem environments not
                          % italicized.
\newtheorem*{thm}{Theorem}
\newtheorem*{cor}{Corollary}
\newtheorem*{lem}{Lemma}
\newtheorem*{exercise}{Exercise}
\newtheorem*{dfn}{Definition}

%Macros for Greek Letters
\renewcommand{\a}{\alpha}
\renewcommand{\b}{\beta}
\renewcommand{\d}{\delta}
\newcommand{\D}{\Delta}
\newcommand{\e}{\varepsilon}
\newcommand{\g}{\gamma}
\newcommand{\G}{\Gamma}
\renewcommand{\l}{\lambda}
\renewcommand{\L}{\Lambda}
\newcommand{\s}{\sigma}
\renewcommand{\th}{\theta}
\renewcommand{\o}{\omega}
\renewcommand{\O}{\Omega}
\renewcommand{\S}{\Sigma}
\renewcommand{\t}{\tau}
\newcommand{\var}{\varphi}
\newcommand{\z}{\zeta}

%Macros for math cal letters
\newcommand{\cA}{{\mathcal A}}
\newcommand{\cB}{{\mathcal B}}
\newcommand{\cC}{{\mathcal C}}
\newcommand{\cD}{{\mathcal D}}
\newcommand{\cE}{{\mathcal E}}
\newcommand{\cF}{{\mathcal F}}
\newcommand{\cH}{{\mathcal H}}
\newcommand{\cI}{{\mathcal I}}
\newcommand{\cK}{{\mathcal K}}
\newcommand{\cL}{{\mathcal L}}
\newcommand{\cM}{{\mathcal M}}
\newcommand{\cN}{{\mathcal N}}
\newcommand{\cO}{{\mathcal O}}
\newcommand{\cP}{{\mathcal P}}
\newcommand{\cS}{{\mathcal S}}
\newcommand{\cT}{{\mathcal T}}
\newcommand{\cU}{{\mathcal U}}
\newcommand{\cV}{{\mathcal V}}
\newcommand{\cW}{{\mathcal W}}
\newcommand{\cY}{{\mathcal Y}}

%Macros for blackboard bold letters
\newcommand{\bZ}{{\mathbb Z}}
\newcommand{\bR}{{\mathbb R}}
\newcommand{\bC}{{\mathbb C}}
\newcommand{\bT}{{\mathbb T}}
\newcommand{\bN}{{\mathbb N}}
\newcommand{\bQ}{{\mathbb Q}}
\newcommand{\bF}{{\mathbb F}}
\newcommand{\bE}{{\mathbb E}}
\newcommand{\bP}{{\mathbb P}}
\newcommand{\bD}{{\mathbb D}}

%Other macros
\renewcommand{\i}{\infty}
\DeclareMathOperator{\Res}{Res}

\title{Complex Analysis Crash Course}
\author{Edward Zeng \\
        \href{mailto:edwardzeng@berkeley.edu}{edwardzeng@berkeley.edu}}
\date{Original: May 2021 \\
      Last Updated: July 2022}

\begin{document}
\maketitle
\setcounter{tocdepth}{0}
\tableofcontents
\newpage

\part{Basics}
\section{Starting Principles}

The state of a quantum system $\cS$ is represented by a vector $\psi$ in the appropriate Hilbert space $\cH$. Two vectors that differ by a scalar represent the same state.
\begin{itemize}
    \item $\cH$ is called the \textbf{state space}.
    \item In classical mechanics, a state represents a \textbf{configuration} of the system. In quantum mechanics, a state instead encodes a probability distribution of configurations. More precisely, each $\psi \in \cH$ is a function on the space of configurations, and the probability of observing a set of configurations $D$ in a state $\psi$ is
    \[
        \frac{\int_D |\psi|^2}{\|\psi\|_2^2}.
    \]
    \item Very often, $\cH$ is the space of (complex) $\cL^2$ functions on the configuration space.
\end{itemize}

A \textbf{classical observable} of $\cS$ (i.e. a physical quantity that can be observed by taking a measurement of $\cS$) is a $\bR$-function $f$ on the configuration space. To each classical observable, there exists a corresponding \textbf{quantum observable} $\hat{f}$ which is a self-adjoint operator on $\cH$.

If $\cS$ is in state represented by unit vector $\psi \in \cH$, then the expected value of measuring $f$ is
\[
    \mathbb E[f] = \langle \psi, \hat{f} \psi \rangle.
\]
This is the key relationship between $f$ and $\hat{f}$. Notice we are always measuring a classical observable. However, in a quantum state, a classical observable no longer takes a definite value; the corresponding quantum observable encodes the probabilities of what values the classical observable can take.

\subsection{Eigenvalues}

Suppose quantum observable $\hat{f}$ has an orthonomormal basis of eigenvectors $e_i$ with eigenvalues $\l_i$. Furthermore, suppose $\cS$ is in a state represented by unit vector $\psi \in \cH$, where
\[
    \psi = \sum a_i e_i.
\]
Note that
\[
    \mathbb E[f] = \sum \l_i |a_i|^2, \quad \sum |a_i|^2 = 1.
\]
This \underline{suggests but does not imply} that we only observe the $\l_i$'s when we measure $f$, with
\[
    \mathbb P(f = \l_i) = |a_i|^2.
\]
However, even though it is not implied, the above statement happens to be always true (indeed, it is the most natural). As a result, the possible values of $f$ are the eigenvalues of $\hat{f}$. Put more directly, \textbf{we only observe eigenvalues}.

To generalize this notion in the case where $\hat{f}$ does not have an orthonormal basis of eigenvectors, we will need introduce the \textbf{spectrum} and use the spectral theorem for unbounded operators. In this general setting, \textbf{we only observe the spectrum}, with probabilities given by spectral projections.

\subsection{Example}
A simple quantum system $\cS$ is a particle on the real line. In this case, we have:
\begin{itemize}
    \item Configuration space: $\bR$. Each configuration is a possible position of the particle.
    \item State space: $\cL^2(\bR)$.
    \item Common quantum observables are the position operator $X\psi(x) = x\psi(x)$ and the momentum operator $P\psi(x) = -i\hbar\frac{d\psi}{dx}$. Indeed, observe if $\psi$ is a unit vector, then
    \[
        \langle \psi, X\psi \rangle = \int_\bR x|\psi(x)|^2dx = \mathbb E[\text{position of particle}].
    \]
\end{itemize}
One may have noticed that the position and momentum operators are not defined on the whole domain $\cH = \cL^2(\bR)$. Also, one may recall that all self-adjoint operators are bounded, and the position and momentum are clearly not bounded. What's going on here?

It happens that the position and momentum operators can be defined on a dense subset of $\cH = \cL^2(\bR)$, and there is a notion of self-adjointness for such operators. In fact, we generalize the definition of linear operators to include such functions.

Also observe the position operator does not have an orthonormal basis of eigenvectors (in fact, it has none!). However, the spectrum of the position operator is $\bR$, implying that the particle may be observed anywhere on the real line $\bR$. (Something similar happens with the momentum operator).

%%% Local Variables:
%%% TeX-master: "main"
%%% End:

\newpage
\part{Harmonic Functions}
\section{Harmonic Functions}
A function $f: \bR^2 \to \bR$ is harmonic if
\[
    \Delta f = \nabla^2 f = 0.
\]
Recall that a harmonic function represents an equilibrium state, e.g. heat equilibrium.

From the C-R equations, one can show that the real and imaginary parts of a holomorphic function are harmonic. Conversely, one can show that every harmonic function is locally the real part of a holomorphic function. This is why complex analysis is often used to study harmonic functions.

The Dirichlet problem concerns the construction of harmonic functions for given boundary values. For instance, the Dirichlet problem on $\bD \subset \bC$ is to find a harmonic function $g$ defined on $\bD$ such that
\[
    \lim_{r \to 1} g(re^{i\theta}) = f(\theta)
\]
for a given continuous $f$. We will solve the Dirichlet problem on the unit disk by considering the Fourier problem.

\section{Fourier Problem}

Given $f: [0, 2\pi) \to \bC$, we can decompose $f$ into its Fourier series
\[
    f(\theta) = \sum_{n = -\i}^\i \hat{f}(n)e^{in\theta}
\]
where the equality is convergence of the infinite sum in the $\cL^2$ Hilbert space. However, sometimes we want something stronger. More specifically, how can we recover $f$ \textit{pointwise} from its Fourier coefficents?

The naive solution is to define
\[
    (s_Nf)(\theta) = \sum_{n = -N}^N \hat{f}(n)e^{in\theta}
\]
and let $N \to \i$. However, this converges pointwise to $f$ only if $f$ is $C^1$ (when seen as a function on $S^1$). A better solution is to define for $0 \le r < 1$
\[
    (D_rf)(\theta) = \sum_{n = -\i}^\i r^{|n|}\hat{f}(n)e^{in\theta}
\]
and let $r \uparrow 1$. As multiplication in the Fourier domain is convolution, we rewrite $D_r$ as a convolution and get
\[
    D_r(f)(\theta) = \int_{-\pi}^\pi \frac{1 -r^2}{1 - 2r\cos(\theta-t) + r^2}f(t)dt = \int_{-\pi}^\pi P_r(\theta-t)f(t)dt
\]
where $P_r$ is called the \textbf{Poisson kernel}. Then a divide-and-conquer computation of the integral
\[
    |D_r(f)(\theta) - f(t)| \le \int P_r(\theta - t)|f(\theta)-f(t)|dt
\]
yields that $D_r(f) \to f$ uniformly as $r \uparrow 1$ when $f$ is continuous. (Another perspective: $P_r(\xi)$ converges to the Dirac delta as $r \uparrow 1$.)

\section{Dirichlet Problem on the Unit Disk}
Remarkably, it happens that we can solve the Dirichlet problem on the unit disk by setting
\[
    g(re^{i\theta}) = D_r(f)(\theta).
\]

%%% Local Variables:
%%% TeX-master: "main"
%%% End:

\newpage
\part{P\'olya Vector Fields}
We present a powerful interpretation of holomorphic functions and contour integration found in Chapter 11 of \cite{Nee97}.

\section{P\'olya Vector Fields}
Observe every $g: U \to \bC$ represents a vector field on $U \subset \bR^2$. Given $f: U \to \bC$, the \textbf{P\'olya vector field} $\tilde{f}$ of $f$ is the vector field represented by $\bar{f}$.

\section{Integrals}
Historically, complex path integrals came about in an attempt to generalize real integrals; complex integration was not created with a geometric (or physical) interpretation in mind. However, this does not mean there is no such interpretation.

Suppose $f: U \to \bC$ is continuous and $\g \subset U$ is a piecewise smooth path. Define $F[\tilde{f}, \g]$ to be the flux of $\tilde{f}$ through $\g$ and define $W[\tilde{f}, \g]$ to be the work done by moving along $\g$ in $\tilde{f}$. Then one can verify that
\begin{empheq}[box=\mybluebox]{equation*}
    \int_\g f dz = W[\tilde{f}, \g] + i F[\tilde{f}, \g].
\end{empheq}

\section{Holomorphic Functions} \label{polya-holo}
Let us suppose that $U$ is simply connected, $f: U \to \bC$, and $f$ is differentiable when regarded as a function $\bR^2 \supset U \to \bR^2$. Then
\begin{align*}
    \{f \text{ holomorphic on } U\} &\iff \{f \text{ satisfies C-R equations on } U\} \\
                                    &\iff \{\text{div}(\tilde{f}) = 0, \text{curl}(\tilde{f}) = 0 \text{ on }U\} \\
                                    &\iff \{F[\tilde{f}, \g] = 0, W[\tilde{f}, \g] = 0 \text{ for any closed loop } \g \subset U\} \\
                                    &\iff \{\tilde{f} \text{ is sourceless and conservative on } U\}.
\end{align*}
This provides an alternative characterization of holomorphic functions as well as an intuitive explantation of why homotopic paths in $U$ have the same integral.

\section{Laurent Series}
Suppose we can expand $f$ as the Laurent series
\[
    f(z) = \sum_{n = -\i}^\i a_n (z-z_0)^n.
\]
Then $\tilde{f}$ is a superposition of P\'olya vector fields, one for each term in the sum. However, one can show that the only term whose P\'olya vector field generates flux and/or circulation is $a_{-1} (z-z_0)^{-1}$. (For a visual illustration, see pages 488-492 in \cite{Nee97}). This provides an intuitive explanation for both Cauchy's integral formula and the residue formula.

\section{Harmonic Functions}
Continuing the discussion in Section \ref{polya-holo}, for $f = u + iv$, it is not hard to conclude
\[
    \{\tilde{f} \text{ is sourceless and conservative on } U\} \implies \{u, v \text{ harmonic on } U\}.
\]
By Gauss' mean value theorem, this implies the average of $\tilde{f}$ on a circle in $U$ is equal to the value of $\tilde{f}$ at the center of the circle.

What is a physical manifestation of this implication? Consider a 2D universe with a circular mass $c$ and a point mass $p$ that are far apart. The 2D gravity vector field induced by $p$ is sourceless and conservative near $c$. In this scenario, Gauss's mean value theorem is Newton's shell theorem, which states that $c$ affects $p$ gravitationally as if all the mass of $c$ is concentrated at its center.


\newpage
\bibliographystyle{alpha}
\bibliography{ref}

\end{document}
