\subsection{Existence of Primitives}
We show that holomorphic functions have primitives (anti-derivatives).
\begin{lem}[Goursat]
    Suppose $f$ is holomorphic on open set $\O$. Furthermore, suppose $\O$ contains the triangle contour $T$ and its interior. Then
    \[
        \int_T f(z) dz = 0.
    \]
\end{lem}
\begin{details}{Proof Overview}
    We split $T$ into the sum of smaller triangle contours (Figure \ref{fig:triangle_trick}), pick the one that yields the largest integral, and repeat. Thus we generate a sequence of nested triangles $T^n$ that converges to some point $z_0$. Note
    \[
        \Big| \int_T f(z) dz \Big| \leq 4^n \Big| \int_{T^n} f(z) dz \Big|.
    \]
    We then study the behavior of $f$ around $z_0$ to show RHS vanishes as $n \rightarrow \i$. That is, we note
    \[
        f(z) = f(z_0) + (z-z_0)f'(z_0) + \mathfrak e(z-z_0), \quad \mathfrak e(z) \ \text{sublinear.}
    \]
\end{details}
\begin{figure}[ht]
    \centering
    \incfig[0.3]{triangle_trick}
    \caption{Triangle Trick}
    \label{fig:triangle_trick}
\end{figure}
\begin{thm}[Existence of Primitives]
    \label{existence-of-primitives}
    Suppose $f$ is holomorphic on the open set $\O$ which has no holes. Then there exists $F$ defined on $\O$ such that $F' = f$.
\end{thm}
\begin{details}{Proof Overview}
    WLOG assume $\O$ is path-connected (else consider each path-connected component in turn) and $0 \in \O$. We construct $F$ by defining
    \[
        F(w) = \int_P f(z)dz
    \]
    where $P$ is any piecewise-linear path from $0$ to $w$. A easy generalization of Goursat to polygons shows this is well-defined. (Note to apply Goursat we need $\O$ to have no holes).
\end{details}

By noting existence of primitive $\implies$ integrals of loops vanish, Cauchy's theorem becomes a simple consequence of the above.
\begin{cor}[Cauchy's Theorem]
    Suppose $f$ is holomorphic on open set $\O$. If $\O$ contains contour $\gamma$ and its interior (defined by the Jordan Separation Theorem), then
    \[
        \int_\gamma f(z) dz = 0.
    \]
\end{cor}
Of course, another way to prove Cauchy's theorem is to consider piecewise linear approximations of $\gamma$ and applying Goursat.
\begin{remark}
    Cauchy's theorem is often used to compute integrals by integrating over an alternate path.
\end{remark}
